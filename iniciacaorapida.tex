\section{Como proceder a uma venda}


\subsection{Conceito de Sessão}

Todas as transacções se operam dentro de sessões válidas.

Quer isto dizer que fora do período de uma sessão, será impossível proceder a qualquer operação.

\subsection{Como iniciar uma Sessão}

Para se iniciar a sessão para começar a fazer vendas, é necessário ir até \textbf{Gerente -> Gestor de Sessão}

\includegraphics[height=3in]{../images/gestorSessao.png}

A partir daqui, deve-se escolher a sessão que se pretende iniciar da lista da esquerda de pressionar o botão \textbf{Iniciar Sessão}.

Quando a sessão inicia, deverá sair na impressora um papel com a informações:
\begin{itemize}
	\item nome da sessão,
	\item hora de início da sessão, 
	\item o nome de todos os funcionários que foram designados para trabalhar na sessão escolhida.
\end{itemize}

\subsection{Proceder a uma venda}

Neste momento já é possível abrir uma mesa. Selecione o tabulador \textbf{Zonas} e escolha uma mesa em que pretende trabalhar.

Na parte de cima terá os \textbf{Índices} e por baixo destes estarão as \textbf{Páginas}.

\includegraphics[height=3in]{../images/mesa.png}

Para colocar um artigo, bastará navegar nas diversas páginas e pressionar o botão correspondente ao artigo prentendido.

\subsection{Como cancelar uma artigo}

Consoante o nível de privilégios que tiver, poderá ser possível ou não proceder ao cancelamento de um artigo. Caso não tenha privilégios, por favor contacte o gerente de loja.

Caso o artigo já tenha sido pedido (as letras ficarão com a cor negra) e ao pressionar a tecla cancelar, irá aparecer a caixa de diálogo de desperdício. Esta caixa diz respeito à movimentação de stock.

\subsection{Como proceder a uma nota de crédito}

Caso tenha registado a venda, a única forma de proceder à anulação, é emitindo um documento de Nota de Crédito.
Para tal terá de ir até ao nível \textbf{Gerente -> História}, procurar o documento que gerou e proceder à emissão de uma Nota de Crédito.


\subsection{Como Fechar a sessão}

\subsection{Como extrair os relatórios}

Ao final da sessão de trabalho, pode-se finalmente proceder à extracção dos relatórios.

Para tal deve-se ir até ao nível gerente, e pressionar o botão \textbf{Relatórios Oficiais}.

Escolha a data e a Sessão correcta e pressione imprimir. 