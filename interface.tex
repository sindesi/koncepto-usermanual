\newpage
\section{Interface Gráfico}

% Todo este programa tenta ser o mais simples possível, contudo
% concordamos que este é um compromisso que nem sempre é simples de conseguir.

% Por esta razão passamos a apresentar o inteface de utilizador de forma a tornar o 
% ambiente o mais familiar possível, permitindo aos utilizadores usarem o programa de 
% uma forma fluida e natural.



\subsection{Setas de Selecção}

\begin{figure}[h]
\begin{center}
\includegraphics[height=5cm]{../images/setasSelecao.png}
\caption[Submanifold]{Setas de Selecção.}
\end{center}
\end{figure}


As setas de selecção permitem seleccionar um elemento de um lista.

As setas são representadas por triângulos em cor cinza no fundo da janela onde se encontra a lista. 

Pressionando as zonas onde se encontram os triângulos, poder-se-á navegar para cima e para baixo nos elementos da lista.

\subsection{Botão Definir}

\begin{figure}[h]
\begin{center}
\includegraphics[height=3cm]{../images/botaodefinir.png}
\caption[Submanifold]{Botão Definir.}
\label{botaodefinir}
\end{center}
\end{figure}


É muito comum aparecer botões de \emph{toggle} ou de comutação. 

Para facilitar o acesso a estes botões, vai encontrar sempre associado
a estas listas de opções um botão \keystroke{Definir}.

Cada vez que pressionar o botão \keystroke{Definir} irá marcar a opção 
como activa ou inactiva alternadamente.

Caso não tenha a certeza da alteração que pretende fazer, terá sempre 
disponível um botão que permite sair da configuração que está a fazer.





\newpage
\subsection{Teclados}

O interface do utilizador é sempre o ecrã, por este motivo é necessário que 
haja uma forma que se possa interagir directamente com o programa. 
Por isso haverá sempre disponível um teclado para todas as interacções do operador.

Estas opções são simplificadas ao máximo de forma a evitar o recurso aos mesmos,
 pois pretende-se que a operabilidade 
seja o mais célere possível.

Existem dois tipos de teclado disponíveis, o alfanumérico e o numérico, que estarão disponíveis
de acordo com as necessidades.


\subsubsection{Teclado alfanumérico}

\begin{figure}[h]
\begin{center}
\includegraphics[height=5cm]{../images/teclado.png}
\caption[Submanifold]{Representação do teclado alfanumérico no ecrã.}
\label{fig:alfanumerico}
\end{center}
\end{figure}



O teclado alfanumérico tem uma abrangência maior, e é invocado em quase todos os casos em
que se prevê a necessidade dele. O teclado está representado na Figura \ref{fig:alfanumerico}

Para colocar maiúsculas poderá pressionar a tecla \keystroke{CAPS} à esquerda, ou simplesmente pressionar
uma das teclas \keystroke{SHIFT} disponíveis em cada lado do teclado.

Se por alguma razão se enganou e pretender voltar ao menu anterior sem modificar os dados, bastará pressionar
a tecla \keystroke{ESC} que se encontra no canto superior esquerdo.

Para validar os dados submetidos, pressione a tecla \keystroke{$\hookleftarrow$} (ou ''Enter'') que aqui é representada como
um caractere de mudança de linha no lado direito do teclado.


\subsubsection{Teclado numérico}

\begin{figure}[h]
\begin{center}
\includegraphics[height=5cm]{../images/numerico.png}
\caption[Submanifold]{Representação do teclado numérico no ecrã.}
\end{center}
\end{figure}


O teclado numérico é utilizado para submissão de palavras chave de utilizadores, códigos de barra,
indicação de quantidades, preços, taxas, etc, ...

Este teclado conta com os botões:


 \begin{table}[ht]
%  \caption{Hardware} 
 \centering
\small
\def\arraystretch{1.5}
 \begin{tabular}{c p{12cm}}  %   l c r r } % centered columns 
 \textbf{Botão} & \textbf{Significado}  \\ % Garantia & Descrição & Qtd & C. Unit & Subtotal \\ [0.5ex]
%  \multicolumn{6}{l}{\textbf{Servidor}} \\			
 \hline
\keystroke{Atrás} & permite apagar o último caractere pressionado. \\
\keystroke{Cancelar} &  voltará ao menu anterior sem modificar qualquer dados. \\
\keystroke{OK} & Validará os dados colocados. \\
%  \multicolumn{6}{l}{\textbf{Bastidor}} \\			
 \hline
 \end{tabular}
 \end{table}







Os números premidos deverão aparecer sempre no ecrã, salvo duas excepções: a introdução de uma 
palavra passe, e a senha de identificação de um cartão cliente.






\subsection{Botão Propriedades}

\begin{figure}[h]
\begin{center}
\includegraphics[height=5cm]{../images/propriedades2.png}
\caption[Submanifold]{Propriedades de Botões.}
\label{propriedadesbotoes}
\end{center}
\end{figure}


Em diversos pontos do programa, pode-se encontrar o botão \keystroke{Propriedades}.

Esta opção permite personalizar botões de forma a que ganhem realce no interface gráfico. As características que se podem modificar são:
 % a secção em causa permitido desde uma simples troca de cor, aplicação de textura ou aplicação de uma imagem.



 \begin{table}[ht]
%  \caption{Hardware} 
 \centering
\small
\def\arraystretch{1.5}
 \begin{tabular}{c p{12cm}}  %   l c r r } % centered columns 
 \textbf{Botão} & \textbf{Significado}  \\ % Garantia & Descrição & Qtd & C. Unit & Subtotal \\ [0.5ex]
%  \multicolumn{6}{l}{\textbf{Servidor}} \\			
 \hline
 	\keystroke{Cor de Fundo} &  Muda a cor de fundo do botão. \\
	\keystroke{Textura de fundo} &  Associa uma imagem como padrão de fundo do botão. \\ 
	\keystroke{Cor de Texto} &  Altera a cor do texto. \\ 
	\keystroke{Letra} & Altera o tipo de letra dos caracteres do botão. \\ 
	\keystroke{Imagem} & Associa uma imagem ao botão. \\ 
	\keystroke{Limpar Imagem} & Desassocia qualquer imagem que esteja associada a este botão. \\ 
	\keystroke{Limpar Propriedades} & este botão limpa todas as características fazendo com que o botão fique com os valores por omissão. \\
%  \multicolumn{6}{l}{\textbf{Bastidor}} \\			
 \hline
 \end{tabular}
 \end{table}





% \begin{bclogo}[couleur=blue!10,arrondi=0.1,logo=\bctakecare,ombre=true]
\bcatencao{Tipos de letra e imagens}{
Quando fizer alterações, tenha presente que:
\begin{itemize}
\item os tipos de letra serão somente as disponibilizadas pelo sistema operativo, podendo em muitos casos não se encontrarem muitos tipos de letra instalados.
\item as imagens disponibilizadas são as imagens padrão incluídas no software, podendo ser aumentado mediante solicitação. 
\end{itemize}
}




% \subsection{Caixas}

% \subsection{Barras de Selecção}


