 \documentclass[a4paper,11pt,openany]{memoir}
%  \documentclass[a4paper,11pt,openany]{book}

%  \topmargin 0in
%  \textheight 7in
%  \textwidth 6in
%  \oddsidemargin 0in
%  \evensidemargin 0in
%  \headheight 1.2in % 7pt
%  \headsep 0.25in




\setlrmarginsandblock{2.5cm}{*}{1} 
\setulmarginsandblock{2.5cm}{2.5cm}{*}
\setmarginnotes{2.5mm}{2cm}{1em}
\checkandfixthelayout

%%%%%%%%%%% HEADER & FOOTER  %%%%%%%%%%%%%%%%%%%%%%%%%%%%%%%%%%%%%%%%%%%%%%%%%%%%%%%%
\makeevenhead{headings}%
    {\thepage}{}{\slshape\bookname~\thebook\qquad\partname~\thepart\qquad\leftmark}
\makeoddhead{headings}{\slshape\rightmark}{}{\thepage}
\makeevenfoot{headings}{}{}{(Copyright notice)}
\makeoddfoot{headings}{(Copyright notice)}{}{}

\copypagestyle{plainnotice}{plain}
\makeevenfoot{plainnotice}{\thepage}{}{(Copyright notice)}% not used with "openright"
\makeoddfoot{plainnotice}{(Copyright notice)}{}{\thepage}
\aliaspagestyle{chapter}{plainnotice}

\copypagestyle{headingsnobook}{headings}
\makeevenhead{headingsnobook}{\thepage}{}{\slshape\leftmark}



%%%%%%%%%%% HEADER & FOOTER  %%%%%%%%%%%%%%%%%%%%%%%%%%%%%%%%%%%%%%%%%%%%%%%%%%%%%%%%


%%%%%%%%%%% PACOTE BCLOGO %%%%%%%%%%%%%%%%%%%%%%%%%%%%%%%%%%%%%%%%%%%%%%%%%%%%%%%%
\usepackage[tikz]{bclogo}


\presetkeys{bclogo}{arrondi=0.3,ombre=true,couleurOmbre=black!60,blur}{}

\newcommand\dica{\includegraphics[width=1.2cm]{../small-n-flat-master/png/96/light-bulb.png}}
\newcommand\exemplo{\includegraphics[width=1.2cm]{../small-n-flat-master/png/96/notepad.png}}
\newcommand\atencao{\includegraphics[width=1.2cm]{../small-n-flat-master/png/96/post-it.png}}
\newcommand\perigo{\includegraphics[width=1.2cm]{../small-n-flat-master/png/96/shield-warning.png}}
\newcommand\relacionados{\includegraphics[width=1.2cm]{../small-n-flat-master/png/96/file-link.png}}


\newcommand{\bcdica}[2]{\vspace{5mm}\begin{bclogo}[couleur=blue!30,logo=\dica]{\hspace{0.7cm}#1}{#2}\end{bclogo}}
\newcommand{\bcexemplo}[2]{\vspace{5mm}\begin{bclogo}[couleur=green!30,logo=\exemplo]{\hspace{0.7cm}#1}{#2}\end{bclogo}}
\newcommand{\bcatencao}[2]{\vspace{5mm}\begin{bclogo}[couleur=yellow!30,logo=\atencao]{\hspace{0.7cm}#1}{#2}\end{bclogo}}
\newcommand{\bcperigo}[2]{\vspace{5mm}\begin{bclogo}[couleur=red!30,logo=\perigo]{\hspace{0.7cm}#1}{#2}\end{bclogo}}
\newcommand{\bcrelacionados}[2]{\vspace{5mm}\begin{bclogo}[logo=\relacionados]{\hspace{0.7cm}Mais sobre este assunto!}{#1}\end{bclogo}}
  


% \begin{bclogo}{logo=\dica}{Dica}{#1}}


% {logo=\dica} 
% \newcommand\bcexemplo{couleur=blue!20,arrondi=0.1,logo=\exemplo,ombre=true,couleurOmbre=black!60,blur}
% \newcommand\bcatencao{couleur=blue!20,arrondi=0.1,logo=\atencao,ombre=true,couleurOmbre=black!60,blur}
% \newcommand\bcperigo{couleur=blue!20,arrondi=0.1,logo=\perigo,ombre=true,couleurOmbre=black!60,blur}
% \newcommand\bcrelacionados{couleur=blue!20,arrondi=0.1,logo=\relacionados,ombre=true,couleurOmbre=black!60,blur}


% \newcommand\bc {\ includegraphics [ width =\ logowidth ]{ bc - fleur }}
% \presetkeys{bclogo}{ombre=true,epBord=1pt}{}

%%%%%%%%%%% PACOTE BCLOGO %%%%%%%%%%%%%%%%%%%%%%%%%%%%%%%%%%%%%%%%%%%%%%%%%%%%%%%%

% \usepackage[colorlinks]{hyperref}
 \usepackage{hyperref}
\usepackage{memhfixc}

%%%%%%%%%%%%%%%%%%%%%%%%%%%%%%%%%%%%%%%%%%%%%%%%%%%%%%%%%%%%%%%%

\usepackage{color,calc,graphicx,soul,fourier}
% \definecolor{nicered}{rgb}{.647,.129,.149}
\definecolor{nicered}{rgb}{.0,.0,.255}
\makeatletter
\newlength\dlf@normtxtw
\setlength\dlf@normtxtw{\textwidth}
\def\myhelvetfont{\def\sfdefault{mdput}}
\newsavebox{\feline@chapter}
\newcommand\feline@chapter@marker[1][4cm]{%
\sbox\feline@chapter{%
\resizebox{!}{#1}{\fboxsep=1pt%
\colorbox{nicered}{\color{white}\bfseries\sffamily\thechapter}%
}}%
\rotatebox{90}{%
\resizebox{%
\heightof{\usebox{\feline@chapter}}+\depthof{\usebox{\feline@chapter}}}%
{!}{\scshape\so\@chapapp}}\quad%
\raisebox{\depthof{\usebox{\feline@chapter}}}{\usebox{\feline@chapter}}%
}
\newcommand\feline@chm[1][4cm]{%
\sbox\feline@chapter{\feline@chapter@marker[#1]}%
\makebox[0pt][l]{% aka \rlap
\makebox[1cm][r]{\usebox\feline@chapter}%
}}
\makechapterstyle{daleif1}{
\renewcommand\chapnamefont{\normalfont\Large\scshape\raggedleft\so}
\renewcommand\chaptitlefont{\normalfont\huge\bfseries\scshape\color{nicered}}
\renewcommand\chapternamenum{}
\renewcommand\printchaptername{}
\renewcommand\printchapternum{\null\hfill\feline@chm[2.5cm]\par}
\renewcommand\afterchapternum{\par\vskip\midchapskip}
\renewcommand\printchaptertitle[1]{\chaptitlefont\raggedleft ##1\par}
}













\makeatother
\chapterstyle{daleif1}



%%%%%%%%%%%%%%%%%%%%%%%%%%%%%%%%%%%%%%%%%%%%%%%%%%%%%%%%%%%%%%%%

\usepackage{
  calc,
  graphicx,
  url,
  fancyvrb,
  multicol,
  kvsetkeys
}


%%%%%%%%%%%%%%%%%%%%%%%%%%%%%%%%%%%%%%%%
% Linguagem e Acentuacao
%%%%%%%%%%%%%%%%%%%%%%%%%%%%%%%%%%%%%%%%
\usepackage[portuges]{babel}
% \usepackage[latin1]{inputenc}      

%%%%%%%%%%%%%%%%%%%%%%%%%%%%%%%%%%%%%%%%
% Tipo de Letra
%%%%%%%%%%%%%%%%%%%%%%%%%%%%%%%%%%%%%%%%
\usepackage{helvet} \renewcommand{\familydefault}{\sfdefault}
% \usepackage[scaled]{uarial}


%%%%%%%%% BEGIN Indice Remissivo %%%%%%%%%%%%%%%%%%%
\usepackage{makeidx}
\makeindex

% Para criar um índice usamos o comando \index{chave}, 
% onde chave é a palavra de entrada do índice.
% Também podemos escrever \index{chave!subchave}.
% 
%  \index{Cor}
%  \index{Cor!Amarelo}
%
%  Para imprimir:  
%  \printindex
%  http://latexbr.blogspot.pt/2013/01/indice-remissivo-no-latex.html
%%%%%%%%% END Indice Remissivo %%%%%%%%%%%%%%%%%%%


%%%%%%%% BEGIN MACRO PARA COLOCAR CIRCULOS A EMULAR BOTOES NO TEXTO %%%%%%%%
%  The quick brown fox jumps over the lazy dog.
%   \keystroke{Strg} The quick brown fox jumps over the lazy dog.
%   \keystroke{Ctrl} The quick brown fox jumps over the lazy dog.
%   \keystroke{Page $\uparrow$} \keystroke{Esc} \keystroke{F1}

\newcommand*\keystroke[1]{%
  \tikz[baseline=(key.base)]
    \node[%
      draw,
      fill=white,
      drop shadow={shadow xshift=0.25ex,shadow yshift=-0.25ex,fill=black,opacity=0.75},
      rectangle,
      rounded corners=2pt,
      inner sep=1pt,
      line width=0.5pt,
      font=\scriptsize\sffamily
    ](key) {#1\strut}
  ;
}

%%%%%%%% END MACRO PARA COLOCAR CIRCULOS A EMULAR BOTOES NO TEXTO %%%%%%%%

%%%%%%%%%%%%%%%%%%%%%%%%%%%%%%%%%%%%%%%%
% Pacote de gráficos
%%%%%%%%%%%%%%%%%%%%%%%%%%%%%%%%%%%%%%%%
\usepackage{graphicx}

\usepackage{fancybox}

\usepackage[utf8]{inputenc}
\usepackage[T1]{fontenc}



%%%%%%%%%%%%%%%%%%%%%%%%%%%%%%%%%%%%%%%%
% Pacote de gráficos
%%%%%%%%%%%%%%%%%%%%%%%%%%%%%%%%%%%%%%%%
% \usepackage{graphicx}

\usepackage{boxedminipage}

 \usepackage{pdfpages}

% For (non-printing) notes  \PWnote{date}{text}
\newcommand{\PWnote}[2]{} 
\PWnote{2009/04/29}{Added fonttable to the used packages}
\PWnote{2009/08/19}{Made Part I a separate doc (memdesign.tex).}

% same
\newcommand{\LMnote}[2]{} 




%%%%%%%%%%%%%%%%%%%%%%%%%%%%%%%%%%%%%%%%%%%%%%%%%%%%%%%%%%%
% Header & Footer
%%%%%%%%%%%%%%%%%%%%%%%%%%%%%%%%%%%%%%%%%%%%%%%%%%%%%%%%%%%
% \usepackage{lastpage}
% \usepackage{fancyhdr}
% \pagestyle{fancy}
% \renewcommand{\headrulewidth}{0.0pt}
% \renewcommand{\footrulewidth}{0.0pt}
%%%%%%%%%%%%%%%%%%%%%%%%%%%%%%%%%%%%%%%%%%%%%%%%%%%%%%%%%%%
% Cabeçalho
%%%%%%%%%%%%%%%%%%%%%%%%%%%%%%%%%%%%%%%%%%%%%%%%%%%%%%%%%%%
% \fancyhf[ROH,LEH]{\includegraphics[scale=0.50]{logo2.eps}}
% \fancyhf[LOH,LEH]{\textsc{\cliente Prop. \proposta, V \versao }}

% \fancyhf[REH,LOH]{\large\nouppercase\leftmark}        % \leftmark prints the current section
% \fancyhf[ROF,LEF]{\large\thepage}
%\fancyhf[ROF,LEF]{Page:\ \thepage\ of \pageref{lastpage}}
% \rfoot{\small{Página\ \thepage\ de \pageref{LastPage}}}

%%%%%%%%%%%%%%%%%%%%%%%%%%%%%%%%%%%%%%%%%%%%%%%%%%%%%%%%%%%
% Rodapé
%%%%%%%%%%%%%%%%%%%%%%%%%%%%%%%%%%%%%%%%%%%%%%%%%%%%%%%%%%%
% \fancyfoot[C]{\rule{100pt}{0.5pt}\hspace{100pt}\rule{100pt}{0.5pt}\\Pela Sindesi\hspace{100pt}Pelo Cliente\\

%\fancyfoot[C]{\rule{400pt}{1pt}\\Pág. \thepage{}\hspace{3pt} de \pageref{LastPage}}

% \fancyhf[REF,LOF,C]{\rule{30pt}{1pt}\\Sindesi}
%\fancyhf[REF,LOF]{\rule{100pt}{0.5pt}\\Sindesi}
% \fancyhf[REF,COF]{\rule{100pt}{0.5pt}\\João Castelhano}
%%%%%%%%%%%%%%%%%%%%%%%%%%%%%%%%%%%%%%%%%%%%%%%%%%%%%%%%%%%
% \fancypagestyle{plain}{
% \fancyhf[ROH,LEH]{\includegraphics[scale=0.50]{logo2.eps}}
% \fancyhf[LOH,LEH]{\textsc{\cliente Prop. \proposta, V \versao }}
% \fancyfoot[C]{\rule{400pt}{1pt}\\Pág. \thepage{}\hspace{3pt} de \pageref{LastPage}}
% }

\newcommand\bcsindesi{\includegraphics[width=\logowidth]{logo2}}











\begin{document}

 \includepdf[pages=1]{capa_livro.pdf}
% \input{capa_livro}


\frontmatter
\pagestyle{empty}


% % half-title page
% \vspace*{\fill}
% \begin{adjustwidth}{1in}{1in}
% \begin{flushleft}
% \HUGE\sffamily The
% \end{flushleft}
% \begin{center}
% \HUGE\sffamily  Memoir
% \end{center}
% \begin{flushright}
% \HUGE\sffamily  Class
% \end{flushright}
%%\begin{center}
%%\sffamily (Draft Edition 7)
%%\end{center}
% \end{adjustwidth}
% \vspace*{\fill}
% \cleardoublepage

% title page
\vspace*{\fill}
\begin{center}
\HUGE\textsf{Koncepto}\par
\end{center}
\begin{center}
\LARGE\textsf{\~}\par
\end{center}
\begin{center}
\HUGE\textsf{FrontOffice}\par
\end{center}

\begin{center}
\Huge\textsf{Manual do Utilizador}\par
\end{center}
\begin{center}
\LARGE\textsf{Sindesi}\par
\bigskip
\normalsize\textsf{Mantido por Nuno Leitão}\par
\medskip
\normalsize\textsf{Correspondendo à versão 4.3.4}\par  
\end{center}
\vspace*{\fill}
\def\THP{T\kern-0.2em S\kern-0.4em P}%   OK for CMR
\def\THP{T\kern-0.15em S\kern-0.3em P}%   OK for Palatino
\newcommand*{\THPress}{The Sindesi Press}%
\begin{center}
\settowidth{\droptitle}{\textsf{\THPress}}%
\textrm{\normalsize \THP} \\
\textsf{\THPress} \\[0.2\baselineskip]
%% \includegraphics[width=\droptitle]{anvil2.mps}
\setlength{\droptitle}{0pt}%
\end{center}
\clearpage



\PWnote{2009/06/26}{Updated the copyright page for 9th impression}
% copyright page
\begingroup
\footnotesize
\setlength{\parindent}{0pt}
\setlength{\parskip}{\baselineskip}
%%\ttfamily
\textcopyright{} 2008 --- 2015 Sindesi e Soluções Informáticas, Lda. \\
% \textcopyright{} 2011 --- 2013 Lars Madsen \\
Todos os direitos reservados.
% All rights reserved.

% The Sindesi Press, Normandy Park, WA.

Printed in the World 

% The paper used in this publication may meet the minimum requirements
% of the American National Standard for Information 
% Sciences --- Permanence of Paper for Printed Library Materials, 
% ANSI Z39.48--1984.

\PWnote{2009/07/08}{Changed manual date to 8 July 2009}
\begin{center}
% 10 09 08 07 06 05 04 03 02 01\hspace{2em}19 18 17 16 15 14 13
Versões Lançadas:
\end{center}
\begin{center}
\begin{tabular}{ll}
Versão 0.1:                        & Outubro 2005 \\
Versão 2.0:			& Maio 2013 \\
Versão 3.0:			& Abril 2015 \\
\end{tabular}
\end{center}
% \ifMASTER
% Manual last changed today% \svnyear/\svnmonth/\svnday
% \fi

Este manual foi escrito com recurso exclusivo a ferramentas open source.

Este manual foi escrito em \LaTeX. As imagens foram criadas com  
recurso aos programas Inkscape e Gimp em ambiente Linux.


\textbf{Informação Legal}

Linux\textcopyright{} é marca registada de Linus Torvalds.

PostgreSQL\textcopyright{} is released under the PostgreSQL Global Development Group.


MS-DOS\textcopyright{} é uma marcas registadas da Microsoft Corporation  % IBM\textcopyright{} (International Business Machines Corporation).



\endgroup


%%%%%%%%%%%%%%%%%% CITAÇAO %%%%%%%%%%%%%%%%%%%%%%%%%%%%%%%%%
% \clearpage
% \vspace*{\fill}
% \begin{quote}
% \textbf{FrontOffice koncepto,} é um livro que levou algum tempo a escrever.
% 
% O objectivo foi torna-lo não num livro de leitura, mas mais um livro de consulta rápida,
% permitindo ao utilizador poder manusear o programa de forma rápida e intuitiva.
%   \hspace*{\fill} 
%       \textit{Nuno Leitão}.
% \end{quote}
%%%%%%%%%%%%%%%%%% FIM DE CITAÇAO %%%%%%%%%%%%%%%%%%%%%%%%%%%%%%%%%

% \vspace{2\baselineskip}
% 
% \begin{quote}
% \textbf{memoir,} \textit{n.} [Fr. \textit{m\'{e}moire,} masc., a memorandum,
%     memoir, fem., memory $<$ L. \textit{memoria,} \textsc{memory}]
%   \hspace{1ex} \textbf{1.} a biography or biographical notice, 
 %      usually written by a relative or personal friend of the subject 
 %  \hspace{1ex} \textbf{2.} [\textit{pl.}] an autobiography, 
%       usually a full or highly personal account
 %  \hspace{1ex} \textbf{3.} [\textit{pl.}] a report or record of 
%       important events based on the writer's personal observation, 
%       special knowledge, etc.
%   \hspace{1ex} \textbf{4.} a report or record of a scholarly 
%       investigation, scientific study, etc.
%   \hspace{1ex} \textbf{5.} [\textit{pl.}] the record of the proceedings
%       of a learned society \\[0.5\baselineskip]
%   \hspace*{\fill} \textit{Webster's New World Dictionary, Second College Edition}.
% \end{quote}

% \vspace{2\baselineskip}


% \begin{quote}
% \textbf{memoir,} \textit{n.} a fiction designed to flatter the subject 
%   and to impress the reader. \\[0.5\baselineskip]
% \hspace*{\fill} With apologies to Ambrose Bierce % and Reuben Thomas
% \end{quote}

\vspace*{\fill}

\cleardoublepage

% ToC, etc
%%%\pagenumbering{roman}
\pagestyle{headings}
%%%%\pagestyle{Ruled}

% \setupshorttoc
% \tableofcontents
% \clearpage
% \setupparasubsecs
% \setupmaintoc
% \tableofcontents
% \setlength{\unitlength}{1pt}
% \clearpage
% \listoffigures
% \clearpage
% \listoftables
% \clearpage
% \listofegresults



\tableofcontents

\pagenumbering{arabic}

% body
\mainmatter

\chapter{Licenciamento}
% \section{FrontOffice Koncepto}     %   1.  FrontOffice Koncepto
\subsection{Licença}

Copyright 2004 - 2015 Sindesi - Sistemas e Soluções Informáticas, Lda.

Todos os direitos reservados.

A utilização do software Koncepto incluído neste pacote foi concebida segundo os termos
descritos no Acordo de Licença de utilização do Software.

O utilizador só poderá realizar as cópias do Software permitidas no Acordo de Licença de
utilização do Software.

O utilizador poderá carregar e executar o Software na memória RAM do computador e utilizá-lo
num computador monoposto ou numa estação de trabalho individual.

O utilizador só poderá utilizar o Software em uma rede de comunicações se a licença de uso o
permitir e descriminar por escrito o número de postos que a licença permite usar em simultâneo.
A duplicação, cópia, distribuição ou qualquer outro tipo de cedência deste manual constitui um
delito contra a propriedade intelectual, se não for autorizada por escrito pela Sindesi - 
Sistemas e Soluções Informáticas, Lda.

Importante: Este acordo contém as condições da Licença de utilização do Software contido no
pacote selado. A abertura do pacote selado significa, da parte do cliente, a aceitação na íntegra,
das condições do presente acordo.
                    %   1.1.      Licença 
\subsection{Acordo de Licença de utilização de Software}
\subsubsection{Licença de Utilização}
A propriedade do Software contido no pacote selado não é
transferida para o cliente. A Sindesi - Sistemas e Soluções Informáticas, Lda, concede ao cliente, uma licença de
utilização, não transferível a terceiros, para usar o Software em uma estação de trabalho. Se o
Software estiver a ser partilhado através de uma rede de comunicações o cliente deverá ter uma
licença que o autorize a partilhar o Software, descriminando o número de postos de trabalho que
estão autorizados a partilhar o Software em simultâneo.


\subsubsection{Restrições de Cópia}
O cliente poderá copiar o Software, objecto da licença, unicamente
para fins de segurança, podendo no máximo dispor de uma cópia de segurança.

\subsubsection{Cumprimento e uso não Autorizado}
O cliente deverá comunicar por escrito à Sindesi - Sistemas e Soluções Informáticas, Lda.
 se tiver conhecimento da utilização não autorizada do Software a si licenciado,
por parte de terceiros e fazer todos os possíveis ao seu alcance para que esta utilização cesse
de imediato.

                     %   1.2.      Acordo de Licença de utilização de Software 
%


\subsection{Para quem se destina este manual}

Este manual destina-se aos utilizadores do sistema de facturação e gestão Koncepto,
e  tem como objectivo servir de apoio na utilização do produto em causa.

Servirá como referência base na iniciação e conhecimento do produto. Pretende-se
que ele sirva como um guia de referência rápida para o apoio na execução das
principais tarefas que o produto disponibiliza.



\subsection{Requisitos mínimos}

Para utilização desta aplicação são considerados os seguintes requisitos mínimos:
\begin{itemize}
\item Conhecimentos básicos na área de negócio.
\item Conhecimentos básicos de informática.
\item Conhecimentos básicos da aplicação de frontoffice Koncepto e conceitos adjacentes.
\end{itemize}





\newpage
\part{Introdução}

\chapter{Antes de começar}

O programa Koncepto foi desenhado para ser intuitivo e de fácil utilização mediante o uso do toque do dedo 
no ecrã para aceder às suas funcionalidades. 

Neste capítulo será explicada a forma de como interagir com o programa nas diversas fases.

\section{Sobre este manual}

Este manual tenta ser um guia prático de como utilizar de uma forma eficiente o programa FrontOffice Koncepto.

No sentido de tornar o seu entendimento mais simples, dividiu-se este manual várias partes.


Este manual é composto por três partes.

Na primeira parte, vai-se fazer uma breve introdução a como ler este manual, algumas informações relevantes no que diz respeito ao software 
e a apresentação de algumas convenções desenvolvidas no sentido de tornar a leitura mais simples, tornando-o quer num livro de consulta, 
quer um manual expositivo e de leitura acessível, e que pretende que qualquer utilizador consiga manusear o software em poucos passos.


\begin{enumerate}
\item um guia de referência rápido, que permite com pequenas dicas começar a usar um programa devidamente instalado em poucos passos.
\item uma secção de apoio à gestão de mesas;
\item uma secção de apoio à gestão de relatório (controlo de ponto, SAF-T PT, controlo de produtos, etc);
\item uma secção de apoio à criação de produtos de venda característicos da área de restauração com explicação exaustiva de cada situação;
\item uma secção de apoio ao conceito de impressoras virtuais, e redireccionamento de pedidos em caso de necessidade; 
\item uma secção de apoio à exploração de zonas de um espaço demonstrando com exemplos práticos as várias situações;
\item uma secção de apoio à criação de clientes e cartões cliente;
\item uma secção de apoio à criação de cartões de consumo, gestão de reservas para controlo rigoroso de restauração colectiva.
\end{enumerate}







\section{Convenções usadas neste livro}


Ao longo do livro, para facilitar a leitura, usam-se diversos \textbf{estilos} para que o leitor
identifique rapidamente os elementos a identificar no programa.


 \begin{table}[ht]
%  \caption{Hardware} 
 \centering
\small
 \begin{tabular}{c p{12cm}}  %   l c r r } % centered columns 
 \textbf{Estilo do texto} & \textbf{Significado}  \\ % Garantia & Descrição & Qtd & C. Unit & Subtotal \\ [0.5ex]
%  \multicolumn{6}{l}{\textbf{Servidor}} \\			
 \hline
 \textsc{Small Caps} &  Refere-se a um nível do programa.  \\
 		    & Assim quando se ler \textsc{Gerente} o leitor deve procurar do lado direito do programa o tabulador tenha escrito a palavra "Gerente". \\
\hline
 \keystroke{Botão}  &  Refere-se a um botão do programa.  \\
		&  Assim quando se ler \keystroke{Sessões} o leitor deve procurar um botão que tenha escrito a palavra "Sessões". \\

%  \multicolumn{6}{l}{\textbf{Bastidor}} \\			
 \hline
 \end{tabular}
 \end{table}


% \vspace{5mm}
% O estilos \textsc{Small Caps} irá servir para referenciar um
% Nível do programa.

% Assim quando se ler \textsc{Gerente} o leitor deve procurar do lado direito do programa o tabulador tenha escrito
% a palavra "Gerente".

% \vspace{5mm}

\vspace{5mm}
Serão ainda apresentado algumas \textbf{caixas de texto} com informação adicional.

Por vezes são informações críticas, ou exemplos descritivos ou simplesmente algumas dicas de utilização.
% sendo de importância crítica, servem para que se possa tirar o maior rendimento das funcionalidades do 
% programa.





\bcdica{Dica}{
Nestas caixas são apresentadas dicas e sugestões para melhor usar o programa. 
}

\bcexemplo{Exemplo}{
Nestas caixas dá-se uma explicação com exemplos práticos de alguma funcionalidade.

Seguramente a forma mais simples e directa de explicar uma funcionalidade.
}


\bcatencao{Atenção}{
Nestas caixas dá-se relevância aos pormenores a que deve ter especial cuidado 
para uma melhor experiência com o programa.
}


\bcperigo{Muito cuidado}{
Nestas caixas dá-se relevância aos pormenores de importância crítica e que podem 
ter impacto directo no sistema de facturação ou no próprio hardware.
}

\bcrelacionados{
Nestas caixas vai-se referenciar os capítulos com informação relevante ao assunto que está a abordar. 
% \textbf{Versão de Software} na secção \ref{versao},  na página \pageref{versao}.
		
Por ex: \textbf{Versão de Software} na secção \ref{versao},  na página \pageref{versao}.
}


% \vspace{5mm}
% \begin{center} 	\ovalbox{ \begin{minipage}[c]{0.15\textwidth} \includegraphics[width=2cm]{../images/info.png} 	\end{minipage} 
% 		\begin{minipage}[c]{0.80\textwidth}
% 		
% 		% Ver mais informações sobre o assunto de 
% 		Como muitos conceitos neste livro se interligam com outros, colocou-se no fim de cada secção uma zona
% 		com links directos para informações relacionadas com o tema abordado.
% 
% \end{minipage}}\end{center}
% 

\newpage
\section{Interface Gráfico}

% Todo este programa tenta ser o mais simples possível, contudo
% concordamos que este é um compromisso que nem sempre é simples de conseguir.

% Por esta razão passamos a apresentar o inteface de utilizador de forma a tornar o 
% ambiente o mais familiar possível, permitindo aos utilizadores usarem o programa de 
% uma forma fluida e natural.



\subsection{Setas de Selecção}

\begin{figure}[h]
\begin{center}
\includegraphics[height=5cm]{../images/setasSelecao.png}
\caption[Submanifold]{Setas de Selecção.}
\end{center}
\end{figure}


As setas de selecção permitem seleccionar um elemento de um lista.

As setas são representadas por triângulos em cor cinza no fundo da janela onde se encontra a lista. 

Pressionando as zonas onde se encontram os triângulos, poder-se-á navegar para cima e para baixo nos elementos da lista.

\subsection{Botão Definir}

\begin{figure}[h]
\begin{center}
\includegraphics[height=3cm]{../images/botaodefinir.png}
\caption[Submanifold]{Botão Definir.}
\label{botaodefinir}
\end{center}
\end{figure}


É muito comum aparecer botões de \emph{toggle} ou de comutação. 

Para facilitar o acesso a estes botões, vai encontrar sempre associado
a estas listas de opções um botão \keystroke{Definir}.

Cada vez que pressionar o botão \keystroke{Definir} irá marcar a opção 
como activa ou inactiva alternadamente.

Caso não tenha a certeza da alteração que pretende fazer, terá sempre 
disponível um botão que permite sair da configuração que está a fazer.





\newpage
\subsection{Teclados}

O interface do utilizador é sempre o ecrã, por este motivo é necessário que 
haja uma forma que se possa interagir directamente com o programa. 
Por isso haverá sempre disponível um teclado para todas as interacções do operador.

Estas opções são simplificadas ao máximo de forma a evitar o recurso aos mesmos,
 pois pretende-se que a operabilidade 
seja o mais célere possível.

Existem dois tipos de teclado disponíveis, o alfanumérico e o numérico, que estarão disponíveis
de acordo com as necessidades.


\subsubsection{Teclado alfanumérico}

\begin{figure}[h]
\begin{center}
\includegraphics[height=5cm]{../images/teclado.png}
\caption[Submanifold]{Representação do teclado alfanumérico no ecrã.}
\label{fig:alfanumerico}
\end{center}
\end{figure}



O teclado alfanumérico tem uma abrangência maior, e é invocado em quase todos os casos em
que se prevê a necessidade dele. O teclado está representado na Figura \ref{fig:alfanumerico}

Para colocar maiúsculas poderá pressionar a tecla \keystroke{CAPS} à esquerda, ou simplesmente pressionar
uma das teclas \keystroke{SHIFT} disponíveis em cada lado do teclado.

Se por alguma razão se enganou e pretender voltar ao menu anterior sem modificar os dados, bastará pressionar
a tecla \keystroke{ESC} que se encontra no canto superior esquerdo.

Para validar os dados submetidos, pressione a tecla \keystroke{$\hookleftarrow$} (ou ''Enter'') que aqui é representada como
um caractere de mudança de linha no lado direito do teclado.


\subsubsection{Teclado numérico}

\begin{figure}[h]
\begin{center}
\includegraphics[height=5cm]{../images/numerico.png}
\caption[Submanifold]{Representação do teclado numérico no ecrã.}
\end{center}
\end{figure}


O teclado numérico é utilizado para submissão de palavras chave de utilizadores, códigos de barra,
indicação de quantidades, preços, taxas, etc, ...

Este teclado conta com os botões:


 \begin{table}[ht]
%  \caption{Hardware} 
 \centering
\small
\def\arraystretch{1.5}
 \begin{tabular}{c p{12cm}}  %   l c r r } % centered columns 
 \textbf{Botão} & \textbf{Significado}  \\ % Garantia & Descrição & Qtd & C. Unit & Subtotal \\ [0.5ex]
%  \multicolumn{6}{l}{\textbf{Servidor}} \\			
 \hline
\keystroke{Atrás} & permite apagar o último caractere pressionado. \\
\keystroke{Cancelar} &  voltará ao menu anterior sem modificar qualquer dados. \\
\keystroke{OK} & Validará os dados colocados. \\
%  \multicolumn{6}{l}{\textbf{Bastidor}} \\			
 \hline
 \end{tabular}
 \end{table}







Os números premidos deverão aparecer sempre no ecrã, salvo duas excepções: a introdução de uma 
palavra passe, e a senha de identificação de um cartão cliente.






\subsection{Botão Propriedades}

\begin{figure}[h]
\begin{center}
\includegraphics[height=5cm]{../images/propriedades2.png}
\caption[Submanifold]{Propriedades de Botões.}
\label{propriedadesbotoes}
\end{center}
\end{figure}


Em diversos pontos do programa, pode-se encontrar o botão \keystroke{Propriedades}.

Esta opção permite personalizar botões de forma a que ganhem realce no interface gráfico. As características que se podem modificar são:
 % a secção em causa permitido desde uma simples troca de cor, aplicação de textura ou aplicação de uma imagem.



 \begin{table}[ht]
%  \caption{Hardware} 
 \centering
\small
\def\arraystretch{1.5}
 \begin{tabular}{c p{12cm}}  %   l c r r } % centered columns 
 \textbf{Botão} & \textbf{Significado}  \\ % Garantia & Descrição & Qtd & C. Unit & Subtotal \\ [0.5ex]
%  \multicolumn{6}{l}{\textbf{Servidor}} \\			
 \hline
 	\keystroke{Cor de Fundo} &  Muda a cor de fundo do botão. \\
	\keystroke{Textura de fundo} &  Associa uma imagem como padrão de fundo do botão. \\ 
	\keystroke{Cor de Texto} &  Altera a cor do texto. \\ 
	\keystroke{Letra} & Altera o tipo de letra dos caracteres do botão. \\ 
	\keystroke{Imagem} & Associa uma imagem ao botão. \\ 
	\keystroke{Limpar Imagem} & Desassocia qualquer imagem que esteja associada a este botão. \\ 
	\keystroke{Limpar Propriedades} & este botão limpa todas as características fazendo com que o botão fique com os valores por omissão. \\
%  \multicolumn{6}{l}{\textbf{Bastidor}} \\			
 \hline
 \end{tabular}
 \end{table}





% \begin{bclogo}[couleur=blue!10,arrondi=0.1,logo=\bctakecare,ombre=true]
\bcatencao{Tipos de letra e imagens}{
Quando fizer alterações, tenha presente que:
\begin{itemize}
\item os tipos de letra serão somente as disponibilizadas pelo sistema operativo, podendo em muitos casos não se encontrarem muitos tipos de letra instalados.
\item as imagens disponibilizadas são as imagens padrão incluídas no software, podendo ser aumentado mediante solicitação. 
\end{itemize}
}




% \subsection{Caixas}

% \subsection{Barras de Selecção}




\part{FrontOffice Koncepto}

\chapter{FrontOffice Koncepto}

O software de FrontOffice Koncepto foi um programa que teve o seu início nos anos 90.

Todo o seu desenvolvimento é consequência de um acompanhar das necessidades das pessoas e dos negócios.

Contudo apesar do acompanhado desenvolvimento no software, o mundo
 continua a evoluir e aceitamos o desafio de continuar a adaptar
 as nossas soluções às necessidades das pessoas, que cada vez mais
 estão familiarizadas com as novas tecnologias permitindo-nos
 assim criar novas sinergias no sentido de multiplicar o potencial das nossas soluções.


Temos a exemplo disto a solução de BackOffice Online, que é parte complementar do software e pode ser consultado também neste manual.

A aplicação com arquitectura Cliente-Servidor, é desenvolvida em C++, com base de dados PostgreSQL, corre sobre o sistema operativo Linux.


É optimizado para pontos de venda com tecnologia touchscreen e permite ligação a um conjunto de periféricos como impressoras de talões,
impressoras de pedidos, leitores de código de barras ou de banda magnética, gavetas de dinheiro, balanças, etc.

Permite comunicações e acesso remoto permanente utilizando simples acessos à Internet de banda larga.
Além das funcionalidades convencionais deste tipo de aplicação, realçamos as seguintes características:
\begin{itemize}
\item Configuração gráfica do ambiente de utilização e documentos por posto;
\item Operações de venda normal com controlo de cancelamentos, anulamentos e notas de crédito;
\item Gestão de sessões (refeições) e pontos de venda com configurações de preço e artigos distintos;
\item Parametrização de utilizadores por níveis de acesso com possibilidade de personalização gráfica;
\item Organização de produtos por páginas e sub-páginas;

\item Criação de produtos tipo “menu”, compostos e temporários;

\item Criação de cartões de consumidor com identificação de utilizador, num. de cartão, nome, etc. ;

\item Operações de vendas a consumidores internos com as seguintes possibilidades de parametrizações:

\begin{itemize}
\item Emissão de facturas ou consumos a crédito;
\item Descontos associados por artigo e cartão de consumo;
\item Restrições de consumo e de descontos por quantidade;
\item Indexação de cartões por centro de custo;
\item Emissão de talões de verificação;
\item Definição de suporte financeiro percentual por cartão de consumidor.
\end{itemize}
\item Emissão de relatórios contabilísticos;
\item Emissão de relatórios de gestão:
\begin{itemize}
\item produtos vendidos;
\item reserva efectuadas;
\item vendas por operadora;
\item outras funcionalidades descritas em detalhe no manual de software.
\end{itemize}
\item Registo em “log” de todas as operações realizadas pelos utilizadores da aplicação;
\item Controlo de ponto e registo de consumo interno;
\item Leitura de cartões de banda magnética, código de barras ou de proximidade;
\item Comunicações on-line com BackOffice Regi;
\item Modulo de Documentos que permite o lançamentos de requisições, notas de recepção e inventários no posto;
\item Modulo Web com acesso directo e parametrizável a informação da base de dados central de BackOffice.

\end{itemize}





\newpage
\section{Conceitos}

A fim de melhorar a experiência do utilizador e permitir melhores análises, foram criados alguns elementos 
que serão utilizados neste programa.

\subsection{Sessão}

A sessão diz respeito ao horário em que se pretende fazer o controlo de caixa.

A partir desta, é feito o controlo de empregado e de vendas de artigos. 

Todos os relatórios são elaborados a partir das Sessões decorridas, pois são estas 
que determinam a abertura e fecho de um estabelecimento e contagem de caixa.



% \begin{bclogo}[couleur=blue!10,arrondi=0.1,logo=\bccrayon,ombre=true]
\bcexemplo{Conceito de Sessões}{
Vejamos o seguinte exemplo:

Para um melhor entendimento do que consiste uma sessão, tomemos o exemplo de um estabelecimento 
que abra às 8h e encerre às 2h do dia seguinte.

Consideremos ainda que este estabelecimento 
trabalha em dois turnos, um das 8h às 17h e outro das 17h às 2h do dia seguinte.

Para agilizar o processo de controlo de caixa faz-se duas sessões, uma referente ao período da manhã e outro 
referente ao período da tarde. 

A partir daqui é possível os funcionários do turno da manhã darem o trabalho por terminado encerrando o caixa. 

O turno da tarde poderá iniciar o trabalho sem que este seja afectado de alguma forma pelo serviço da manhã.
}


\bcdica{Configuração de Sessões}{
As sessões podem ser configuradas para funcionar de uma forma manual, permitindo ao gerente ter algum controlo
sobre o fecho de sessão, ou automática, fazendo com que as contas transitem entre sessões. 
}


\bcatencao{Mapa de IVA por Controlo de Caixa}{
Há uma diferença entre o Mapa de IVA obtido mediante o controlo de caixa (que diz respeito aos intervalos das
sessões) e o Mapa de IVA diário, que diz respeito aos períodos de 24 horas. Obviamente o controlo de caixa privilegia
o controlo do negócio. Ambos os mapas podem ser extraídos do software, mas é importante perceber a diferença entre eles.
}


\subsection{Família}

Por uma família, entende-se por agrupamento de produtos similares. 

Em vez de atribuir determinadas características por produto pode-se atribuir uma regra para associar a todos os produtos de um mesmo tipo, facilitando 
assim a interacção com o programa.

Para além disso, criando uma família, é possível ter um entendimento sobre a venda de determinados 
tipos de produtos e como estes afectam o volume de vendas. 


\bcexemplo{Análise por Família}{
Por exemplo, agrupando todos os artigos de um determinado fornecedor numa única família, 
permite fazer uma leitura imediata do volume de vendas de um determinado produto, 
assim como o impacto percentual na facturação aquando o fecho de caixa.
}


\subsection{Perfil de Utilizador}

Consoante a função de cada funcionário, pode-se especificar a cada momento os privilégios de cada um,
retirando as responsabilidade da parte humana e delegando ao software a tarefa de gestão de contas.


Em cada perfil, pode-se definir ainda as zonas disponíveis, dando a um funcionário o acesso a apenas
uma sala e não a toda a área comercial.

Pode-se ainda fazer perfil de máquina para que a partir de um tablet, PDA ou \emph{smartphone} se 
fazer somente ordens de produção (botão \keystroke{Pedir}) para a cozinha.



\bcexemplo{Funções por utilizador}{
Por exemplo, impedindo vários funcionário de facturar (fazendo apenas pedidos) e 
delegando a apenas a um gerente a tarefa de fazer o serviços e fechar contas terá
sempre a certeza de que tem controlo absoluto do seu negócio.

Note-se que é impossível cobrar ou fazer chegar um pedido ao cliente sem que este tenha passado pelo sistema.
}


\subsection{Mesa}

A Mesa é onde se realizam as transacções.

Numa mesa pode-se definir opções como consumo mínimo, consumo máximo, número de pessoas, cartão de consumo, etc.


\subsection{Submesa}

\begin{HUGE}
algum texto em falta
\end{HUGE}



\subsection{Zona}

Entende-se por Zona uma área física ou convencionada a que fazem parte uma ou mais mesas. 

Havendo separações por Zonas, permite-se ter uma percepção gráfica do estado das 
contas de um espaço comercial, e permite verificar quais as mesas que estão prestes a ser
fechadas.

Numa zona é possível definir características a grupos de mesas, assim como o preço dos artigos a cobrar em cada área,

\bcdica{Preços por Zona}{
É normal e legítimo que um espaço queira cobrar mais pelo serviço feito numa área privilegiada.

Para isso é necessário atribuir ainda outro preço ao mesmo artigo, e definir que numa zona específica,
será cobrado aquela categoria de preço. 
}
 

\subsection{Página de Produtos}

A página de produtos é onde ficam listados para venda os diversos produtos de venda.

Cada página poderá albergar quantos produtos quantos se queiram, contudo o espaço destinado a estes é 
limitado e por isso colocar demasiados artigos por página irá dificultar a navegabilidade


\subsection{Índices}

\index{índices}
Os índices são as categorias globais onde se encontram as páginas. 

Cada índice poderá ter na sua estrutura outros índices abaixo deste, permitindo conseguir ter num pequeno
espaço aceder a milhares de artigos com pouca navegação.

Os índices são associados a zonas, e assim permite-se que em espaços diferentes do mesmo espaço se vendam 
artigos distintos sem que as áreas de trabalho não se afectem. 


\bcexemplo{Índices por Zonas}{
Num exemplo de um restaurante com bar de esplanada, pode-se restringir o acesso à venda de gelados
à zona da esplanada associando o índice "Gelados" (com todos os gelados individuais) à zona "Esplanada".

Desta forma os artigos de gelataria não são apresentados no postos dentro do restaurante. 

Da mesma forma pode-se restringir o acesso do posto do bar de esplanada às sobremesas do restaurante, fazendo assim um controlo por zona.

Claro que se tiver privilégios para tal, o operador do posto poderá aceder a qualquer zona, de qualquer equipamento.
}
  




% \subsection{Perfil de Máquina}
% Em situações 





% \section{Nomenclatura utilizada neste manual}


\chapter{Guia de Iniciação Rápida}  %  

Neste capítulo será descrito resumidamente os passos necessários para a utilização imediata do sistema no dia de trabalho.

Note que este guia poderá ser ajustado em função dos equipamentos ou metodologias implementadas em cada negócio.

\begin{enumerate}
	\item Ligar equipamentos;
	\item Certifique-se que tem uma sessão a decorrer;
	\item Proceda à facturação;
	\item Feche a sessão;
	\item Desligue computador.
\end{enumerate}


\section{Ligar equipamentos}

Deverá ligar todo o equipamento, começando pelo hardware periférico e de seguida
o ponto de venda.

Verifique que os cabos estão bem ligados e caso haja uma UPS no local, que esta esteja 
bem ligada e em funcionamento normal.

\subsection{Como ligar a impressora e outros periféricos?}
Esta operação depende da marca e do modelo disponível. De modo geral, deverá
pressionar um botão (on/off), e obter como resposta o acendimento de um led
indicador.

Para mais indicações deverá recorrer ao manual disponibilizado pelo fabricante do
equipamento.

\bcdica{Pontos de venda integrados}{
Alguns equipamentos estão alimentados directamente ao equipamento ponto de venda, 
pelo que só deverão responder quando tudo estiver ligado.
}

\bcperigo{Cuidado com as UPS}{
Se possível evite ligar impressoras ou outros aparelhos de elevado consumo nas UPS,
pois não só aumentam bastante o desgaste das mesmas, como reduzem significativamente o tempo de autonomia
em caso de quebra de energia.
}



\section{Como aceder à aplicação?}

Ao iniciar o ponto de venda, deverá aguardar o arranque do sistema operativo, após
o qual a aplicação será iniciada automaticamente. É apresentada a interface
principal com a listagem de utilizadores pertencentes à sessão actual. Apenas estes
têm acesso à aplicação.


\begin{figure}
\begin{center}
\includegraphics[height=7cm]{../images/user2.png}
\caption[Submanifold]{Imagem de início.}
\label{welcomeScreen}
\end{center}
\end{figure}


Neste momento deverá aparecer uma imagem semelhante à apresentada na Figura \ref{welcomeScreen}

Note que neste momento todo programa está com acesso restringido, obrigando a que 
o utilizador se identifique para que possa executar qualquer operação
	
Para aceder deverá pressionar o botão com o utilizador respectivo, e indicar a chave
de acesso, caso lhe seja requisitada.


Caso já tenha uma sessão iniciada, logo que se autentique poderá começar a facturar.



\section{Como determinar se existe alguma sessão activa?}



Todas as transacções se operam dentro de sessões válidas.

Quer isto dizer que fora do período de uma sessão, será impossível proceder a qualquer operação.


O nome da sessão deverá aparecer no ecrã principal (ver figura \ref{userlogged}).

Caso não haja nenhuma sessão a decorrer, deverá aparecer a mensagem ''\textbf{No Session}''.

Na dúvida, acedendo ao nível \textsc{Gerente}, escolhe-se o botão \keystroke{Gestor de Sessão} e verifica-se qual
a sessão principal a decorrer.


\begin{figure}
\begin{center}
\includegraphics[height=7cm]{../images/user.png}
\caption[Submanifold]{"Sessão Diária" em funcionamento.}
\label{userlogged}
\end{center}
\end{figure}



\section{Como iniciar uma sessão?}
Caso a sessão seja definida como manual, será necessário iniciar a sessão (ver figura \ref{gestorSessao}). 

Para iniciar uma sessão basta aceder ao nível de \textsc{Gerente}, no botão \keystroke{Gestor de Sessão} 
e carregar no botão \keystroke{Iniciar Sessão}.

\begin{figure}[h]
\begin{center}
\includegraphics[height=7cm]{../images/gestorSessao.png}
\caption[Submanifold]{Gestor de Sessão}
\label{gestorSessao}
\end{center}
\end{figure}



Quando a sessão inicia, deverá sair na impressora um papel com a informações:
\begin{itemize}
	\item nome da sessão,
	\item hora de início da sessão, 
	\item o nome de todos os funcionários que foram designados para trabalhar na sessão escolhida.
\end{itemize}

% \begin{bclogo}[couleur=red!30,arrondi=0.1,logo=\bctakecare,ombre=true]
\bcatencao{\emph{''Falhou a ligação ao printserver''}}{ 
Caso a impressora esteja desligada quando se tentar imprimir algo, irá surgir a mensagem \emph{''Falhou a ligação ao printserver''}.

Por defeito o programa espera que esta falha esteja associada a hardware, e não volta a tentar enviar mais documentos para essa impressora 
para não atrasar o sistema deixando documentos no spool.

Se tem a certeza de que o problema não é da impressora, reinicie o sistema e certifique-se de que a impressora se encontra bem ligada antes 
de imprimir um documento.

\vspace{5mm}
Pode-se contornar esta característica alterando as definições da impressora para serem unidireccionais.
O erro deixará de aparecer, mas não é situação recomendável
}




\section{Como criar um produto?}

No nível de \textsc{Base de Dados}, tocar no botão \keystroke{Produtos}.

Dentro da janela pressionar o
botão \keystroke{Novo produto}, aparecerá um teclado virtual onde deve atribuir um nome ao
produto. %  este não deve conter apóstrofes pois não é compatível com o sistema.


\begin{figure}[h]
\begin{center}
\includegraphics[height=6cm]{../images/produtos.png}
\caption[Submanifold]{Quadro Produtos}
\label{produtos}
\end{center}
\end{figure}





Logo que coloque o nome, surge uma caixa de diálogo para que se seleccione a família a 
que pertence o produto. Seleccionando a família, ficará associado a este produto as 
regras inerentes a essa família:
\begin{itemize}
\item Tipo de produto;
\item Desconto máximo;
\item Imposto associado;
\item se Pede quantidade; 
\item se Pede valor do produto.
\end{itemize}

Se não atribuir nenhuma família, poderá editar os campos manualmente, contudo é fortemente recomendado
que o produto faça parte de uma família para que possa extrair informações posteriormente nos relatórios.

Seleccione o \keystroke{Tipo de Produto} como "Normal". Isto significa que se trata de um produto de venda 
simples sem componentes. Posteriormente serão abordados os outros tipos de produto.


No campo \keystroke{Preço 1} coloque o preço a que pretende vender o artigo. 

O sistema trabalha com três casas decimais e este caso pode ser útil em situações de 
venda de produtos a peso ou volume. 

Pode estipular até sete preços por artigo, mas considere por agora definir apenas o 
valor para "Preço 1". 


Defina a taxa de impostos associada ao produto. No caso de um artigo com uma taxa de IVA de 23\%, deverá
preencher o campo \keystroke{Taxa \% 1} com o valor "23".

Por defeito o programa assume que os artigos têm  os valores dos impostos incluídos no preço final,
embora seja possível configurar o comportamento do programa para se comportar de forma diferente.
% De seguida ainda nesse mesmo separador Comum deve indicar a que Família pertence
% esse produto, o Preço também deve ser indicado assim como a respectiva Taxa.


Por fim no último separador \keystroke{Páginas}, deverá indicar para que página(s) irá
o produto criado. Para isso deverá seleccionar a página pretendida e pressionar \keystroke{Adicionar}.

% \begin{bclogo}[couleur=blue!10,arrondi=0.1,logo=\bclampe,ombre=true]
\bcdica{O mesmo artigo em múltiplas Páginas}{
Repare que podendo associar o mesmo produto a várias páginas, poderá colocar o mesmo artigo 
à venda em espaço diferentes do programa facilitando a navegação entre páginas de produtos.

Crie uma página designada "Rápidas" e coloque lá os artigos que tenham mais saída. Desta forma garante
um serviço mais rápido pois o operador escusa de procurar os artigos em todos os menus 
cada vez que for facturar.
}

% \begin{bclogo}[couleur=blue!10,arrondi=0.1,logo=\bccrayon,ombre=true]
\bcexemplo{Exemplo Restaurante Cafetaria}{
Num espaço que trabalhe simultaneamente como restaurante e como cafetaria pode por exemplo colocar
o artigo "Água 0.33 cl" na página dos artigos de "Cafetaria" e na página de "Águas". 

Este pequeno exemplo permite que o operador consiga ter sempre visíveis os artigos que têm mais saída a 
cada momento acelerando o processo de registo. 
}

\bcrelacionados{ Consulte mais informações sobre este assunto no Capítulo \textbf{\ref{ch:produtos} Produtos} na página \pageref{ch:produtos}.

Consulte mais informações sobre este assunto na Secção \textbf{\ref{sec:paginas} Páginas} na página \pageref{sec:paginas}.
}


\section{Proceder a uma venda}

Neste momento já é possível abrir uma mesa. Seleccione o Nível \textsc{Zonas} e escolha uma mesa em que pretende trabalhar (Figura \ref{nivelMesa}).

Na parte de cima terá os \textbf{Índices} e por baixo destes estarão as \textbf{Páginas}.


\begin{figure}[h]
\begin{center}
\includegraphics[height=7cm]{../images/mesa.png}
\caption[Submanifold]{Nível Mesa}
\label{nivelMesa}
\end{center}
\end{figure}



Para colocar um artigo, bastará navegar nas diversas páginas e pressionar o botão correspondente ao artigo pretendido.

Uma vez seleccionado o artigo este aparecerá na mesa. Caso tenha o botão \keystroke{Pedir} deverá pressionar para dar as ordens de producão nas devidas estações.

\subsection{Como cancelar uma artigo}

Para cancelar um artigo, basta seleccioná-lo da lista e pressionar o botão \keystroke{Cancelar}.

Caso o artigo já tenha sido pedido (as letras ficarão com a cor destacada) e consoante o nível de privilégios que tiver, poderá ser possível ou não proceder ao cancelamento de um artigo. Caso não tenha privilégios, por favor contacte o gerente de loja.


%% irá aparecer a caixa de diálogo de desperdício. Esta caixa diz respeito à movimentação de stock.


\bcrelacionados{Poderá consultar mais informação sobre este assunto na página \pageref{VendaArtigos}.}

\newpage
\section{Facturação}


Uma vez colocado todos os artigos, deverá pressionar o botão \keystroke{Factura}. 

Imediatamente irá aparecer um painel a pedir o número contribuinte do cliente. 
\begin{figure}[h]
\begin{center}
\includegraphics[height=3cm]{../images/fatura2.png}
\caption[Submanifold]{Insira NIF do cliente ou Cancele}
\label{fatura}
\end{center}
\end{figure}

\begin{figure}[h]
\begin{center}
\includegraphics[height=3cm]{../images/fatura3.png}
\caption[Submanifold]{Pesquise para preencher os dados do cliente}
\label{fatura}
\end{center}
\end{figure}



Se preencher o contribuinte, pressione em \keystroke{Pesquisar} e se o cliente existir na base de dados, os dados serão preenchidos automaticamente,
 e poderá proceder à facturação, caso contrário terá de preencher os restantes campos.

Se no painel pressionar \keystroke{Cancelar}, então seguirá imediatamente para o painel de facturação.

No painel de facturação aparece do lado esquerdo o valor a pagar, e o método de pagamento usado.

Do lado direito aparece o valor entregue. Quando se pressiona \keystroke{OK}, surge uma caixa de diálogo com o valor do troco a entregar ao cliente (figura \ref{fatura}).


\begin{figure}[h]
\begin{center}
\includegraphics[height=6cm]{../images/fatura1.png}
\caption[Submanifold]{Facturação}
\label{fatura}
\end{center}
\end{figure}





\section{Como ver um documento já emitido?}
Através do botão \keystroke{História} que se encontra no nível de \textsc{Gerente}.

Este botão permite o
acesso a documentos passados que surgem da utilização da aplicação. Podem ser
vistos os detalhes dos documentos, imprimir duplicados,
criar notas de crédito e anular consumos.

\bcrelacionados{Consulte a secção \textbf{\ref{historia} História} na página \pageref{historia} para mais informações.}




\section{Como proceder a uma nota de crédito}

Caso tenha registado uma venda indevidamente, a única forma de proceder à anulação é emitindo um documento de Nota de Crédito.

Para tal terá de ir até ao nível \textsc{Gerente} e em \keystroke{História}, procurar o documento que gerou e proceder à anulação pressionando o botão \keystroke{Nota de Crédito} (figura \ref{Historia}).


\begin{figure}
\begin{center}
\includegraphics[height=7cm]{../images/historia.png}
\caption[Submanifold]{Painel História}
\label{Historia}
\end{center}
\end{figure}


Quando se pressiona em  \keystroke{Nota de Crédito}, surge uma nova caixa de diálogo que pergunta se se pretende refazer a factura ou não.

Independentemente da resposta, será impressa uma Nota de Crédito com os dados vigentes na factura em causa. 

Caso se tenha optado por refazer a factura, o sistema irá criar uma mesa temporária onde colocará todos os artigos que foram anulados para que se possa fazer a correcção e imprimir a factura correta.

% \begin{bclogo}[couleur=blue!10,arrondi=0.1,logo=\bclampe,ombre=true]
\bcdica{Configuração de Sessões}{
Todas as notas de crédito emitidas, são realizadas sobre vendas efectuadas.

O sistema não permite gerar crédito a favor de outrem sem que tenha havido anteriormente um documento de venda.
}


\subsection{Como encerrar uma sessão?}


Para encerrar uma sessão basta aceder ao nível de \textsc{Gerente}, no botão \keystroke{Gestor de Sessão} 
 seleccionar a sessão a fechar e clicar no botão \keystroke{Parar Sessão} e fechar a
respectiva janela.

\section{Relatórios}

\label{reportStarter}
Para extrair os relatórios referentes à sessão, é necessário ir ao nível de 
\textsc{Gerente}, no botão \keystroke{Relatórios Oficiais} e seleccionar o período
que se pretende extrair o relatório.

Os relatórios são extraídos por sessão, e por isso não é possível extrair relatórios referentes a sessões 
que estejam ainda a decorrer.

\bcrelacionados{Consulte o Capítulo \textbf{\ref{ch:relatorios} Relatórios} na página \pageref{ch:relatorios} para mais informações.}


\section{Como desligar a máquina?}
O POS deverá ser desligado através do software. No nível de \textsc{Utilizador} encontrará
um botão \keystroke{Desligar Máquina}, que ao pressionar fará aparecer uma mensagem a
confirmar a operação. 

Confirmando, a máquina será desligada.

% \begin{bclogo}[couleur=blue!10,arrondi=0.1,logo=\bclampe,ombre=true]
\bcdica{Acesso a desligar a máquina}{
Em certas configurações esta operação só pode ser efectuada após a validação de
determinado tipo de utilizadores – geralmente um técnico, supervisor ou gerente.
Em alguns modelos de equipamento, é necessário pressionar um botão físico
(on/off) no hardware.
}

Pressionando este botão, irá aparecer uma nova caixa de diálogo (figura \ref{desligarmaquina})a perguntar se tem realmente a certeza
de que pretende desligar o equipamento.


\begin{figure}[h]
\begin{center}
\includegraphics[height=2cm]{../images/desligarmaquina.png}
\caption[Submanifold]{Caixa de diálogo para desligar máquina}
\label{desligarmaquina}
\end{center}
\end{figure}




Se confirmar, o computador irá dar início ao processo de encerramento, desligando o sistema progressivamente
até ao momento em este ficará totalmente inactivo.

% \begin{bclogo}[couleur=red!30,arrondi=0.1,logo=\bcbombe,ombre=true]
\bcperigo{Desligar o Equipamento}{
        Caso o sistema não possua suporte a gestão de energia,
	aguarde até aparecer a mensagem \textbf{System Halted} e então será seguro desligar o equipamento.
	
	Não desligue o equipamento de forma abrupta, pois poderá danificar o dispositivo de massa, 
	podendo levar à perda irreversível de dados.
}






Para mais informação consulte o manual do fabricante.




% \newpage

% \section{Como proceder a uma venda}


\subsection{Conceito de Sessão}

Todas as transacções se operam dentro de sessões válidas.

Quer isto dizer que fora do período de uma sessão, será impossível proceder a qualquer operação.

\subsection{Como iniciar uma Sessão}

Para se iniciar a sessão para começar a fazer vendas, é necessário ir até \textbf{Gerente -> Gestor de Sessão}

\includegraphics[height=3in]{../images/gestorSessao.png}

A partir daqui, deve-se escolher a sessão que se pretende iniciar da lista da esquerda de pressionar o botão \textbf{Iniciar Sessão}.

Quando a sessão inicia, deverá sair na impressora um papel com a informações:
\begin{itemize}
	\item nome da sessão,
	\item hora de início da sessão, 
	\item o nome de todos os funcionários que foram designados para trabalhar na sessão escolhida.
\end{itemize}

\subsection{Proceder a uma venda}

Neste momento já é possível abrir uma mesa. Selecione o tabulador \textbf{Zonas} e escolha uma mesa em que pretende trabalhar.

Na parte de cima terá os \textbf{Índices} e por baixo destes estarão as \textbf{Páginas}.

\includegraphics[height=3in]{../images/mesa.png}

Para colocar um artigo, bastará navegar nas diversas páginas e pressionar o botão correspondente ao artigo prentendido.

\subsection{Como cancelar uma artigo}

Consoante o nível de privilégios que tiver, poderá ser possível ou não proceder ao cancelamento de um artigo. Caso não tenha privilégios, por favor contacte o gerente de loja.

Caso o artigo já tenha sido pedido (as letras ficarão com a cor negra) e ao pressionar a tecla cancelar, irá aparecer a caixa de diálogo de desperdício. Esta caixa diz respeito à movimentação de stock.

\subsection{Como proceder a uma nota de crédito}

Caso tenha registado a venda, a única forma de proceder à anulação, é emitindo um documento de Nota de Crédito.
Para tal terá de ir até ao nível \textbf{Gerente -> História}, procurar o documento que gerou e proceder à emissão de uma Nota de Crédito.


\subsection{Como Fechar a sessão}

\subsection{Como extrair os relatórios}

Ao final da sessão de trabalho, pode-se finalmente proceder à extracção dos relatórios.

Para tal deve-se ir até ao nível gerente, e pressionar o botão \textbf{Relatórios Oficiais}.

Escolha a data e a Sessão correcta e pressione imprimir. 

% \newpage

% \section{Perguntas Frequentes}


\subsection{Quando tento tirar um relatório aparece a mensagem A operação não pode ser terminada}


\subsection{Quando tento tirar um relatório, o sistema demora muito tempo a responder e aparece a mensagem Falhou a ligação ao pserver}

% \part{Procedimentos}

% \begin{bcbombe}{Mon Titre}
% Du texte qui se répète encore et encore pour l ’ exemple , du texte qui
% se répète encore et encore pour l ’ exemple , du texte qui se répète
% encore et encore pour l ’ exemple \ dots
% \end{bcbombe}





% Na aplicação Koncepto, ao realizar uma venda é gerado um documento do tipo
% modelo de Fatura.
% 
% O Koncepto não emite facturas. Caso a implementação realizada
% o permita, poderão ser realizados consumos a crédito, sendo a facturação realizada
% à posteriori numa aplicação de backoffice, tal como o \textbf{Regi X12}.

\chapter{Gestão de Mesas}

\begin{figure}
\begin{center}
\includegraphics[height=7cm]{../images/zonas.png}
\caption[Submanifold]{Nível Zonas}
\label{divisao1}
\end{center}
\end{figure}





A gestão das mesas é feita no nível \textsc{Zonas} e neste painel poderemos identificar:
\begin{itemize}
\item um selector de zona;
\item um quadro com as mesas disponíveis na zona seleccionada no selector;
\item um botão \keystroke{Juntar};
\item um botão \keystroke{Transferir};
\item um teclado numérico para invocar a mesa (se aplicável);
\end{itemize}


No quadro das mesas, será possível observar:
\begin{itemize}
\item o valor dos artigos dispostos na mesa;
\item o nome do empregado responsável pelo serviço da mesa;
\item o nome do cliente ou cartão cliente (se aplicável).
\end{itemize}

\section{Juntar e Transferir Mesas}
\label{VendaArtigos}



Na parte das \textbf{Zonas}, pode-se fazer o acompanhamento do estado das mesas e assim
permitir verificar em tempo real o plano geral dos consumos do estabelecimento.

Na parte de gestão das mesas, pode-se considerar úteis duas operações:

 \begin{table}[ht]
%  \caption{Hardware} 
 \centering
\small
\def\arraystretch{1.5}
 \begin{tabular}{c p{12cm}}  %   l c r r } % centered columns 
 \textbf{Botão} & \textbf{Significado}  \\ % Garantia & Descrição & Qtd & C. Unit & Subtotal \\ [0.5ex]
%  \multicolumn{6}{l}{\textbf{Servidor}} \\			
 \hline
 \keystroke{Juntar} &  permite juntar uma ou mais mesas como se fossem uma única.
Pode ser útil para situações de:
\begin{itemize}
\item Se juntarem fisicamente num espaço duas ou mais mesas, facilitando a gestão da zona;
\item Situação em por exemplo vários elementos do mesmo agregado familiar
se distribuem por mesas distintas e então se pretende uma factura única com todos os itens
consumidos.
\end{itemize}\\
\keystroke{Transferir} & para situação em que por alguma razão uma ou mais pessoas de uma mesa
se mudam para outro local ou zona do espaço comercial. \\
%  \multicolumn{6}{l}{\textbf{Bastidor}} \\			
 \hline
 \end{tabular}
 \end{table}


\subsection{Juntar mesas}

Pressione o botão \keystroke{Juntar} que se encontra no nível de \textsc{Zonas}.

Clique numa das mesas que pretende juntar.

clique na segunda mesa que pretende juntar.

Agora as duas mesas têm o mesmo nome e o somatório dos valores das duas mesas.


\subsection{Transferir mesas}

Pressione o botão \keystroke{Juntar} que se encontra no nível de \textsc{Zonas}.

Clique numa das mesas que pretende juntar.

clique na segunda mesa que pretende juntar.

Surgirá um quadro no ecrã com teclas de navegação que permitem manusear os vários artigos entre as mesas seleccionas.

\newpage
\section{Colocando artigos nas mesas}

No Nível \textsc{Mesa} podemos encontrar vários botões que nos permitem gerir as contas.


Passaremos a expor cada botão.



\begin{table}[ht]
%  \caption{Hardware} 
 \centering
\small
\def\arraystretch{1.5}
 \begin{tabular}{c p{11cm}}  %   l c r r } % centered columns 
 \textbf{Botão} & \textbf{Descrição}  \\ % Garantia & Descrição & Qtd & C. Unit & Subtotal \\ [0.5ex]
%  \multicolumn{6}{l}{\textbf{Servidor}} \\			
 \hline
\keystroke{Cancelar} &  remove o produto da linha de produção, gerando um documento de Cancelamento, ou Anulação conforme o estado. \\
\keystroke{Pedir} &  gera a ordem de entrega de todos os produtos seleccionados para
consumo, ainda por pedir, remetendo o pedido para as impressoras pré-definidas
para tal (caso existam e estejam definidas). A funcionalidade “Pedir” pode ser
configurada de modo a que ocorra automaticamente ao fazer o registo. \\
\keystroke{Factura} & Finaliza o documento emitindo respectivo talão para o cliente com o descritivo da conta. \\
\keystroke{Registar} & Finaliza o documento sem que haja impressão de documento. O documento poderá impresso posteriormente, 
contudo a utilização deste botão restringe-se à situação de haver algum problema físico na impressora. \\
\keystroke{Conta} & Gera uma consulta de mesa para ser levada ao cliente. \\ 
\keystroke{Quantidade} & Permite aumentar a quantidade de itens na mesa. Pressionando este botão surge no ecrã uma caixa de diálogo
a pedir o número a acrescentar ao item seleccionado. \\
\keystroke{Gaveta} & Abre a gaveta do ponto de venda. \\
\keystroke{Dividir} & Divide a conta por duas ou mais pessoas. \\
% \item \keystroke{Mudar Preço} --- 
% \item \keystroke{Reserva} --- 
% \item \keystroke{Extra} --- 
\keystroke{Cliente} & Permite identificar um cliente para lhe poder aplicar uma regra específica de facturação. \\
\keystroke{Código de Barras} &  Permite a escrita manual de um código de barras de um artigo, no caso de não ser possível utilizar o leitor apropriado, 
ou o item não estiver disponível para venda nas páginas. \\
\keystroke{Procura} & Permite procurar um artigo pelo nome. \\
%  \multicolumn{6}{l}{\textbf{Bastidor}} \\			
 \hline
 \end{tabular}
 \end{table}


\begin{figure}[h]
\begin{center}
\includegraphics[height=6cm]{../images/mesa.png}
\caption[Submanifold]{Nível Mesa}
\label{nivelMesa}
\end{center}
\end{figure}





% \subsection{Como pedir?}
% O botão \keystroke{Pedir}


% \subsection{Como cancelar um produto já seleccionado?}
% Antes de registar, deverá seleccionar o produto a cancelar, navegando na lista de
% produtos adicionados para consumo. Depois de seleccionado, deverá pressionar o
% botão Cancelar, sendo o mesmo removido da listagem.


% \subsection{Como registar?}
% No caso de uma venda a dinheiro o registo é realizado através do botão Registo ou
% Factura. A mesa é fechada e é apresentada uma janela onde é permitido escolher o
% método de pagamento, fazer o troco e anexar um cliente ao pagamento.

% \subsection{Como abrir a gaveta?}
% Através do botão \keystroke{Gaveta} no nível de \textsc{Mesas}, quando disponível.






\newpage
\section{Divisão de contas}


É normal que numa mesa peçam para dividir a conta entre duas ou mais pessoas. 

Vamos considerar a situação de estarem duas bebidas numa mesa, e que se pretende dividir a conta.

Depois de colocados os itens na mesa, deve-se pressionar o botão \keystroke{Dividir}, como pode ser observado na figura \ref{divisao0}

\begin{figure}[h]
\begin{center}
\includegraphics[height=7cm]{../images/divisao0.png}
\caption[Submanifold]{O botão Dividir.}
\label{divisao0}
\end{center}
\end{figure}



Pressione o botão \keystroke{Dividir}, e uma caixa de diálogo irá surgir pedindo o número de divisões a fazer.

Por omissão, o programa coloca o valor ''2''.

\begin{figure}
\begin{center}
\includegraphics[height=3cm]{../images/divisao1.png}
\caption[Submanifold]{Caixa de diálogo da divisão de contas.}
\label{divisao1}
\end{center}
\end{figure}


Colocando o número de pessoas que se vai dividir a conta, irá aparecer uma caixa
dividida em três quadros.

O primeiro quadro representa a mesa original com todos os artigos. 

O segundo quadro representa o conteúdo das submesas.

O terceiro quadro indica a submesa que está a ser criada, e o seu valor.


Logo a seguir surgem quatro botões:
\begin{itemize}
\item \keystroke{ $\triangleright$ } - transfere o artigo seleccionado da mesa principal para a submesa seleccionada no quadro da direita;
\item \keystroke{ $\triangleright\triangleright$ } - transfere todos os artigos da mesa principal para a submesa seleccionada no quadro da direita;
\item \keystroke{ $\triangleleft$ } - transfere o artigo seleccionado da submesa de volta para a mesa principal;
\item \keystroke{ $\triangleleft\triangleleft$ } - transfere todos os artigos da submesa de volta para a mesa principal;
\end{itemize}


Tendo noção do que deve figurar em cada conta, vamos completar a primeira sub mesa, como está representado na figura \ref{submesa1}. 

A divisão (artigos/preços) deverá ser realizada de forma manual.


\begin{figure}[h]
\begin{center}
\includegraphics[height=7cm]{../images/divisao2.png}
\caption[Submanifold]{Colocação de artigos na primeira sub mesa.}
\label{submesa1}
\end{center}
\end{figure}


Após a conclusão da inserção de artigos na primeira mesa, vai-se preenchendo as seguintes até que a mesa 
principal fique completamente vazia (figura \ref{submesa2}).


\begin{figure}[h]
\begin{center}
\includegraphics[height=7cm]{../images/divisao3.png}
\caption[Submanifold]{Colocação de artigos na segunda sub mesa.}
\label{submesa2}
\end{center}
\end{figure}


Depois de ter a divisão feita, clique me \keystroke{OK} para aplicar as alterações.


Repare que o valor da mesa passou para metade do valor. 

Ao clicar no botão com o nome da mesa \keystroke{$\bigtriangledown$} poderá aceder a cada submesa individualmente conforme demonstrado na figura \ref{submesa4} .


\begin{figure}[h]
\begin{center}
\includegraphics[height=7cm]{../images/divisao4.png}
\caption[Submanifold]{Como aceder às sub mesas}
\label{submesa4}
\end{center}
\end{figure}

É possível, depois, controlar o que se pretende em cada uma das divisões, podendo mesmo acrescentar ou cancelar produtos.

Pode voltar a unir as sub mesas ou redistribuir os artigos pressionando novamente em \keystroke{Dividir}.


% \begin{bclogo}[couleur=blue!10,arrondi=0.1,logo=\bclampe,ombre=true]
\bcdica{Divisão de contas}{
Se carregar no quadro do total de conta, surgirá no ecrã uma calculadora para auxiliar o operador em algum cálculo que seja necessário.
}

\begin{figure}[h]
\begin{center}
\includegraphics[height=4cm]{../images/calculadora.png}
\caption[Submanifold]{Calculadora disponível pressionando no campo do ''Total''.}
\label{submesa4}
\end{center}
\end{figure}

\newpage
\section{Fluxo documental nas mesas}
\label{sec:fluxoMesas}

No nível das \textsc{Zonas} ao pressionar o botão com a mesa pretendida a aplicação muda
para o nível das \textsc{Mesas}, mostrando os produtos organizados por índices e páginas.
Neste nível, deverá escolher os produtos pretendidos, pressionando o botão com o
seu nome. Estes serão apresentados numa listagem com o conteúdo de todos os
produtos a fornecer.


\begin{figure}[h]
\begin{center}
\includegraphics[height=7cm]{../images/fluxoMesas.png}
\caption[Submanifold]{Fluxo Documental no Nivel Mesa}
\label{fig:fluxoMesas}
\end{center}
\end{figure}



Para um melhor entendimento, passa-se a expor o fluxo de documentos dentro do programa.


Seleccionando uma mesa, e colocando artigos, estes deverão ficar com uma cor vermelha (ou outra dependendo da configuração). 

Isto indica que o produto em causa não deu entrada em produção. Cancelando o artigo nesta altura, dará lugar a um movimento de \textbf{Cancelamento}.

Cada produto pode estar associado a uma impressora de produçao distinta, e quando se pressiona o botão \keystroke{Pedir}, os produtos passam a outro estado. 

No ecrã as cores do texto ficam diferentes, e deve ficar claro para o operador que aqueles produtos já estão em produção.




Neste momento poderá cancelar um artigo neste estado (caso tenha privilégios para tal) e se o fizer irá dar origem a um documento de \textbf{Anulação}. Nas impressoras onde foram emitidas as ordens de produção será impresso um papel de anulação com a descrição dos itens para cancelar a produção.

Imediatamente surge no ecrã uma caixa de diálogo "Cancelamento com desperdício ou sem desperdício?". Um exemplo simples do cancelamento sem desperdício é o caso de uma lata de refrigerante que, não tendo sido aberta, voltará para o frio sem prejuízo.

Simultaneamente na impressora local, deverá sair um documento de \textbf{Justificação de Anulação} com data, hora e o nome dos artigos anulados, com algumas linhas para justificar o motivo da anulação.

Após o serviço realizado, a mesa deverá ser fechada.

Para tal basta pressionar o botão \keystroke{Factura} e caso o cliente solicite, preencher os elementos. Deverá sair um documento com a descrição dos artigos
consumidos.


Pode-se dar a situação de se tratar de uma venda a crédito. Neste caso deve pressionar o botão \keystroke{Cliente}. É uma situação comum na gestão de eventos ou caso haja acordos com empresas na área.
Neste caso a venda será associada a um cliente e posteriormente será transformada numa factura (requer programa de BackOffice Regi).
Para saber mais sobre este assunto, consulte o capítulo correspondente.






\chapter{Artigos de Venda}

% \chapter{Base de Dados}

\begin{figure}[h]
\begin{center}
\includegraphics[height=7cm]{../images/basedados.png}
\caption[Submanifold]{O nível Base de Dados.}
\label{basedados}
\end{center}
\end{figure}


Toda as configurações deste capítulo são efectuadas no nível \textsc{Base de Dados}

Neste nível permite-se fazer toda a configuração do comportamento do sistema e por isso o acesso
ao mesmo aos funcionários deve ser feito de uma forma criteriosa.

Aqui é possível proceder à criação e ou modificação de todas as variáveis de gestão, desde as 
características dos artigos, a forma como aparecem no interface dos funcionários, o horário
de funcionamento do sistema, os privilégios dos funcionários, gestão de cartões cliente, etc etc.

% \nopagebreak





\section{Famílias}

No botão \keystroke{Famílias}, é-nos facultada a possibilidade de configurar famílias de produtos,
facilitando assim o processo de criação de artigos posteriormente.


Pode-se definir regras por grupo de artigos, como as taxas de IVA a aplicar 
em determinados produtos, descontos máximos, direccionamento das ordens de impressão, etc.


Por uma família, entende-se por agrupamento de produtos similares. 

Em vez de atribuir determinadas características por produto pode-se atribuir uma regra para associar a todos os produtos de um mesmo tipo, facilitando 
assim a interacção com o programa.

Para além disso, criando uma família, é possível ter um entendimento sobre a venda de determinados 
tipos de produtos e como estes afectam o volume de vendas. 


% \begin{bclogo}[couleur=blue!10,arrondi=0.1,logo=\bccrayon,ombre=true]
\bcexemplo{Análise por Família}{
Por exemplo, agrupando todos os artigos de um determinado fornecedor numa única família, 
permite fazer uma leitura imediata do volume de vendas de um determinado produto, 
assim como o impacto percentual na facturação aquando o fecho de caixa.
}





\begin{figure}
\begin{center}
\includegraphics[height=6cm]{../images/familias.png}
\caption[Submanifold]{Famílias}
\label{familias}
\end{center}
\end{figure}

\textbf{NOTA:} é contudo possível parametrizar um artigo individualmente bastando para isso ir ao painel
Produtos.



\bcrelacionados{
		Impressoras na secção \ref{impressoras},  na página \pageref{impressoras}.

		Relatório de Produtos na secção \ref{relatorioGestao}, na página \pageref{relatorioGestao}.

		Restrições por Família na secção \ref{TiposCliente}, na página \pageref{TiposCliente}.


	} 


\newpage
\section{Produtos}
\label{ch:produtos}
\begin{figure}[h]
\begin{center}
\includegraphics[height=6cm]{../images/produtos.png}
\caption[Submanifold]{Produtos, tabulador ''Comum''.}
\label{produtos}
\end{center}
\end{figure}



Pressionando o botão \keystroke{Produtos}, surge no ecrã um painel com uma lista do lado esquerdo e quatro tabuladores na parte superior conforme pode ser visto 
na figura \ref{produtos}.


Em baixo há cinco botões: 




\begin{table}[ht]
%  \caption{Tipo de Artigo} 
 \centering
\small
\def\arraystretch{1.5}
 \begin{tabular}{c p{11cm}}  %   l c r r } % centered columns 
 \textbf{Botão} & \textbf{Descrição}  \\ % Garantia & Descrição & Qtd & C. Unit & Subtotal \\ [0.5ex]
%  \multicolumn{6}{l}{\textbf{Servidor}} \\			
 \hline
\keystroke{Criar Produto} &  Cria um produto novo. \\ 
\keystroke{Apagar Produto} & Elimina o produto seleccionado. \\
\keystroke{Duplicar Produto} &  Cria um novo artigo baseado no que está seleccionado actualmente. Pede somente um nome novo. \\
\keystroke{OK} &  Processa na base de dados as operações executadas. \\
\keystroke{Cancelar} &  Cancela todas as operações executadas. \\
 \hline
 \end{tabular}
 \end{table}


Os quatro tabuladores serão explicados seguidamente.

\newpage
\subsection{Comum}

No separador \textbf{Comum} é onde se encontram as principais propriedades de cada artigo, nomeadamente:
% \begin{itemize}
% \item \textbf{Tipo de Artigo} - está relacionado com a categoria em que este artigo se insere. Um artigo pode-se categorizar como:
% \end{itemize}

\begin{table}[ht]
%  \caption{Tipo de Artigo} 
 \centering
\small
\def\arraystretch{1.5}
 \begin{tabular}{c p{11cm}}  %   l c r r } % centered columns 
 \textbf{Tipo de Artigo} & \textbf{Descrição}  \\ % Garantia & Descrição & Qtd & C. Unit & Subtotal \\ [0.5ex]
%  \multicolumn{6}{l}{\textbf{Servidor}} \\			
 \hline
 \textbf{Normal} & Artigo de venda normal. \\
 \textbf{Matéria Prima} & Artigo de compra que será alvo de transformação. \\
 \textbf{Extra de Texto} & artigos que se destinam a descrever uma propriedade do artigo, para informar a produção, sem afectar o custo final do mesmo. Por ex: “mal passado”. \\
 \textbf{Extra de Texto com preço} & Artigos que complementam artigos ou grupo de artigos pré definidos e com custo associado. Por ex. “com ovo”. \\
\textbf{Menu de Acerto} & Permite a integração de todos os artigos de uma mesa num menu único com preço pré-definido. \\ 
 \textbf{Menu pré-definido} & Obriga o utilizador a escolher os componentes de uma ou mais listas pré-definidas impedindo a venda de outros artigos (explicado na página \pageref{sec:menupredefinido}). \\
 \hline
 \end{tabular}
 \end{table}


\begin{table}[ht]
% \caption{} 
 \centering
\small
\def\arraystretch{1.5}
 \begin{tabular}{c p{11cm}}  %   l c r r } % centered columns 
 \textbf{Botão} & \textbf{Descrição}  \\ % Garantia & Descrição & Qtd & C. Unit & Subtotal \\ [0.5ex]
%  \multicolumn{6}{l}{\textbf{Servidor}} \\			
 \hline
 \textbf{Família} & diz respeito à família a que este artigo se insere. \\
 \textbf{Descrição} & o nome que é impresso nos documentos. \\
 \textbf{Nome} & a designação do artigo que irá aparecer no botão - permite várias traduções para o
 caso de haver operadores de diferentes línguas (ver \textbf{perfis} ). \\
 \textbf{Preço} & apresentado com três casas decimais. Cada artigo pode ter até 7 preços diferentes, 
podendo ser cada um atribuído a cada secção (ver \textbf{ \ref{sec:sessoes} Sessoes} pág. \pageref{sec:sessoes}). \\
 \textbf{Código de Barras} & Atribuindo um código ao artigo, pode-se fazer a leitura do 
mesmo com leitor específico, seja este um leitor de mesa, de pistola ou outro leitor similar. \\
 \textbf{PLU} & \emph{Price Look Up Code} ou código de busca de artigo, pode ser usado para 
invocar artigos rapidamente por teclado numérico no ponto de venda ou remotamente. \\
 \textbf{Pede quantidade} & quando activa irá pedir a quantidade de artigos a servir. \\
 \textbf{Pede preço Unitário} &  quando activa pergunta qual o preço do artigo que se vai servir. \\
 \hline
 \end{tabular}
 \end{table}





\newpage
\subsection{Outros}


\begin{figure}[h]
\begin{center}
\includegraphics[height=6cm]{../images/produtos2.png}
\caption[Submanifold]{Tabulador ''Outros'' de Produtos.}
\label{produtos}
\end{center}
\end{figure}




Aqui deverá indicar a Unidade do produto.

Se por exemplo vender produtos ao Kg, deverá definir aqui a unidade.


Se pretender
que este produto seja pedido deverá indicar a que \textbf{Impressora Virtual} ele pertence
através do botão \keystroke{Definir}, depois da impressora desejada se encontrar seleccionada.



\bcrelacionados{		
		Ver mais informações sobre o assunto de impressoras em \ref{impressoras} Impressoras,  na página \pageref{impressoras}.
	} 


% No separador Tipo deverá indicar o Tipo de produto , se é um produto normal, extra,
% entre outros. Por fim no último separador Páginas deverá indicar para que página irá
% o produto criado para isso deverá seleccionar a página pretendida e fazer \keystroke{Adicionar}.




\newpage
\subsection{Relações}

\begin{figure}[h]
\begin{center}
\includegraphics[height=6cm]{../images/produtos3.png}
\caption[Submanifold]{Tabulador Relações de Produtos.}
\label{produtos}
\end{center}
\end{figure}






Aqui configuram-se os produtos unidos e as páginas dos artigos que devem figurar em menu (caso o produto seja do tipo menu pré-definido --- ver \ref{sec:menupredefinido})

		
\bcrelacionados{	\textbf{Criação de produtos para reserva} na secção \ref{ch:restauracaocolectiva},  na página \pageref{ch:restauracaocolectiva}.
}





\subsection{Página}


\begin{figure}[h]
\begin{center}
\includegraphics[height=6cm]{../images/produtos4.png}
\caption[Submanifold]{Tabulador Páginas de Produtos.}
\label{produtos}
\end{center}
\end{figure}



\begin{figure}
\end{figure}


Aqui definem-se a página ou as páginas onde o artigo deve estar apresentado.
É possível o mesmo artigo estar apresentado em diversos locais. Por exemplo o artigo café, 
pode ser configurado para se poder ser invocado nas páginas ''rápida'', ''cafetaria'' ou ''sobremesas''.


\bcrelacionados{		
	\textbf{Páginas} na secção \ref{sec:paginas},  na página \pageref{sec:paginas}.

	\textbf{Famílias} na secção \ref{sec:paginas},  na página \pageref{sec:paginas}.

	\textbf{Sessões} na secção \ref{sec:paginas},  na página \pageref{sec:paginas}.

	\textbf{Zonas} na secção \ref{sec:paginas},  na página \pageref{sec:paginas}.

	\textbf{Reservas} na secção \ref{ch:restauracaocolectiva},  na página \pageref{ch:restauracaocolectiva}.

	\textbf{Relatório de Produtos} na secção \ref{relatorioGestao},  na página \pageref{relatorioGestao}.
}



\section{Criação de Menu pré-definido}
\label{sec:menupredefinido}

Quando há necessidade de criar um menu com componentes específicos é importante guardar registo de quais os itens que foram vendidos para assim se relacionar 
nas quebras de produtos.

Para isso existe o tipo de produto \textbf{Menu pré-definido}.

Para usar este menu é necessário criar uma página com somente os artigos que se pretendem no menu. 

Por exemplo ''menu cachorro + sumo lata''.

Cria-se uma página com um nome identificável, por exemplo ''menu bebidas'', e nessa página inserem-se as bebidas que se pretende que componham o menu.

De seguida vai-se a \keystroke{Produtos}, cria-se o artigo, tendo o cuidado de no tabulador \keystroke{Comum}, 
o colocar do \textbf{Tipo} \keystroke{Menu pré-definido}.

Depois no tabulador \keystroke{Relações} pressiona-se \keystroke{Adicionar} e selecciona-se a página que se criou.





\section{Criação de artigos a peso\/volume\/medida}
\label{sec:pesomedida}

\begin{HUGE}
algum texto em falta
\end{HUGE}


\section{criação de artigos do tipo ''extra''}

\begin{huge}
algum texto em falta
\end{huge}


\section{criação de artigos do tipo ''Menu Acerto''}

\begin{huge}
algum texto em falta
\end{huge}



\chapter{Configuração do sistema}

Anteriormente mencionou-se que cada artigo poderá ter vários preços. Neste capítulo vai-se abordar
como configurar um sistema composto por vários postos de forma a poder segmentar as funções.

Pretende-se expôr como os vários conceitos do programa se relacionam de forma a responder às necessidades dos clientes.

% Vamos abordar portanto:
% \begin{itemize}
% \item Zonas e Mesas
% \item Sessões
% \item Perfis
% \item Empregados
% \end{itemize}



Há procedimentos fundamentais, a partir dos quais se pode configurar um espaço, e em 
qualquer altura se pode redefinir as regras para adaptar o sistema o negócio.


\begin{itemize}
\item Colocar os \textbf{Produtos} dentro de uma ou mais \textbf{Páginas}  (ver \textbf{\ref{sec:paginas} Páginas} na página \pageref{sec:paginas});

\item Colocar os \textbf{Páginas} dentro de um ou mais \textbf{Índices}  (ver \textbf{\ref{sec:indices} Índices} na página \pageref{sec:indices});

\item Uma vez definido o espaço, é necessário criar uma ou mais \textbf{Zonas} e os respectivas \textbf{Mesas} onde serão prestados os serviços (ver \textbf{\ref{sec:mesaszonas} Mesas e Zonas} na página \pageref{sec:mesaszonas});

\item Em cada \textbf{Perfil}, define-se que \textbf{Zona} é que este terá acesso (ver \textbf{\ref{sec:perfis} Perfis} na página \pageref{sec:perfis};


\item Nas zonas define-se os índices disponíveis em cada zona 

\item Atribui-se os perfis criados aos funcionários e equipamentos (ver \textbf{\ref{sec:equipasfuncionarios} Equipas e Funcionários} na página \pageref{sec:equipasfuncionarios});

\item Nas sessões definem-se os preços de que se devem fazer por Zona. (ver \textbf{\ref{sec:sessoes} Sessões} na página \pageref{sec:sessoes}).
\end{itemize}




% \section{Produtos}
% 
% No nível de \textsc{Base de Dados}, tocar no botão \keystroke{Produtos}.
% 
% Dentro da janela pressionar o
% botão \keystroke{Novo produto}, aparecerá um teclado virtual onde deve atribuir um nome ao
% produto. %  este não deve conter apóstrofes pois não é compatível com o sistema.
% 
% 
% \begin{figure}[h]
% \begin{center}
% \includegraphics[height=6cm]{../images/produtos.png}
% \caption[Submanifold]{Quadro Produtos}
% \label{produtos}
% \end{center}
% \end{figure}


% Logo que coloque o nome, surge uma caixa de diálogo para que se seleccione a família a 
% que pertence o produto. Seleccionando a família, ficará associado a este produto as 
% regras inerentes a essa família:
% \begin{itemize}
% \item Tipo de produto;
% \item Desconto máximo;
% \item Imposto associado;
% \item se Pede quantidade; 
% \item se Pede valor do produto.
% \end{itemize}

% Se não atribuir nenhuma família, poderá editar os campos manualmente, contudo é fortemente recomendado
% que o produto faça parte de uma família para que possa extrair informações posteriormente nos relatórios.

% Seleccione o \keystroke{Tipo de Produto} como "Normal". Isto significa que se trata de um produto de venda 
% simples sem componentes. Posteriormente serão abordados os outros tipos de produto.


% No campo \keystroke{Preço 1} coloque o preço a que pretende vender o artigo. 

% O sistema trabalha com três casas decimais e este caso pode ser útil em situações de 
% venda de produtos a peso ou volume. 

% Pode estipular até sete preços por artigo, mas considere por agora definir apenas o 
% valor para "Preço 1". 


% Defina a taxa de impostos associada ao produto. No caso de um artigo com uma taxa de IVA de 23\%, deverá
% preencher o campo \keystroke{Taxa \% 1} com o valor "23".

% Por defeito o programa assume que os artigos têm  os valores dos impostos incluídos no preço final,
% embora seja possível configurar o comportamento do programa para se comportar de forma diferente.
%% De seguida ainda nesse mesmo separador Comum deve indicar a que Família pertence
%% esse produto, o Preço também deve ser indicado assim como a respectiva Taxa.


% Por fim no último separador \textsc{Páginas}, deverá indicar para que página(s) irá
% o produto criado para isso deverá seleccionar a página pretendida e pressionar \keystroke{Adicionar}.

% \begin{bclogo}[couleur=blue!10,arrondi=0.1,logo=\bclampe,ombre=true]{Mesmo produto em várias páginas}
% Repare que podendo associar o mesmo produto a várias páginas, poderá colocar o mesmo artigo 
% à venda em espaço diferentes do programa facilitando a navegação entre páginas de produtos.
% \end{bclogo}

% \begin{bclogo}[couleur=blue!10,arrondi=0.1,logo=\bccrayon,ombre=true]{Configuração de Sessões}
% Num espaço que trabalhe simultaneamente como restaurante e como cafetaria pode por exemplo colocar
% o artigo "Água 0.33 cl" na página dos artigos de "cafetaria" e na página de "Águas". 
% 
% Este pequeno exemplo permite que o operador consiga ter sempre visíveis os artigos que têm mais saída a 
% cada momento acelerando o processo de registo. 
% \end{bclogo}



% Ainda nesse mesmo separador através do botão \keystroke{Propriedades} pode mudar a cor do
% botão, a cor da letra entre outros, para se tornar mais apelativo e fácil de se
% localizar. 

% No separador \keystroke{Outros} deverá indicar a Unidade do produto. Se pretender
% que este produto seja pedido deverá indicar a que Impressora Virtual ele pertence
% através do botão Definir, depois da impressora desejada se encontrar seleccionada.
% No separador Tipo deverá indicar o Tipo de produto , se é um produto normal, extra,
% entre outros. Por fim no último separador Páginas deverá indicar para que página irá
% o produto criado para isso deverá seleccionar a página pretendida e fazer \keystroke{Adicionar}.

% No fim fazer \keystroke{OK} para salvar a informação. 

% Esta ultima operação de incluir produtos
% nas páginas também pode ser feita através do botão Páginas, para isso deverá
% aceder ao botão Páginas, dentro da janela, seleccionar a página pretendida, e
% através do botão Produtos colocar o(os) produto(s) pretendidos para dentro da
% página. Para terminar fazer OK. 

% Também podemos definir famílias de produtos
% através do botão Famílias de Produtos para isso deverá clicar no botão Nova
% Família , aparecerá um teclado virtual onde deve dar um nome à família, de seguida
% deverá definir as propriedades da família tal como indicado na janela. No fim fazer
% OK para salvar as alterações.

% \subsection{Como alterar o preço de um produto?}

% Através do botão Produtos no nível de Base de Dados. Este irá abrir uma janela com
% todos os produtos, deverá seleccionar o produto pretendido e pressionar em cima do
% preço para que este lhe abra uma janela onde poderá apagar e colocar o novo
% preço. No fim fazer OK para salvar as alterações. Também através do botão Mudar
% Preço, no nível de Mesas, poderá alterar o preço, mas só se este se encontrar
% activo.

\newpage
\section{Páginas}
\label{sec:paginas}

Através do botão \keystroke{Páginas} no nível de \textsc{Base de Dados} pode aceder à configuração das páginas do software. 


Este irá abrir um painel em que do lado esquedo irá apresentar todas as páginas existentes, do lado direito 
os detalhes de cada uma.

Para criar uma nova terá que clicar no botão \keystroke{Nova Página}.

De seguida surgirá um teclado virtual onde deve dar um nome à nova
página. 


\begin{figure}[h]
\begin{center}
\includegraphics[height=6cm]{../images/paginas.png}
\caption[Submanifold]{Painel de Páginas}
\label{fig:paginas}
\end{center}
\end{figure}

% \subsection{Como adicionar um produto a uma página?}
% Através do botão Páginas no Nível de Base de Dados. Este irá abrir uma janela com
% todas as páginas existentes. Seleccionar a página pretendida e clicar no botão
% Produtos. Este irá abrir uma janela com todos os produtos existentes na Base de
% Dados do lado esquerdo da janela. Deverá então seleccionar o produto pretendido e
% clicar na seta para a direita para este ser inserido na página. No fim fazer OK para
% guardar a informação.



Para colocar os produtos dentro dessa página, verique que tem essa página seleccionada no lado esquerdo e pressione o botão \keystroke{Produtos}.

Este abrirá uma janela com todos os produtos existentes. Deverá
seleccionar o produto pretendido e passá-lo um a um para a página através do botão \keystroke{ $\triangleright$ }, que indica para o lado direito.

Se quiser tirar um produto da página terá de o
seleccionar e clicar no botão \keystroke{ $\triangleleft$ }.

\begin{figure}[h]
\begin{center}
\includegraphics[height=6cm]{../images/paginasprodutos.png}
\caption[Submanifold]{Colocar produtos nas Páginas}
\label{fig:paginasprodutos}
\end{center}
\end{figure}





Os produtos ficarão listado na página por ordem alfabética.

Também é permitida, através
do botão \keystroke{Estruturar Página} alterar a disposição e dimensão dos botões segundo um critério específico.

No fim
fazer \keystroke{OK} para salvar a informação introduzida. 


\begin{figure}[h]
\begin{center}
\includegraphics[height=6cm]{../images/estruturarPaginas.png}
\caption[Submanifold]{Configuração de ''Estruturar Páginas''}
\label{fig:estruturarPaginas}
\end{center}
\end{figure}

% \begin{bclogo}[couleur=blue!10,arrondi=0.1,logo=\bctakecare,ombre=true]
\bcatencao{Estruturar Páginas}{
Ao estruturar a página está a forçar o software a usar uma regra rígida para dispor os produtos.

Como tal, para acrescentar um novo produto a uma página estruturada, deverá sempre desfazer a estrutura previamente feita.
}

\newpage
\section{Índices}
\label{sec:indices}
Depois de criar a página deverá ir ao
botão \keystroke{Índices} no nível \textsc{Base de Dados} e colocar a página no Índice pretendido para
que a operação anterior surta efeito senão a página não aparecerá.

Para isso deverá
na janela dos Índices, no separador Páginas, seleccionar o índice pretendido e
colocar a página dentro desse índice. Para salvar as alterações pressione \keystroke{OK}.


Existem dois tipos de \textbf{Índices}: os índices principais e os índices normais.

Para que um índice possa aparecer numa \textbf{Zona}, este deverá ser do tipo ''Principal''.

Os Índices normais têm como propósito agregar páginas para permitir maior navegabilidade 


No painel Índices, marque o Índice que pretende configurar e pressione \keystroke{Estrutura}.

De seguida irá surgir o seguinte painel.

Do lado esquerdo surgem as páginas criadas e os índices normais (não Principais).

do lado direito aparece o conteúdo do índice que está a ser configurado.

A ordem como os itens surgem no ecrã será sempre primeiramente as páginas, ordenadas por data de criação, seguidas pelos índices.


\begin{figure}[h]
 \begin{center}
\includegraphics[height=6cm]{../images/indiceReservasEstrutura.png}
\caption[Submanifold]{Painel de estruturação de um Índice.}
\label{fig:painelestruturaindice}
\end{center}
\end{figure}

\bcrelacionados{Consulte a Secção \textbf{\ref{sec:mesaszonas} Mesas e Zonas} na página \pageref{sec:mesaszonas}}



\subsection{Índices de Índices}
% \begin{bclogo}[couleur=blue!10,arrondi=0.1,logo=\bccrayon,ombre=true]{Índices de Índices}
Na situação em que se tem várias dezenas de artigos da mesma categoria que se pretendem controlar,
torna-se complicado e moroso para o operador gerenciar as saídas destes produtos.


A forma elegante de resolver esta questão, será poder colocar várias páginas dentro de um índice e depois agrupar toda a informação.

O exemplo mais vulgar desta situação, são as garrafas de vinho. 

Para resolver esta situação a forma mais elegante de resolução é colocar um diferenciador que permita 
ao operador reconhecer imediatamente a posição de cada item.

Por exemplo:

Índice Principal: Vinho
\begin{itemize}
\item índice "V. Branco" 
\item índice "V. Tinto"
\end{itemize}

índice "V. Branco" com as páginas: Douro, Dão,  Alvarinho e Setúbal.  

e índice "V. Tinto" com as páginas Douro, Alentejo, Setúbal e Bairrada


\begin{figure}[h]
 \begin{center}
\includegraphics[height=6cm]{../images/indicesindices.png}
\caption[Submanifold]{Exemplo de uso de Índices não principais.}
\label{indicesindices}
\end{center}
\end{figure}





\newpage
\section{Mesas e Zonas}
\label{sec:mesaszonas}
Através do botão \keystroke{Mesa} no nível de \textsc{Base de Dados}. Este irá abrir uma janela onde
deverá carregar no botão \keystroke{Nova Mesa}.


Deverá dar um nome à mesa e fazer \keystroke{OK} para
guardar o registo.


Características das Mesas


 \begin{table}[ht]
%  \caption{Hardware} 
 \centering
\small
\def\arraystretch{1.5}
 \begin{tabular}{c p{12cm}}  %   l c r r } % centered columns 
 \textbf{Botão} & \textbf{Significado}  \\ % Garantia & Descrição & Qtd & C. Unit & Subtotal \\ [0.5ex]
%  \multicolumn{6}{l}{\textbf{Servidor}} \\			
 \hline
\textbf{Nome} & Nome com que a mesa é reconhecida no software. \\
\textbf{Herdar Zona} & As mesas podem ser configuradas individualmente, ou pode-se atribuir as regras a uma \textbf{Zona} e definir que a Mesa herda as regras definidas na zona em que se encontra.\\
\textbf{Valor Mínimo} & Pode-se definir que a mesa tem um valor mínimo a ser cobrado. \\
\textbf{Valor Máximo} &  Define o valor máximo até ao qual é possível facturar. Atingindo o valor máximo, não é possível colocar mais artigos nas mesas. \\
\textbf{Valor Fixo} &  Define-se o valor definido a facturar pela mesa. \\
\textbf{Código de Barras} &  Código inserido via cartão ou elemento biométrico que dará acesso à mesa \\
%  \multicolumn{6}{l}{\textbf{Bastidor}} \\			
 \hline
 \end{tabular}
 \end{table}


Na situação de \textbf{Herdar Zonas} estar habilitado, é possível definir qualquer regra na zona, pois o botão das configurações fica desabilitado, como pode ser observado na figura \ref{fig:mesaherdazona}.

\begin{figure}[h]
 \begin{center}
\includegraphics[height=6cm]{../images/mesas.png}
\caption[Submanifold]{Mesa definida para herdar as regras que estão definidas em Zona.}
\label{fig:mesaherdazona}
\end{center}
\end{figure}


Apesar da designação ''Mesa'', esta não tem de se referir forçosamente a uma mesa física, podendo estar associada a um cliente que se identifica com um cartão.

Nessa situação faz sentido voltar a descrever os significados de cada uma das configurações, pelo que:

--- a referência ao valor mínimo (por exemplo, cartão de consumo mínimo de uma discoteca, em que o cliente pagará sempre um valor 
previamente acordado. 

--- O valor máximo, diz respeito a uma situação em que um cliente possa carregar um crédito na mesa e que poderá consumir posteriormente.

--- O valor fixo diz respeito a por exemplo a um serviço de catering, em que se define um preço fixo por pessoa.



\newpage

\subsection{Zonas}



\begin{figure}[h]
\begin{center}
\includegraphics[height=6cm]{../images/z1.png}
\caption[Submanifold]{Detalhes de Zona.}
\label{fig:detalheszona}
\end{center}
\end{figure}

Entende-se por zonas as áreas que compõem o estabelecimento comercial.

Cada zona é composta por mesas e tem regras específicas no que diz respeito ao comportamento de cada zona.

O painel das \textbf{Zonas} tem três tabuladores: \textbf{Detalhes}, \textbf{Mesas} e \textbf{Índices}.

Nos detalhes será definido a forma como cada zona se deverá comportar.

Nas mesas define qual ou quais a mesas que fazem parte de uma zona. 

No tabulador \textbf{Índices} pode-se seleccionar os produtos que estão disponíveis em cada Zona.

\bcexemplo{Conceito de Zona}{Uma zona poderá ser um balcão composto por cinco lugares sentados, ou um salão com quinze mesas. Cada mesa poderá estar apenas numa zona.} 

\newpage
Entrando no tabulador mesas, terá do lado direito todas as mesas que não estejam ainda associadas a nenhuma zona, e do lado esquerdo já atribuídas a esta zona. 

Para que depois a mesa apareça, deverá ir ao botão \keystroke{Zonas} no
nível Base de Dados e no separador Mesas deverá colocar a mesa na zona
pretendida. 

No fim fazer \keystroke{OK} para guardar a informação.


As mesas serão sempre apresentadas no nível \textsc{Zona} pela ordem em que foram criadas.


\begin{figure}[h]
\begin{center}
\includegraphics[height=6cm]{../images/z2.png}
\caption[Submanifold]{Mesas da Zona.}
\label{fig:mesaszonas}
\end{center}
\end{figure}







Entrando no tabulador \textbf{Índices}, terá do lado direito todas os Índices Principais, e do lado esquerdo já atribuídas a esta zona. 




\begin{figure}[h]
\begin{center}
\includegraphics[height=6cm]{../images/z3.png}
\caption[Submanifold]{Índices por Zona.}
\label{fig:indiceszona}
\end{center}
\end{figure}

\bcrelacionados{Consulte a secção \textbf{\ref{sec:indices} Índices} na página \pageref{sec:indices}}

\newpage
\section{Perfis de máquina e de Utilizador}
\label{sec:perfis}


É a partir deste painel que se define os privilégios dos utilizadores e\/ou as restrições de cada equipamento.

O painel apresenta do lado esquerdo a lista de perfis criados na base de dados, e à esquerda as definições de cada um, sendo 
parametrizados a partir dos tabuladores:


 \begin{table}[ht]
%  \caption{Hardware} 
 \centering
\small
\def\arraystretch{1.5}
 \begin{tabular}{c p{12cm}}  %   l c r r } % centered columns 
 \textbf{Tabulador} & \textbf{Significado}  \\ % Garantia & Descrição & Qtd & C. Unit & Subtotal \\ [0.5ex]
%  \multicolumn{6}{l}{\textbf{Servidor}} \\			
 \hline
\textbf{Detalhe} &  onde se define o nome, o tipo de perfil (utilizador\/máquina), a moeda por defeito e o idioma. \\
\textbf{Zona} & as zonas do espaço comercial a que este perfil tem acesso. \\
\textbf{Propriedades} &  as características gráficas do perfil (que definem os botões disponíveis). \\
\textbf{Cantinas} &  indicação do Índice a usar nas reservas para cantinas, se aplicável.\\
%  \multicolumn{6}{l}{\textbf{Bastidor}} \\			
 \hline
 \end{tabular}
 \end{table}




% \begin{itemize}
% \item \textbf{Detalhe} --- onde se define o nome, o tipo de perfil (utilizador\/máquina), a moeda por defeito e o idioma;
% \item \textbf{Zona} --- as zonas do espaço comercial a que este perfil tem acesso; 
% \item \textbf{Propriedades} --- as características gráficas do perfil (que definem os botões disponíveis)
% \item \textbf{Cantinas} --- indicação do Índice a usar nas reservas para cantinas, se aplicável.
% \end{itemize}









\begin{figure}[h]
\begin{center}
\includegraphics[height=6cm]{../images/perfilestrangeiro.png}
\caption[Submanifold]{Perfis.}
\label{fig:perfis}
\end{center}
\end{figure}





Aqui se associam perfis gráficos a utilizadores e máquinas.

Aqui pode-se determinar a língua, zonas e permissões que ficam vinculadas a cada utilizador ou máquina.

itens relacionados: zonas, sessões, máquinas, reservas.




\subsection{Criando um perfil em lingua estrangeira}

Através do botão \keystroke{Perfis} no nível \textsc{Base de Dados}. 
Crie um novo perfil de utilizador com o botão \keystroke{Novo Perfil} e altere o idioma para inglês.

Verifique que as restantes configuraçoes estão como o perfil de funcionário já existente.



\begin{figure}[h]
\begin{center}
\includegraphics[height=6cm]{../images/perfilestrangeiro.png}
\caption[Submanifold]{A criar um perfilo novo}
\label{sessoes3}
\end{center}
\end{figure}

Pressione \keystroke{OK}

Pressione \keystroke{Empregados} e altere o perfil do empregado para o novo perfil criado.


Pressione \keystroke{OK}


\begin{figure}[h]
\begin{center}
\includegraphics[height=6cm]{../images/perfilestrangeiro2.png}
\caption[Submanifold]{Alterar o perfil em ''Empregados''}
\label{sessoes3}
\end{center}
\end{figure}


Vá ao Nível \textsc{Utilizador} e autentique-se com o funcionário que modificou.

\begin{figure}[h]
\begin{center}
\includegraphics[height=6cm]{../images/perfilestrangeiro3.png}
\caption[Submanifold]{painel em inglês}
\label{sessoes3}
\end{center}
\end{figure}


Tudo fica na lingua escolhida.




\newpage
\section{Sessões}
\label{sec:sessoes}


\index{Sessões}
\index{Sessões!Operador}
\index{Sessões!Horário}
\index{Sessões!Preços por Zona}
\index{Preços por Zona}



As sessões podem ser configuradas através do botão \keystroke{Sessões} no nível \textsc{Base de Dados}.


Em sessões, pode-se definir os turnos de trabalho que se fazem por dia. As sessões podem ser parametrizadas para serem manuais ou automáticas 
na secção de \textbf{Definições Globais}.

Nas sessões pode-se definir:
\begin{itemize}
\item Os horários de funcionamento de cada turno;
\item O preço a aplicar em cada zona em cada turnos;
\item Os funcionários por turno;
\end{itemize}

% Uma sessão permite a definição
% do preço pretendido por zona e é definida por períodos de tempo específicos e pelos
% empregados que nela operam. 


O painel \textsc{Sessões} divide-se em duas partes. Na parte da direita estão listadas todas as sessões já criadas. 
Do lado esquerdo estão as configurações, que são feitas ao longo de três tabuladores, como será explicado seguidamente.


Para criar uma nova sessão deverá, na janela de
Sessões, no nível \textsc{Base de Dados}, pressionar na opção \keystroke{Nova Sessão}, dar um nome à nova
sessão.

\subsection{Detalhes da Sessão}

O primeiro tabulador diz respeito aos \textbf{Detalhes}. Aqui define-se o nome da sessão e atribui-se o preço a cobrar em cada 
zona do espaço comercial.


\begin{figure}[h]
\begin{center}
\includegraphics[height=6cm]{../images/sessoes1.png}
\caption[Submanifold]{Detalhes das Sessões}
\label{sessoes1}
\end{center}
\end{figure}

\bcdica{Preços por zona}{Por omissão, somente o primeiro preço do artigo é preenchido. certifique-se de que tem todos os preços atribuidos aos artigos da zona em causa antes de activar esta opção.}


\bcrelacionados{Ver secção \textbf{\ref{produtos} Produtos} na página \pageref{produtos}}




\subsection{Intervalos da Sessão}


De seguida, no separador \textbf{Intervalos}, deverá indicar a hora de
inicio e a hora de fim da sessão.

Em vez de seleccionar as sessões dia a dia, pode especificar um horário e clicar na opção \keystroke{Adicionar Semana}, para que este
assuma esse horário para todos os dias da semana. 

Posteriormente pode definir se pretende que a sessão seja controlada manualmente pelo operador, ou que seja o relógio do computador a determinar a abertura e fecho das sessões

\begin{figure}[h]
\begin{center}
\includegraphics[height=6cm]{../images/sessoes2.png}
\caption[Submanifold]{Intervalos das Sessões}
\label{sessoes2}
\end{center}
\end{figure}

\bcrelacionados{

Ver secção \textbf{\ref{gestorSessao} Gestor de Sessão} na página \pageref{gestorSessao}

Ver secção \textbf{\ref{defglobaisSessoes} Definições Globais} na página \pageref{defglobaisSessoes}}



\subsection{Empregados por Sessão}


Pode achar conveniente que nao apareçam todos os funcionários na página de autenticação, especialmente se tiver vários 
turnos a decorrer e tiver um mapa de turnos definido.

Neste painel, poderá eleger os funcionários que irão aparecer na página de autenticação.

\begin{figure}[h]
\begin{center}
\includegraphics[height=6cm]{../images/sessoes3.png}
\caption[Submanifold]{Empregados das Sessões}
\label{sessoes3}
\end{center}
\end{figure}




Por fim ira até ao separador \textbf{Empregados} e colocar os empregados que 
estão a trabalhar na respectiva sessão.

\bcexemplo{Empregados por turno} {Se houver uma sessão de manhã e outra à tarde, poderá colocar alguns empregados no turno da manhã e outros no turno da tarde e assim o ecran inicial fica mais 
fácil de usar. Desta forma, se impossibilita que um funcionário registe por vez de outro
que não esteja presente.}

\bcdica{Não perca o histórico de Empregados}{Se por qualquer motivo deixar de contar com um colaborador, não o elimine da lista dos empregados, mas remova-o das sessões de trabalho. Assim na eventualidade de voltar a contar com ele, poderá voltar a tirar relatórios de empregado em períodos alargados.}



No fim fazer \keystroke{OK} para salvar as
alterações.  Se se enganar, bastará pressionar \keystroke{Cancelar} para voltar as colocar as definições que estavam anteriormente.

\newpage
\section{Funcionários}
\label{sec:equipasfuncionarios}

Quando se define um negócios, acabamos por desenvolver relações, e haverá elementos em quem confiamos e assim podemos 
definir estruturas de privilégios.


\bcexemplo{Exemplo de aplicação prática}{
Convencionemos que existe um  perfil de \textbf{Supervisor}, que pode criar os produtos, alterar preços, gerir empregados, consultar relatórios, modificar as páginas. 

Podemos ainda definir o papel de \textbf{Gerente} que poderá somente consultar o botão História, sem poder emitir Notas de Crédito, e extrair os relatórios.


Podemos ainda definir o papel de \textbf{Empregado de Mesa}  que poderá somente fazer pedidos, e não terá acesso à caixa.

Tendo estes conceitos definidos  sobre a forma que se vai desenrolar o negócio, pode-se então criar três equipas cada uma com o seu respectivo perfilo de utilizador.

Pode obviamente definir outros nomes e outros critérios de privilégio. Este exemplo serviu para melhor explicar 
os seguintes botões.
}

\subsection{Equipas}


Através do Botão Equipas no nível Base de Dados. Este irá abrir uma janela, onde
para criar uma nova Equipa terá de tocar no botão Nova Equipa e dar um nome a
esta. De seguida irá escolher as definições para essa nova Equipa. No fim fazer OK
para guardar as alterações.


\subsection{Empregado}
Tocando no botão Empregados no nível Base de Dados, pode definir os funcionários
da loja. Para além dos dados pessoais, deve definir os tipos de acções operativas
permitidas e a integração, ou não, nas equipas de trabalho existentes.

Estas definições não alteram o perfil de utilizador, mas restringem a utilização de
funcionalidades nele disponíveis (ex: possibilidade de utilizar códigos para autentificar
clientes).

Quando pertence a uma equipa, o empregado assume por defeito as
definições dela, mas cada empregado pode ser tratado de forma independente,
através da definição das suas permissões próprias, que se sobrepõe às definições
do grupo. Além destes dados, deverá ser indicado um perfil para cada um dos
empregados.

\bcdica{Privilégios de funcionários}
{Em qualquer altura pode alterar os privilégios dos funcionários individualmente, sem recorrer a 
\textbf{Equipas}. As equipas servem para facilitar a criação de operadores de forma a evitar erro.}


\bcexemplo{Como criar um empregado}
{Para criar um empregado, deverá depois de dar um nome e atribuir privilégios no painel \textbf{Empregados}, deverá 
ir a até a \keystroke{Sessões} ao tabulador \textbf{Empregados} e colocá-lo nas sessões correspondentes}



\section{Outros}
\subsection{Controlo de caixa e Fundo maneio}


Para haver fundo maneio, é necessário definir as moedas e notas que se pretendem controlar
isso é feito no botão \keystroke{Contagem}.

Depois de definidas as moedas, vá ao nível Gerente e através do botão \keystroke{Fundo de Maneio} faça  o controlo do
Fundo de Maneio. Este pode ser definido por máquina ou por utilizador.

\subsection{Como aplicar um desconto a um produto?}
Através do botão Generosidade que se encontra no nível de Gerente, podemos
aplicar descontos sobre a venda, estes podem ser aplicados sobre o total da mesa
e/ou produto a produto através de uma percentagem e/ou de um valor.

\subsection{Versão}
\begin{figure}[h]
\begin{center}
\includegraphics[height=5cm]{../images/versao.png}
\caption[Submanifold]{Versão de Software.}
\label{versao}
\end{center}
\end{figure}






Pressionando no botão \keystroke{Versão} é apresentada a versão do software.

Em Portugal, para que o software seja considerado legal deverá ser certificado.
Assim, a versão do software Koncepto deverá ser \textbf{4.2.29} ou superior, conforme pode ser visto na imagem da figura \ref{versao}.

O software \textbf{Koncepto v 4.2.29}, produzido pela empresa \textbf{Sindesi - Sistemas e Soluções Informáticas, Lda} foi certificado pela Autoridade Tributária pelo \textbf{número 1183} a \textbf{6 de Abril de 2011}, conforme pode ser obsrevado na figura \ref{softwareCertificadoAT}. 

\begin{figure}[h]
\begin{center}
\includegraphics[height=5cm]{../images/softwareCertificadoAT.png}
\caption[Submanifold]{Menção do software no portal das finanças.}
\label{softwareCertificadoAT}
\end{center}
\end{figure}




Para que o software esteja devidamente licenciado, não poderá apresentar nesse quadro a palavra ''DEMO'' --- indicadora de que está em modo de demonstração.




A versão do software é composta por:
\begin{itemize}	
\item Versão dos binários
\item Versão da base de dados
\item Módulos extra habilitados (standart, PDAs e cantinas)
\end{itemize}	

Apresentamos a tabela de versões lançadas deste software, desde a sua origem em 2004.
A versão começa em 4, pois foi concebida em sequência da versão 3 que foi
desenvolvida para ser executada em MS-DOS e que se quebrou o suporte a partir de 2005.





\begin{table}[h]
\begin{center}
\begin{tabular}{|c|c|c|}
\hline
Data de Lançamento &	Versão de Binários &	Versão de Base de dados \\ \hline
2004 & 	4.2.0  &	 1.2.0 \\
2005 &	4.2.19 &	1.2.19 \\
2005 &	4.2.20 & 	1.2.20 \\
2006 &	4.2.23test3 & 	1.2.23 \\
2007 &	4.2.24 & 	1.2.24 \\
2007 &	4.2.25 & 1.2.25 \\
2008 &	4.2.26 & 	1.2.26 \\
2010 &	4.2.29 & 	1.2.29 \\
2011 &	4.3.01 & 	1.3.01 \\
2012 &	4.3.02 & 	1.3.02 \\
2012 &	4.3.03 & 	1.3.03 \\ \hline
\end{tabular}
\caption[Submanifold]{Tabela de Correspondência de Versões.}
\end{center}
\end{table}

\newpage

\subsection{Manutenção}


\begin{figure}[h]
\begin{center}
\includegraphics[height=5cm]{../images/manutencao.png}
\caption[Submanifold]{Manutenção.}
\label{manutencao}
\end{center}
\end{figure}



Pressionando o botão \keystroke{Manutenção} tem-se acesso a um painel que permite verificar a ligação ao exterior. 

O painel está divido em duas partes. Na parte de cima existe o botão \keystroke{Mostrar} que quando pressionado irá 
fazer uma  ligação à Internet e devolver na janela ao lado o IP externo.

A mensagem que é apresentada imediatamente após pressionar o botão é ''Loading page, please wait...'' o que se traduz
para ''A carregar a página, por favor aguarde...''. 

Caso tenha havido sucesso a obter ligação externa, deverá a aparece a mensagem ''Current IP Address: \#\#\#.\#\#\#.\#\#\#.\#\#\#'' que 
se traduz para ''Endereço IP Actual: \#\#\#.\#\#\#.\#\#\#.\#\#\#''.


Na parte de baixo do painel tem um quadro designado por \textbf{Ligação} onde figura o \textbf{IP do escritório} com o valor ''127.0.0.1''.

Estas definições dizem respeito a uma configuração de BackOffice obsoleta e somente se mantém por motivos históricos.

Os botões \keystroke{Ligar} e \keystroke{Desligar} são botões parametrizáveis que poderão ser usados chamar funções específicas ao sistema operativo.







\subsection{Relatórios de Contabilidade}

\begin{figure}
\includegraphics[width=15cm]{../images/diagramaRelatOficiais.png}
\caption[Submanifold]{Funcionamento dos Relatórios Oficiais.}
\end{figure}

No relatório de contabilidade figuram os elementos correspondentes à sessão de vendas seleccionada.

Será impossível extrair o relatório de contabilidade de uma sessão actualmente em curso ou de
 um dia em que não tenha havido uma sessão a decorrer

Para uma dada sessão válida, o relatório de contabilidade irá mostrar os determinados elementos:
\begin{itemize}
\item Dados da empresa
\item Intervalo de Vendas a Dinheiro emitidas
\item Intervalo de Notas de Crédito emitidas
\item Total ilíquido facturado
\item Total líquido facturado
\item Total de IVA por taxa de Incidência
\end{itemize}

Opcionalmente pode-se adicionar outras informações, como por exemplo: 
\begin{itemize}
\item Número de clientes
\item Consumo médio realizado por cada cliente.
\item etc.
\end{itemize}

\newpage
\section{Definições Globais}
\subsection{Sessões}
\begin{figure}[h]
\begin{center}
\includegraphics[height=6cm]{../images/defglobaisSessoes.png}
\caption[Submanifold]{Comportamento de Sessões}
\label{fig:defglobaisSessoes}
\end{center}
\end{figure}

Estando o painel de Sessões bem configurado


\bcrelacionados{Ver secção \textbf{\ref{sec:sessoes} Sessões} na página \pageref{sec:sessoes}}

\subsection{Pagamentos}

\begin{figure}[h]
\begin{center}
\includegraphics[height=6cm]{../images/defglobaisPagamentos.png}
\caption[Submanifold]{Métodos de Pagamento.}
\label{fig:defglobaisPagamentos}
\end{center}
\end{figure}

Este painel define qual o método de pagamento de deve aparecer por omissão no quadro de facturação. 

Permite ainda inserir as taxas de câmbio para moedas estrangeiras, caso necessário.

\bcrelacionados{Ver \textbf{\ref{fatura} Faturação} na página \pageref{fatura}}



\subsection{Gaveta}

\begin{figure}[h]
\begin{center}
\includegraphics[height=6cm]{../images/defglobaisGaveta.png}
\caption[Submanifold]{Configuração de Gaveta.}
\label{fig:defglobaisGaveta}
\end{center}
\end{figure}

Neste quadro define-se os métodos de pagamento que abrem gaveta.

Pode-se por exemplo definir que o operador não possa abrir a gaveta manualmente, e que esta somente poderá abrir em caso de pagamentos a dinheiro.

\bcrelacionados{Ver \textbf{\ref{botoesCliente} Venda} na página \pageref{botoesCliente}}

\begin{figure}[h]
\begin{center}
\includegraphics[height=6cm]{../images/defglobaisCancelamentos.png}
\caption[Submanifold]{Cancelamentos.}
\label{fig:defglobaisCancelamentos}
\end{center}
\end{figure}

\bcrelacionados{Ver \textbf{\ref{botoesCliente} Venda} na página \pageref{botoesCliente}}

\begin{figure}[h]
\begin{center}
\includegraphics[height=6cm]{../images/defglobaisCantinas.png}
\caption[Submanifold]{Definição de Cantinas e Reservas.}
\label{fig:defglobaisCantinas}
\end{center}
\end{figure}

\bcrelacionados{Ver \textbf{\ref{ch:restauracaocolectiva} Restauração Colectiva} na página \pageref{ch:restauracaocolectiva}}

\begin{figure}[h]
\begin{center}
\includegraphics[height=6cm]{../images/defglobaisVerifMesa.png}
\caption[Submanifold]{Comportamento das mesas.}
\label{defglobaisVerifMesa}
\end{center}
\end{figure}

\bcrelacionados{Ver \textbf{\ref{ch:restauracaocolectiva} Restauração Colectiva} na página \pageref{ch:restauracaocolectiva}}



\begin{figure}[h]
\begin{center}
\includegraphics[height=6cm]{../images/defglobaisPeso.png}
\caption[Submanifold]{Pesos e Balanças.}
\label{defglobaisPeso}
\end{center}
\end{figure}

\begin{figure}[h]
\begin{center}
\includegraphics[height=6cm]{../images/defglobaisTaxas.png}
\caption[Submanifold]{Cálculo de Taxas.}
\label{defglobaisTaxas}
\end{center}
\end{figure}

\bcperigo{Cálculo de Taxas}
{Esta secção é de extremo cuidado, e deverá ter a certeza do que 
pretende fazer. 

Caso não tenha a certeza das configurações que pretende, não altere os parâmetros.} 

\begin{figure}[h]
\begin{center}
\includegraphics[height=6cm]{../images/defglobaisSeries.png}
\caption[Submanifold]{Séries Documentais.}
\label{defglobaisSeries}
\end{center}
\end{figure}

\begin{figure}[h]
\begin{center}
\includegraphics[height=6cm]{../images/defglobaisOutros.png}
\caption[Submanifold]{Configurações Extra.}
\label{defglobaisOutros}
\end{center}
\end{figure}











\chapter{Clientes e cartões cliente}



A secção clientes diz respeito aos clientes ''especiais''. 

Entende-se por cliente especial, todo aquele que é identificado e reconhecido pelo sistema. 

Um cliente especial poder ter associado regras de consumo, tais como:
\begin{itemize}
\item ter as suas despesas transformadas em consumo a crédito (para pagamento posterior);
\item ter descontos em determinados artigos;
\item não ter qualquer regra de pagamento, mas no entanto aparecem os seus elementos no documento de venda.
\end{itemize}

Para além disto, é possível no caso de haver descontos impor restrições de quantidade de artigos consumidos
por dia, impedindo assim situações de possível abuso.



Existem três botões para gerenciar a forma como se pode faturar os clientes.
Premissas fundamentais:
\begin{itemize}
\item Cada \textbf{Cliente} é único e pode estar associado a um ou mais tipos de \textbf{cartão cliente}
\item Cada \textbf{Cartão Cliente} deverá ter um código de acesso único e inequívoco;
\item Cada \textbf{Cartão Cliente} está associado a uma regra definida de \textbf{Tipo de Cartão}.
\end{itemize}

% \begin{figure}[h]
% \includegraphics[width=15cm]{../images/diagramaClientes.jpg}
% \caption[Submanifold]{Estrutura de cliente especial.}
% \end{figure}


Para aceder às configurações existem três botões no nível \textsc{Base de Dados} como pode ser visto na figura \ref{botoesCliente}.



\begin{figure}[h]
\begin{center}
\includegraphics[height=2cm]{../images/botoesCliente.png}
\caption[Submanifold]{Botões para configuração de Clientes}
\label{botoesCliente}
\end{center}
\end{figure}

Para um melhor entendimento, serão apresentadas várias configurações possíveis. 
A partir destas será simples obter um entendimento para permitir adaptar o programa à realidade de cada negócio.

\begin{itemize}
\item Tabela de preços específica para um grupo de utilizadores;
\item Consumos internos de empregados;
\item Vendas a crédito para empresas;
\item Venda a cliente com NIF.
\end{itemize}

Existe uma outra particularidade conseguida a partir destes botões que será abordada 
no capítulo respeitante à Restauração Colectiva (página \pageref{ch:restauracaocolectiva}).

\newpage
\section{Tabela de preços específica para um grupo de utilizadores}

Por vezes há necessidades de fornecer bens com uma tabela diferente. 

Vamos tomar por exemplo o caso de um centro comercial que requer uma tabela de preços especiais para Lojistas, ou
um acordo com uma empresa local em oferecer desconto percentual sobre algumas categorias de artigos.


Pressiona-se o botão \keystroke{Tipos de Cliente} e vamos criar o tipo ''Lojistas''.

Certifique-se de deixa um visto na opção \textbf{Paga}.

Se qualquer pessoa puder dar o desconto, no campo \textbf{Tipo de Validação} selecciona ''Pode Validar''.

\begin{figure}[h]
\begin{center}
\includegraphics[height=6cm]{../images/TipoClienteLojista.png}
\caption[Submanifold]{Criação da Regra ''Lojistas''}
\label{TipoClienteLojista}
\end{center}
\end{figure}

E aplicar as restrições, para isso pressiona-se em \keystroke{Restriçoes por Produto}
e vamos definir que no artigo ''Carioca Café'' paga apenas 70\% do valor.

\begin{figure}[h]
\begin{center}
\includegraphics[height=6cm]{../images/TipoClienteLojistaRestricoes.png}
\caption[Submanifold]{Aplicaçao das restriçoes em ''Lojistas''}
\label{TipoClienteLojistaRestricoes}
\end{center}
\end{figure}


Estando a regra criada, é necessário atribuir a regra a um cartão.

Pressiona-se o botão \keystroke{Tipos de Cartões} e faz-se corresponder a regra ao cartão.


\begin{figure}[h]
\begin{center}
\includegraphics[height=6cm]{../images/TipoCartaoEmpregados.png}
\caption[Submanifold]{Associação da regra Empregados ao Cartão Empregados}
\label{TipoCartaoEmpregados}
\end{center}
\end{figure}




De seguida criar o cliente ''Lojista''. Deve-se ir ao botão \keystroke{Cliente} e pressionar \keystroke{Novo}.

Preencha os elementos \textsc{Info} conforme figura \ref{clienteLojista}.

\begin{figure}[h]
\begin{center}
\includegraphics[height=6cm]{../images/clienteLojista.png}
\caption[Submanifold]{Criação de Cliente tipo ''Lojista''}
\label{clienteLojista}
\end{center}
\end{figure}


De seguida pressione em \keystroke{Cartões}. Vamos criar o cartão ''4'' que tem vai ser um cartão do tipo ''Cartão Lojista''.

No código do cartão, insira o número ''4''. Este código deverá ser invocado quando for para identificar o cliente.

Alternativamente, pode passar um código de barras, um cartão de banda magnético ou outro dispositivo biométrico. A informação 
irá ser mostrada no campo ''MSR'' e futuramente só terá de voltar este método para invocar o cliente.

\begin{figure}[h]
\begin{center}
\includegraphics[height=6cm]{../images/clienteLojistaCartao.png}
\caption[Submanifold]{Criação de Cliente tipo ''Lojista''}
\label{clienteLojistaCartao}
\end{center}
\end{figure}

\newpage
\section{Consumos internos de empregados}

Muitos estabelecimentos necessitam controlar os consumos dos funcionários.

Esta ferramenta permite ter um controlo absoluto das saídas de produtos.

Primeiro vamos criar uma regra.

Por exemplo, vamos admitir que um funcionário poderá consumir um artigo da família \textbf{diárias} por dia.



\begin{figure}[h]
\begin{center}
\includegraphics[height=7cm]{../images/clienteFuncionario.png}
\caption[Submanifold]{Criação de funcionário como cliente}
\label{clienteFuncionario}
\end{center}
\end{figure}

\begin{figure}[h]
\begin{center}
\includegraphics[height=7cm]{../images/clienteFuncionarioCartao.png}
\caption[Submanifold]{Atribuição do código único ao funcionário}
\label{clienteFuncionarioCartao}
\end{center}
\end{figure}






mas pode ser feito um cartão genérico para muita gente

por exemplo, tens um cartão VIP
que te dá acesso imediato a uma tabela de preços nova
por exemplo, os preços de lojista
na altura de pagar, o programa vai buscar a tabela de preços nova
certo
em vez de usar a tabels pre-definida
podes fazer vários tipos de regras
por exemplo, desconto em 30\% em tudo excepto nas bebidas alcoolicas




O programa Koncepto trabalha na base de
\begin{itemize}
\item Clientes
\item Cartões Cliente
\end{itemize}

A razão disto, é que um cliente poderá ter outras opções 

um conceito de cartões cliente 


Na configuração de cartões de consumo  é importante restringir os c 




\chapter{Restauração colectiva}
\label{ch:restauracaocolectiva}

\section{Consumos a crédito}

O Koncepto poderá ser configurado para que realize sempre vendas a dinheiro,
sempre consumos a crédito, ou funciona de uma forma composta. Avançamos com
indicações base para esta última situação, a mais complexa, que integra as
anteriores.

\begin{figure}[h]
\begin{center}
\includegraphics[height=7cm]{../images/TiposCliente.png}
\caption[Submanifold]{Tipos de Cliente}
\label{TiposCliente}
\end{center}
\end{figure}

\begin{figure}[h]
\begin{center}
\includegraphics[height=7cm]{../images/TiposClienteRestricoes.png}
\caption[Submanifold]{Tipos de Cliente}
\label{TiposClienteRestricoes}
\end{center}
\end{figure}

\begin{figure}[h]
\begin{center}
\includegraphics[height=7cm]{../images/TiposClienteRestrProd.png}
\caption[Submanifold]{Tipos de Cliente}
\label{TiposClienteRestrProd}
\end{center}
\end{figure}




\subsection{Como faço um consumo a crédito?}
Um consumo a crédito pode ser efectuado através do botão \keystroke{Cliente}  onde
surge uma janela com três opções: Venda Normal onde volta à venda normal,
Código de Cliente onde se pode colocar o número de cartão manualmente, e por
último Cancelar.

Através do botão \keystroke{Cartão de Cliente} (se activo), no nível \textsc{Mesa}, abrirá uma janela que
permite fazer uma venda a determinado cliente. Isto possibilita atribuir pontos,
descontos ou consumos a clientes específicos.

% \subsection{Como faço o cancelamento de um consumo a crédito?}
% 
% Através do botão \keystroke{História} no nível \textsc{Gerente}.
% Para cancelar o registo de um consumo
% a crédito deve identificar e seleccionar o consumo na listagem apresentada e
% pressionar no botão \keystroke{Nota de crédito}.

\section{Configuração de ambiente para se fazer reservas}

O Koncepto poderá ser configurado de forma a emitir reservas. Estas podem ser
apenas indicativas ou debitadas, e neste último caso, através da adição automática
de um produto ao documento de consumo. Estas parametrizações deverão ser
definidas na configuração inicial, aquando da instalação do sistema,

\subsection{Como criar Índices para as reservas?}
Para criar Índices para as Reservas deverá primeiro ir ao nível \textsc{Base de Dados} e
no botão \keystroke{Índices} criar um Índice principal ex: \textbf{Índice Reservas}.

\begin{figure}[h]
\begin{center}
\includegraphics[height=7cm]{../images/indiceReservas.png}
\caption[Submanifold]{Criação do índice de Reservas}
\label{indiceReservas}
\end{center}
\end{figure}




De seguida deverá
criar um sub – índice ex: \textbf{Menus Reserva}. Para isso \textbf{não deve} marcar a opção \textbf{Índice Principal} conforme visto na figura \ref{indiceReservas}. 

Dentro deste sub–índice deverá colocar a página com os produtos
pretendidos para o nível \textsc{Reservas}. Para isso deve clicar no botão \keystroke{Estrutura} e na
janela onde diz ''Páginas'' colocar a página pretendida seleccionado-a e colocando-a
através do botão \keystroke{ $\triangleright$ }.

\begin{figure}[h]
\begin{center}
\includegraphics[height=7cm]{../images/indiceReservasEstrutura.png}
\caption[Submanifold]{Estrutura do índice de Reservas}
\label{indiceReservas}
\end{center}
\end{figure}






%   \keystroke{Page $\uparrow$} \keystroke{Esc} \keystroke{F1}
\begin{figure}[h]
\begin{center}
\includegraphics[height=7cm]{../images/perfisCantinas.png}
\caption[Submanifold]{Habilitação do índice de Reservas}
\label{perfisCantinas}
\end{center}
\end{figure}


 
Depois, no Índice principal clicar no botão \keystroke{Estrutura}.
Na parte da janela onde diz Índices colocar o sub- índice \textbf{Menus Reserva} através
da seta indicada para a direita. Fazer \keystroke{OK} para guardar as alterações.

\begin{figure}[h]
\begin{center}
\includegraphics[height=7cm]{../images/perfisCantinas.png}
\caption[Submanifold]{Habilitação do índice de Reservas}
\label{perfisCantinas}
\end{center}
\end{figure}




Para que depois este apareça no Nível \textsc{Reservas}, deverá ir ao nível de \textsc{Base de Dados},
 botão \keystroke{Perfis}, e no separador \textbf{Cantinas}, na opção Índice para as Reservas, 
conforme pode ser visto na figura \ref{perfisCantinas}.

Deverá indicar então o Índice principal criado anteriormente \textbf{Índice Reservas}.

Não esquecer de o indicar para todos os perfis. No fim fazer \keystroke{OK}.


\section{Criar e cancelar reservas}

Quando se pretende criar uma reserva debitada no acto de emissão é necessário,
durante a venda ou realização de um consumo a crédito, carregar no botão \keystroke{Reserva}
para aceder ao nível \textsc{Reservas}, onde poderá identificar o produto pretendido e a
data do seu consumo. 

Após completar esta operação, deverá gerar a reserva
pressionando o botão \keystroke{Terminar} – por vezes também identificado como \keystroke{Imprimir}.

Automaticamente voltará ao nível \textsc{Mesa} para terminar o registo.

Quando pretender criar uma reserva que será debitada apenas no acto do consumo,
esta é realizada de forma independente do registo do consumo. A qualquer
momento poderá aceder ao nível \textsc{Reservas}, identificar o cliente através da
passagem do cartão, ou introdução do PIN, carregando no botão \keystroke{PIN} ou \keystroke{Cliente},
identificar os dados da reserva e terminar ou imprimir.

Em certas parametrizações, após um registo a aplicação pode abrir
automaticamente o nível das reservas.

\subsection{Como cancelar uma reserva?}
Caso seja uma reserva pré paga será necessário criar uma nota de crédito (ver como criar uma Nota de Crédito na página \pageref{historia}),
 que
anulará tanto a venda ou consumo como a reserva propriamente dita.

Em termos
operacionais, caso exista senha esta deve ser recolhida pelo operador.

Se refizer o
documento a todas as linhas do documento original são importadas para o novo,
incluindo a reserva, mas a senha foi cancelada.

Deverá então cancelar a reserva
e adicionar de novo, fazendo o registo e forçando a criação da nova marcação e
emissão de senha.

No caso de uma reserva indicativa, basta aceder ao nível
\textsc{Reserva}, identificar o cliente através da passagem do cartão de cliente no MSR ou
carregando no botão \keystroke{Código cliente} e digitar o respectivo PIN e, de seguida,
identificar a data e sessão da reserva em causa, obtendo como resultado a
visualização produto anteriormente reservado. 

Para cancelar o produto deverá seleccioná-lo,
e pressionar o botão \keystroke{Cancelar}. 

Para finalizar, deverá carregar no botão \keystroke{Terminar}.


\subsection{Como verificar o estado das reservas?}
Através do botão \keystroke{Estado das Reservas} que se encontra no nível de Gerente. Aqui
pode observar os produtos reservados que já foram consumidos e ou aqueles por
consumir. Através do botão \keystroke{Relatórios de Gestão}, pode imprimir um \textbf{Relatório de
Reservas}.

\subsection{Como bloquear a criação de reservas?}
Através do botão \textbf{Bloquear Reservas} no nível Gerente. Este bloqueia a realização de
reservas para a sessão escolhida, a partir desse momento. Desta forma, no futuro
não será possível efectuar mais reservas para essa mesma sessão.

\subsection{Como fazer um controlo das reservas não consumidas?}
No Nível de \textsc{Gerente}, no botão \keystroke{Definições Globais}, no separador Cantinas
deve activar a opção \textbf{Converte reservas não consumidas em consumos de cantina}.

\begin{figure}[h]
\begin{center}
\includegraphics[height=6cm]{../images/defglobaisCantinas.png}
\caption[Submanifold]{Definições Globais, tabulador ''Cantinas''}
\label{defglobaisCantinas}
\end{center}
\end{figure}


Esta opção vai agarrar nas reservas não consumidas, após o fecho de sessão, de
clientes com consumo a crédito e marca-as como consumidas e faz uma venda a
dinheiro para a reserva. Esta opção funciona apenas com clientes com consumo a
crédito.



% \begin{bclogo}[couleur=blue!10,arrondi=0.1,logo=\bclampe,ombre=true]

\section{Definição da data sugerida para reserva}



Aceder ao nível \textsc{Gerente}, seleccionar o botão \keystroke{Definições Globais} e no separador
\textbf{Cantinas} pode definir se a data sugerida para a reserva é dinâmicamente alterada
para o próximo dia de trabalho ou se é fixa. Neste caso é necessário identificar a
data pretendida.

\subsection{Feriados}

Aqui configuram-se as partes referentes aos intervalos em que as reservas de cantinas não devem funcionar.

Considerando os dias úteis definidos em sessões, esta função permite criar as excepções em que as cantinas não funcionam.

\begin{figure}[h]
\begin{center}
\includegraphics[height=7cm]{../images/feriados.png}
\caption[Submanifold]{Marcação de dias em que a cantina não está em funcionamento}
\label{feriados}
\end{center}
\end{figure}



\chapter{Relatórios}
\label{ch:relatorios}
\label{reports}
O Koncepto poderá ser configurado de forma a permitir a emissão de relatórios para
certos utilizadores. Por defeito, todos os relatórios estão disponíveis para
utilizadores do tipo gerente ou superior. 


%  \vspace{5mm}
Existem três botões que emitem relatórios, que estão acessíveis no nível \textsc{Gerente} 
a partir dos botões \keystroke{Relatórios Oficiais} e \keystroke{Relatórios Gestão} 
e \keystroke{SAF-T PT}.


Para além destes botões, que permitem fazer análises de períodos anterior, existe ainda o botão \keystroke{História} que permite analisar em detalhe
todas as operações executadas ao longo do dia de trabalho.

 \vspace{5mm}
Os botões \keystroke{Relatórios Oficiais} e \keystroke{Relatórios Gestão} 
apresentam os dados directamente na impressora destinada à impressão de relatórios,
e deverão estar acessíveis apenas aos operadores com maiores privilégios. O botão \keystroke{História} apresenta a informação no ecrã, mas pode-se restringir
as funcionalidades na configuração de privilégios.

\vspace{5mm}
O botão \keystroke{SAF-T PT} gera o ficheiro SAF-T(PT) (Standard Audit File for Tax Purposes – Portuguese version).
Um ficheiro normalizado (em formato XML) com o objectivo de permitir uma exportação fácil, e em qualquer altura, de um conjunto predefinido de registos contabilísticos, de facturação, de documentos de transporte e recibos emitidos, num formato legível e comum, independentemente do programa utilizado, sem afectar a estrutura interna da base de dados do programa ou a sua funcionalidade.

\section{História}
\label{historia}

Para perceber o conceito de História, é necessário entender os tipos de documentos gerados pelo programa.

Há seis tipos de documentos gerados. Quatro deles já foram analisados na secção 
\ref{sec:fluxoMesas} \textbf{Fluxo documental nas mesas}.

Faltam analisar dois que não podem ser executados nas mesas, pois exigem privilégios superiores, e é neste quadro que poderão ser gerados, que são a \textbf{Nota de Crédito} e a \textbf{Anulação de Consumo a Crédito}

%% \begin{itemize}
% \item \textbf{Nota de Crédito} --- Uma Nota de Crédito, é um documento de 
% anulação de um documento de venda. Neste deverá ser mencionado o documento que lhe deu origem. Tal como a Facura, trata-se de um documento contabilistico.
% \item \textbf{Anulação de Consumo a Crédito} --- 
% \end{itemize}


Tipos de documentos que poderão ser encontrados no painel História



\begin{table}[ht]
%  \caption{Hardware} 
 \centering
\small
\def\arraystretch{1.5}
 \begin{tabular}{p{4cm} p{11cm}}  %   l c r r } % centered columns 
 \textbf{Documento} & \textbf{Descrição}  \\ % Garantia & Descrição & Qtd & C. Unit & Subtotal \\ [0.5ex]
%  \multicolumn{6}{l}{\textbf{Servidor}} \\			
 \hline
\textbf{Factura} & Cada factura corresponde a uma transacção. As facturas são documentos legais que cumprem o código do IVA. \\
\textbf{Consumo a Crédito} & Documentos referentes a vendas que não foram facturadas. \\
\textbf{Anulação} & Cancelamentos efectuados após se ter dado as ordens de produção. \\
\textbf{Cancelamento} & Cancelamentos antes de fazer o pedido para produção. \\
\textbf{Nota de Crédito} &  Documento de anulação de uma factura. Deve mencionar a factura a que diz respeito. \\
\textbf{Anulação de Consumo a Crédito} & Anulação do registo de um consumo a crédito. \\
%  \multicolumn{6}{l}{\textbf{Bastidor}} \\			
 \hline
 \end{tabular}
 \end{table}





\begin{figure}[h]
\begin{center}
\includegraphics[height=12cm]{../images/historia.png}
\caption[Submanifold]{Painel História}
\label{fig:historia}
\end{center}
\end{figure}


O painel \textbf{História} divide-se em três partes:
\begin{itemize}
\item \textbf{Documentos} --- é apresentada a lista de documentos de acordo com os parâmetros especificados em filtro e intervalo. Premindo o cabeçalho, 
poderá ordenar os elementos de acordo com a coluna seleccionada. Na ausência de parâmetros, 
serão apresentados todos os documentos do dia corrente ordenados por data.
\item \textbf{Filtro} --- Especifica a busca por um tipo ou número de documento;
\item \textbf{Intervalo} --- Data de início e de fim para apresentar os documentos.
\end{itemize}

Por baixo surgem quatro botões:

\begin{table}[ht]
%  \caption{Hardware} 
 \centering
\small
\def\arraystretch{1.5}
 \begin{tabular}{c p{12cm}}  %   l c r r } % centered columns 
 \textbf{Botão} & \textbf{Significado}  \\ % Garantia & Descrição & Qtd & C. Unit & Subtotal \\ [0.5ex]
%  \multicolumn{6}{l}{\textbf{Servidor}} \\			
 \hline
\keystroke{Facturas} & Imprime uma segunda via do documento seleccionado. \\
\keystroke{Ver Documento} & Apresenta no ecrã o conteúdo do documento seleccionado .\\
\keystroke{Nota de Crédito} & Faz uma anulação do documento seleccionado, gerando dependo do documento, uma \textbf{Nota de Crédito} ou uma \textbf{Anulação de Consumo a Crédito}. \\
\keystroke{Fechar} & Fecha o painel. \\
%  \multicolumn{6}{l}{\textbf{Bastidor}} \\			
 \hline
 \end{tabular}
 \end{table}




%  \begin{figure}[h]
% \begin{center}
% \includegraphics[height=7cm]{../images/historia.png}
% \caption[Submanifold]{Painel História}
% \label{Historia}
% \end{center}
% \end{figure}


\begin{figure}[h]
\begin{center}
\includegraphics[height=7cm]{../images/fluxoMesas.png}
\caption[Submanifold]{Fluxo Documental no Nivel Mesa}
\label{fig:fluxoMesasHistoria}
\end{center}
\end{figure}





% \subsection{Como fazer uma nota de crédito?}
%  Através do botão \keystroke{História} que se encontra no nível \textsc{Gerente}.

%  Deverá localizar o
% documento a creditar, ver o documento de modo a garantir que seleccionou o
% documento correcto e então emitir a nota de crédito, pressionando o botão \keystroke{Nota de Crédito}
% mesmo nome.



% \subsection{Como reimprimir um documento?}
% Esta função pode ser efectuada na janela aberta pelo botão \keystroke{História}, no nível de
% Gerente. Aí, deverá localizar o documento em causa e depois pressionar \keystroke{Factura}.

\bcdica{Impressão de documentos}
{Nem todos os documentos permitem re-impressão.

Por omissão somente as Facturas podem ser re-impressas, mas pode-se permitir a re-impressão de notas de crédito em Definições Globais }

\bcrelacionados{Consulte a secção Definições Globais.  }






% \subsection{Como ver um documento já emitido?}
% Através do botão \keystroke{História} que se encontra no nível de \textsc{Gerente}.

% Este botão permite o
% acesso a documentos passados que surgem da utilização da aplicação. Podem ser
% vistos os detalhes dos documentos, imprimir duplicados,
% criar notas de crédito e anular consumos.






\subsection{Criação de uma nota de crédito de um documento emitido}

Caso tenha registado uma venda indevidamente, a única forma de proceder à anulação é emitindo um documento de Nota de Crédito.

Para tal terá de ir até ao nível \textsc{Gerente} e em \keystroke{História}, procurar o documento que gerou e proceder à anulação pressionando o botão \keystroke{Nota de Crédito}. %  (figura \ref{Historia}).


Quando se pressiona em  \keystroke{Nota de Crédito}, surge uma nova caixa de diálogo que pergunta se se pretende refazer a factura ou não.

Independentemente da resposta, será impressa uma Nota de Crédito com os dados vigentes na factura em causa. 

Caso se tenha optado por refazer a factura, o sistema irá criar uma mesa temporária onde colocará todos os artigos que foram anulados para que se possa fazer a correcção e imprimir a factura correta.


% \begin{bclogo}[couleur=blue!10,arrondi=0.1,logo=\bclampe,ombre=true]
\bcdica{Configuração de Sessões}{
Todas as notas de crédito emitidas, são realizadas sobre vendas efectuadas.

O sistema não permite gerar crédito a favor de outrem sem que tenha havido anteriormente um documento de venda.
}






\newpage

\section{Relatórios Oficiais}



% através de objectos diferentes: Relatórios Oficiais e Relatórios Gestão.

\begin{figure}[h]
\begin{center}
\includegraphics[height=3cm]{../images/relatorios.png}
\caption[Submanifold]{Botões de Relatórios}
\label{relatorioGestao}
\end{center}
\end{figure}





\begin{figure}[h]
\begin{center}
\includegraphics[height=6cm]{../images/relatorioOficial.png}
\caption[Submanifold]{Relatórios Oficiais}
\label{relatorioOficial}
\end{center}
\end{figure}



O botão \keystroke{Relatórios Oficiais} imprime um relatório das
vendas totais de um determinado período de tempo. 
Nos relatórios oficiais, só pode aceder a
informação relativa a sessões de trabalho encerradas. Os dados da sessão a decorrer nunca são incluídos.
Enquanto a sessão não for encerrada, aparecerá uma mensagem de
alerta a dizer que a operação não pode ser concluída. 
Isto acontece porque uma das informações do relatório é o intervalo de documentos gerado, e enquanto 
a sessão não estiver terminada, não é possível determinar o número do último documento.
Esta restrição impede que haja relatórios referentes ao mesmo dia com valores diferentes, algo que pode causar alguma confusão em termos contabilísticos.

% Os  primeiros dois relatórios são apresentados num ambiente gráfico comum, o qual se passa a
% descrever.

À esquerda, apresenta-se o selector \textbf{Documentos}. Neste caso somente estará disponível o relatório \textbf{Relatório Contabilidade}
%  Deverá ser seleccionado o relatório a apresentar.
À direita, no campo \textbf{Data}, define-se o intervalo de tempo que se deseja extrair o relatório. A data por omissão é a data do equipamento.


% de tempo no qual tiveram início as sessões a incluir no relatório,
Abaixo, o campo \textbf{Sessão} lista as sessões definidas no sistema.

Deverá usar o botão \keystroke{Definir} para marcar as sessões que deseja imprimir.


% Deverá ser indicado o intervalo de tempo a
% 
% considerar, e os tipos de sessões a incluir.

No fundo, um botão para \keystroke{Imprimir} o relatório com as características identificadas, e
outro para \keystroke{Sair}.




\begin{figure}
\includegraphics[width=15cm]{../images/diagramaRelatOficiais.png}
\caption[Submanifold]{Funcionamento dos Relatórios Oficiais.}
\label{funcrelatoficiais}
\end{figure}


Na figura \ref{funcrelatoficiais} explica-se resumidamente o funcionamento dos Relatórios Oficiais.

Todas as vendas só podem ser efectuadas em Sessões. Quando se imprime um relatório referente a uma Sessão, 
o programa vai analisar os documentos dentro das sessões e vai agrupá-los para reunir a informação de uma forma condensada.

% No relatório de contabilidade figuram os elementos correspondentes à sessão de vendas selecionada.

% Será impossível extrair o relatório de contabilidade de uma sessão actualmente em curso ou de
%  um dia em que não tenha havido uma sessão a decorrer

Para uma dada sessão válida, o relatório de contabilidade irá mostrar os determinados elementos:
\begin{itemize}
\item Dados da empresa;
\item Intervalo de documentos de venda emitidos;
\item Intervalo de Notas de Crédito emitidas;
\item Total ilíquido facturado;
\item Total líquido facturado;
\item Total de IVA por taxa de incidência.
\end{itemize}

Opcionalmente pode-se adicionar outras informações, como por exemplo: 
\begin{itemize}
\item Número de clientes;
\item Consumo médio realizado por cada cliente.
\item etc.
\end{itemize}



% dd\vspace{10mm}
% \begin{bclogo}[couleur=blue!10,arrondi=0.1,logo=\bclampe,ombre=true]
\bcdica{\emph{Relatório de Contabilidade para uso restrito}}{
O botão \keystroke{Relatórios Oficiais} está acessível através de um botão específico para que se possam conceder privilégios de acesso a diferentes operadores
para permitir uma melhor gestão do espaço comercial.
%  este irá apresentar um conjunto
% de dados necessários à contabilidade.
}



% \begin{bclogo}[couleur=blue!10,arrondi=0.1,logo=\bctakecare,ombre=true]
\bcatencao{\emph{''A operação não pode ser terminada''}}{
A data de um relatório \textbf{diz sempre respeito ao início de uma sessão}. Se por por acaso surgir o erro \emph{''A operação não pode ser terminada''} 
deve-se ao caso de não se ter aberto a sessão nesse dia, e as vendas acumularão com o dia(s) anterior(es)
}

% \vspace{10mm}
% \begin{bclogo}[couleur=red!20,arrondi=0.1,logo=\bcbombe,ombre=true]
\bcperigo{Relatório em falta}{
Pode suceder que um determinado relatório esteja em falta e só se aperceba algum tempo depois.
Verifique no botão \keystroke{Gestor de Sessão} de que \textbf{não tem} nenhuma sessão por fechar.
Caso exista uma sessão em aberto, deverá fechá-la então voltar a tentar retirar o relatório. 

O facto de ter uma sessão antiga em aberto não tem significado contabilista. O conceito de ''Sessão'' é uma consequência de uma metodologia de gestão 
e não afecta a contabilidade.
}





\newpage

\section{Relatórios de Gestão}
% Aceder ao nível de \textsc{Gerente}, e pressionar o botão \keystroke{Relatórios de Gestão}, obtendo uma
% janela na qual deverá identificar o Período de Tempo e o Tipo de Sessões que teve
% início nesse período, que pretende incluir no relatório. Após estas opções, deverá
% seleccionar o relatório pretendido e carregar em Imprimir.
\begin{figure}[h]
\begin{center}
\includegraphics[height=6cm]{../images/relatorioGestao.png}
\caption[Submanifold]{Relatórios de Gestão}
\label{relatorioGestao}
\end{center}
\end{figure}


O comportamento deste botão é similar ao dos \keystroke{Relatórios Oficiais}, 
passando a estar disponíveis os seguintes relatórios:



\begin{table}[ht]
%  \caption{Hardware} 
 \centering
\small
\def\arraystretch{1.5}
 \begin{tabular}{p{4cm} p{11cm}}  %   l c r r } % centered columns 
 \textbf{Documento} & \textbf{Descrição}  \\ % Garantia & Descrição & Qtd & C. Unit & Subtotal \\ [0.5ex]
%  \multicolumn{6}{l}{\textbf{Servidor}} \\			
 \hline

\textbf{Relatório de Empregados} & Discrimina todas as acções dos utilizadores por sessão, os
cancelamentos, as anulações com e sem desperdício, os descontos as notas
de crédito e vendas a dinheiro. Ainda indica o inicio e o fim de ponto com o
total de horas.. Pode escolher tirar um relatório de todos os empregados ou
um a um seleccionando o empregado pretendido. \\
 \textbf{Relatório de Produtos} & 
Apresenta todos os produtos vendidos e também por família de produtos. \\
 \textbf{Relatório de Vendas por Hora} & 
Disponibiliza o total de vendas a dinheiro por hora. \\
 \textbf{Relatório de Vendas por Máquina} & 
Apresenta as vendas feitas na máquina onde se encontra. \\
 \textbf{Relatório de Reservas (Cantina)} & 
Lista as reservas realizadas onde inclui a sessão, o produto e a quantidade. \\
 \textbf{Relatório de Vendas (Cantina)} & 
Exibe a listagem de consumos a crédito realizados onde inclui a sessão a
data o produto e a quantidade. \\
 \textbf{lista de Produtos} & 
Imprime uma lista com todos os produtos definidos. \\
%  \multicolumn{6}{l}{\textbf{Bastidor}} \\			
 \hline
 \end{tabular}
 \end{table}









\begin{figure}
\includegraphics[width=15cm]{../images/diagramaRelatGestao.jpg}
\caption[Submanifold]{Funcionamento dos Relatórios de Gestão.}
\label{funcrelatgestao}
\end{figure}

Na figura \ref{funcrelatgestao} explica-se resumidamente o funcionamento dos Relatórios de Gestão.

Os relatórios de gestão não vão fazer a pesquisa por sessão, mas por horário.

Este facto poderá ser revelador de informações relacionada a períodos específicos.

A principal diferença destes relatórios é que não procuram por exigem que a sessão esteja encerrada para que 
possa consultar informações directamente do ponto de venda.


% \begin{bclogo}[couleur=blue!10,arrondi=0.1,logo=\bctakecare,ombre=true]
\bcatencao{\emph{''A operação não pode ser terminada''}}{ 
Em oposição ao relatório extraído em \keystroke{Relatórios Oficiais}, em \keystroke{Relatórios de Gestão} não se exige que a 
sessão tenha terminado para se extrair a informação.

Contudo é necessário ter o cuidado de colocar como período de análise a data de início de sessão. Se por algum motivo a sessão 
do dia não tiver sido aberta, deverá seleccionar a data do dia imediatamente anterior, pois estará a acumular as vendas nesse dia.
}

% \begin{bclogo}[couleur=blue!10,arrondi=0.1,logo=\bclampe,ombre=true]
\bcdica{Controlo de Ponto}{
Quando se faz o primeiro login, o sistema regista a hora em que este foi efectuado, 
pelo que quando que um funcionário dá o seu dia por terminado, deverá fazer login novamente, 
ir ao nível de \textsc{Utilizador} e pressionar \keystroke{Fechar Turno}.

Uma vez esta operação efectuada, ficará registada a hora de saída do funcionário.
}

\newpage
\section{SAF-T PT}

% O botão \keystroke{SAF-T PT} gera o ficheiro SAF-T(PT) (Standard Audit File for Tax Purposes – Portuguese version).
% Um ficheiro normalizado (em formato XML) com o objectivo de permitir uma exportação fácil, e em qualquer altura, de um conjunto predefinido de registos contabilísticos, de facturação, de documentos de transporte e recibos emitidos, num formato legível e comum, independentemente do programa utilizado, sem afectar a estrutura interna da base de dados do programa ou a sua funcionalidade.

Para extrair o ficheiro SAF-T PT, deverá inserir uma \emph{pen drive} num porto USB do equipamento.

Pouco segundos depois pressione o botão \keystroke{SAF-T PT}. 

\begin{figure}[h]
\begin{center}
\includegraphics[height=3cm]{../images/exportarSAF-T.png}
\caption[Submanifold]{Exportar SAF-T PT para a pen.}
\label{relatorioGestao}
\end{center}
\end{figure}

Irá aparecer uma caixa de diálogo com os selectores para ''Ano'' e ''Mês'' e um selector com o dispositivo de destino. 
O nome da \emph{pen} deverá aparecer nesse campo. 

Uma vez seleccionado o ano e o mês pretendidos, pressione \keystroke{Gerar SAF-T}.

Ao fim de alguns segundos, irá aparecer uma mensagem a informar que o ficheiro foi gerado e que pode retirar a \emph{pen}.

\subsection{SAF-T por email}

Em alternativa a extrair o ficheiro localmente, pode-se agendar o envio do ficheiro para um email à escolha.

O ficheiro é então gerado na data prevista, validado em acordo com o validador fornecido pela Autoridade Tributária e enviado para um ou mais emails definidos. 

\begin{figure}[h]
\begin{center}
\includegraphics[height=7cm]{../images/mailSAFT.png}
\caption[Submanifold]{exemplo de email enviado com SAF-T em anexo.}
\label{relatorioOficial}
\end{center}
\end{figure}

% \begin{bclogo}[couleur=blue!10,arrondi=0.1,logo=\bctakecare,ombre=true]
% \bcatencao{Requer ligação à Internet}{
Este serviço terá de ser configurado no próprio sistema operativo, e operador não terá acesso 
a estas parametrizações.
 
Esta ferramenta serve o propósito de facilitar o trabalho, evitar o extravio, e garantir a entrega do ficheiro em boas condições aos serviços de contabilidade.











\chapter{Impressoras e outros periféricos}

% \section{Máquinas e dispositivos}

% Todo o ponto de venda assenta na rede. 

\section{Máquinas}


\begin{figure}[h]
\begin{center}
\includegraphics[height=3cm]{../images/maquinasdispositivos.png}
\caption[Submanifold]{Botões Máquinas e Dispositivos.}
\label{Maquinas}
\end{center}
\end{figure}



No botão \keystroke{Máquinas} vai encontrar todos os equipamentos instalados no local, e a relação entre os seus IPs (endereço de rede) e 
o nome que foi convencionado dentro do sistema.

\begin{figure}[h]
\begin{center}
\includegraphics[height=6cm]{../images/Maquinas.png}
\caption[Submanifold]{Botões relacionados com a gestão de impressão.}
\label{Maquinas}
\end{center}
\end{figure}

Neste menu é possível parametrizar o perfil que cada máquina pode usar, sendo possível restringir o acesso de uma máquina a uma única zona. 

\section{Dispositivos}
\label{dispositivos}

No botão \keystroke{Máquinas} vai encontrar todos os periféricos instalados  --- em particular as impressoras e os seus nomes que vai permitir 
associar a ordens de trabalho dentro do sistema.

\begin{figure}[h]
\begin{center}
\includegraphics[height=6cm]{../images/dispositivosRede.png}
\caption[Submanifold]{Dispositivos ligados a cada máquina.}
\label{dispositivosRede}
\end{center}
\end{figure}



\section{Impressoras}

\begin{figure}[h]
\begin{center}
\includegraphics[height=3cm]{../images/impressoras.png}
\caption[Submanifold]{Botões relacionados com a gestão de impressão}
\label{impressoras}
\end{center}
\end{figure}


Considerando que esteja uma ou mais impressoras físicas configuradas para funcionar, é necessário associar estes
periféricos às tarefas que vão desempenhar.



% \section{Impressoras Virtuais}


\begin{figure}[h]
\begin{center}
\includegraphics[height=5cm]{../images/impressorasvirtuais.png}
\caption[Submanifold]{Botões relacionados com a gestão de impressão.}
\label{impressoras}
\end{center}
\end{figure}
É necessário criar as impressoras que o programa vai usar e atribuir as funções a cada uma, e isso faz-se pressionando o botão \keystroke{Impressoras Virtuais}.

Para este caso vamos criar as impressoras \textbf{Bar}, \textbf{Cozinha}, \textbf{Impressora Local} e \textbf{Gerente}.

Vamos assumir desde já que as impressoras  \textbf{Bar}, \textbf{Cozinha} servirão para receber os pedidos de bebidas e de refeições, respectivamente.

Na impressora \textbf{Impressora Local} sairão os talões de conta e as factura e impressora \textbf{Gerente} sairão os relatórios.




% Uma vez definidas as impressoras físicas, é preciso associar uma nome à impressora.

Basta pressionar no campo com o nome para que surja um teclado no ecrã onde poderá inserir um nome sugestivo para a impressora.

Pode-se adicionar ou remover impressoras com os botões \keystroke{Novo} e \keystroke{Apagar} conforme pode ser observado na figura \ref{impressoras}.

Ao fazer isto, está-se a criar uma camada de abstracção que é importante manter para simplificar a gestão do espaço.




\section{Ligação Impressoras}


\begin{figure}[h]
\begin{center}
\includegraphics[height=6cm]{../images/impressorasligacao.png}
\caption[Submanifold]{Associação das impressora virtuais às impressoras físicas.}
\label{impressorasligacao}
\end{center}
\end{figure}


Depois de criadas as impressoras, é necessário configurar a forma como as mesmas são vistas por cada posto.

Pressionando no botão \keystroke{Ligação Impressoras}, temos acesso a um painel onde se pode observar na parte de cima o posto que se pretende preparar, 
e na parte de baixo três colunas.

Na primeira coluna figura os nomes das impressoras disponíveis;

Na segunda coluna aparecem novamente os nomes dos postos apresentados acima;

Na terceira coluna aparecem as impressoras físicas associadas a cada posto seleccionado na coluna do meio (ver \ref{dispositivos}).


No exemplo da figura, pretende-se que o posto \textbf{POS\_1} imprima os talões na impressora física que está ligada no próprio.

Para isso deve-se seleccionar o \textbf{POS\_1} no quadro de cima, depois marcar em baixo a  \textbf{Local Printer}, 
voltar a seleccionar o  \textbf{POS\_1} (pois será neste que sairão as facturas) e a impressora configurada, neste caso a  \textbf{EPSON T88III}.

\vspace{5mm}



%\begin{bclogo}[couleur=blue!10,arrondi=0.1,logo=\bclampe,ombre=true]
\bcdica{Personalização de espaço}{
Esta funcionalidade permite configurações muito específicas.


% Considerando um espaço com vários postos, entre os quais um \textbf{Bar}, é possível parametrizar 
% que o \textbf{Bar} não imprima os pedidos gerados pelo próprio.


% Havendo dois postos, um dentro do bar e outro mais na zona do restaurante, é 
% possível definir que no posto de restaurante, os pedidos de bar serão emitidos na impressora 
% no posto de bar.
No posto \textbf{Bar}, dado que será a pessoa que regista a mesma a atender o pedido, pode-se
considerar % que não faz sentido estar a emitir uma ordem impressa. 
que o posto \textbf{Bar} não imprima os pedidos gerados pelo próprio.

Todos os outros postos poderão estar configurados para enviar os seus pedidos de bebidas para este Posto, sem prejuízo da configuração.
}


\section{Impressão de Documentos}

\begin{figure}[h]
\begin{center}
\includegraphics[height=5cm]{../images/impressorasdocumentos.png}
\caption[Submanifold]{Tipos de Documentos e onde devem ser impressos.}
\label{impressorasdocumentos}
\end{center}
\end{figure}


Quando se definiu as tarefas de cada impressora, cabe agora decidir onde vai ser impresso cada documento.

Para tal pressiona-se o botão \keystroke{Impressão Documentos}, e surge um quadro com duas colunas. 

Na coluna da esquerda estão listados todos os documentos passíveis de ser impressos, e na coluna da esquerda as impressoras disponíveis.

Bastará portanto associar cada documento à respectiva impressora.



\begin{verbatim}
TEXTO ANTIGO

\end{verbatim}



\begin{figure}
\includegraphics[width=15cm]{../images/diagramaMaquinas.jpg}
\caption[Submanifold]{Organização de hardware.}
\end{figure}

\section{Máquinas}

Aqui definem-se os pontos de venda  e respectivos IPs.

Neste ponto pode-se associar um perfil gráfico a uma máquina.

Pode-se ainda criar as parametrizações de X64 que funcionam usando PLUs dos artigos.

Pode-se por exemplo associar a regra de que a máquina "Restaurante" tem somente acesso às mesas da zona restaurante. 

itens relacionados: perfis, produtos.

\section{Dispositivos}

Aqui definem-se todos os periféricos e definem-se onde ficam ligados a cada equipamento.

Pode-se criar impressoras, \emph{displays} de cliente, balanças, gavetas de valores, leitores vários, etc

Este quadro é separado por quatro quadros relacionados: 
\begin{itemize}
\item máquinas
\item dispositivo
\item tipo de ligação
\item configuração da ligação.
\end{itemize}

As parametrizações de baixo nível, tais como configurações de porta série são feitas a nível de sistema operativo.

seleccionando um periférico, pode-se alterar o nome do mesmo, e caso seja um dispositivo de entrada --- como é o caso 
de um leitor biométrico --- é possível criar a expressão regular para validar o valor lido.

Caso haja um sistema de entrada de interface PS2 (entrada convencionalmente utilizada pelo teclado), 
a expressão regular deverá ser utilizada em \textbf{Definições Globais}, afectando todos os equipamentos da rede.

itens relacionados: Definições Globais.

\section{Impressoras Virtuais}

Com impressoras virtuais entende-se como o papel que uma impressora toma no desenrolar das suas tarefas num espaço comercial.

Pode ser cómodo designar uma impressora com um nome específico, criando um nível de abstracção do sistema informático
permitindo ao gerente fazer o redireccionamento de pedidos de uma forma expedita sem quaisquer conhecimentos técnicos.

itens relacionados: Ligação de Impressoras, Dispositivos, Impressão de Documentos.

\section{Ligação de Impressoras}

A partir daqui pode-se definir que impressoras imprimem consoante o local em que nos encontramos.

Pode-se definir que se estou num local de transformação, e se fizer um pedido de um produto transformado neste local,
não faz sentido que haja emissão de papel.

Por outro lado se estiver longe do local de transformação, deverá sair sempre um pedido na zona de transformação, 
identificando o local para onde se destina e o funcionário que fez a requisição.

itens relacionados: Impressoras Virtuais, Dispositivos, Impressão de Documentos.

\section{Impressão de Documentos}

Pode-se aqui definir em que impressora deverá ser impresso cada documento.

itens relacionados: Impressoras Virtuais, Ligação de Impressoras, Dispositivos.

\section{Upload de logo para Impressora}

Para exportar uma imagem para a impressora, é necessário satisfazer as seguintes condições:
\begin{itemize}
\item a imagem esteja em formato \emph{bmp};
\item a imagem  esteja a duas cores;
\item o ficheiro esteja no directório apropriado do software.
\end{itemize}















\part{BackOffice Online}

% \chapter{Introdução}              
 \chapter{Os níveis de utilização}

Como em qualquer indústria em que segmenta cada área de trabalho a cada trabalhador,
 este programa foi desenhado de forma a simplificar o mais possível o interface com
o utilizador


Desta forma, o programa adapta-se de acordo com a pessoa que utiliza o programa, facultando
a possibilidade de que a pessoa que introduz as compras tenha um perfil completamente distinto
de quem serve à mesa, que por sua vez terá um aspecto completamente diferente do supervisor
que no fim do dia pretende fazer as leituras do dia ou do mês.


O programa tem várias áreas onde acedendo a cada uma delas poderá fazer operações específicas. 
essas áreas denominam-se por \textbf{camadas} ou \textbf{níveis}.

% \includegraphics[height=3in]{../images/user.png}

Seguidamente serão apresentadas as opçõe disponíveis em cada um desses mesmos níveis.
                 


\begin{figure}
\includegraphics[width=15cm]{../images/koncepto.png}
\caption[Submanifold]{Representação gráfica do conceito de Níveis.}
\end{figure}

\begin{figure}
\includegraphics[width=15cm]{../images/konceptoTree.jpg}
\caption[Submanifold]{Árvore de configurações do programa}
\end{figure}




\chapter{Utilizador} 

O nível \textbf{utilizador} é o nível onde se inicia tudo, e é o estado nativo do programa.
De um dos lados do ecrã estarão visíveis as outras áreas do programa, que serão desbloqueadas 
consoante o perfil do funcionário que se identificar (por código numérico, leitor de cartões ou interface biométrico)

Neste nível poderão estar visíveis:
\begin{itemize}
\item o relógio
\item lista de utilizadores
\item o botão desligar máquina
\item o botão fechar turno
\item o botão bloquear níveis.
\end{itemize}
\begin{figure}
\includegraphics[height=3in]{../images/user.png}
\caption[Submanifold]{Exemplo de um interface de utilizador.}
\end{figure}

\section{Bloquear Níveis}

Este botão tem uma importância vital na gestão e permite ao utilizador sair da sua área de trabalho
impedindo assim que outros utilizadores possam consultar dados que normalmente não teriam acesso.

Este botão obriga a que o próximo funcionário que queira utilizar o computador tenha de se identificar.

Esta simples metodologia permite fazer leitura correctas nos relatórios de empregados (capítulo Gerente).

\section{Fechar Turno}

Este botão permite fazer o controlo de ponto de funcionário

Ver capítulo gerente, fundo maneio
Ver capítulo gerente, relatório de empregados


\section{Desligar Máquina}

Pressionando este botão, irá aparecer uma nova caixa de diálogo a perguntar se tem realmente a certeza
de que pretende desligar o equipamento.

Se confirmar, o computador irá dar início ao processo de encerramento, desligando o sistema progressivamente
até ao momento em este ficará totalmente inactivo.

\begin{boxedminipage}{\textwidth}
        Caso o sistema não possua suporte a gestão de energia,
	aguarde até aparecer a mensagem \textbf{System Halted} e então será seguro desligar o equipamento.
\end{boxedminipage}













%% \printindex



\end{document}
