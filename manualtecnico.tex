 \documentclass[a4paper,11pt,openany]{memoir}
%  \documentclass[a4paper,11pt,openany]{book}

%  \topmargin 0in
%  \textheight 7in
%  \textwidth 6in
%  \oddsidemargin 0in
%  \evensidemargin 0in
%  \headheight 1.2in % 7pt
%  \headsep 0.25in




\setlrmarginsandblock{2.5cm}{*}{1} 
\setulmarginsandblock{2.5cm}{2.5cm}{*}
\setmarginnotes{2.5mm}{2cm}{1em}
\checkandfixthelayout

\usepackage[tikz]{bclogo}

% \usepackage[colorlinks]{hyperref}
 \usepackage{hyperref}
\usepackage{memhfixc}

%%%%%%%%%%%%%%%%%%%%%%%%%%%%%%%%%%%%%%%%%%%%%%%%%%%%%%%%%%%%%%%%

\usepackage{color,calc,graphicx,soul,fourier}
% \definecolor{nicered}{rgb}{.647,.129,.149}
\definecolor{nicered}{rgb}{.0,.0,.255}
\makeatletter
\newlength\dlf@normtxtw
\setlength\dlf@normtxtw{\textwidth}
\def\myhelvetfont{\def\sfdefault{mdput}}
\newsavebox{\feline@chapter}
\newcommand\feline@chapter@marker[1][4cm]{%
\sbox\feline@chapter{%
\resizebox{!}{#1}{\fboxsep=1pt%
\colorbox{nicered}{\color{white}\bfseries\sffamily\thechapter}%
}}%
\rotatebox{90}{%
\resizebox{%
\heightof{\usebox{\feline@chapter}}+\depthof{\usebox{\feline@chapter}}}%
{!}{\scshape\so\@chapapp}}\quad%
\raisebox{\depthof{\usebox{\feline@chapter}}}{\usebox{\feline@chapter}}%
}
\newcommand\feline@chm[1][4cm]{%
\sbox\feline@chapter{\feline@chapter@marker[#1]}%
\makebox[0pt][l]{% aka \rlap
\makebox[1cm][r]{\usebox\feline@chapter}%
}}
\makechapterstyle{daleif1}{
\renewcommand\chapnamefont{\normalfont\Large\scshape\raggedleft\so}
\renewcommand\chaptitlefont{\normalfont\huge\bfseries\scshape\color{nicered}}
\renewcommand\chapternamenum{}
\renewcommand\printchaptername{}
\renewcommand\printchapternum{\null\hfill\feline@chm[2.5cm]\par}
\renewcommand\afterchapternum{\par\vskip\midchapskip}
\renewcommand\printchaptertitle[1]{\chaptitlefont\raggedleft ##1\par}
}
\makeatother
\chapterstyle{daleif1}



%%%%%%%%%%%%%%%%%%%%%%%%%%%%%%%%%%%%%%%%%%%%%%%%%%%%%%%%%%%%%%%%

\usepackage{
  calc,
  graphicx,
  url,
  fancyvrb,
  multicol,
  kvsetkeys
}


%%%%%%%%%%%%%%%%%%%%%%%%%%%%%%%%%%%%%%%%
% Linguagem e Acentuacao
%%%%%%%%%%%%%%%%%%%%%%%%%%%%%%%%%%%%%%%%
\usepackage[portuges]{babel}
% \usepackage[latin1]{inputenc}      

%%%%%%%%%%%%%%%%%%%%%%%%%%%%%%%%%%%%%%%%
% Tipo de Letra
%%%%%%%%%%%%%%%%%%%%%%%%%%%%%%%%%%%%%%%%
\usepackage{helvet} \renewcommand{\familydefault}{\sfdefault}
% \usepackage[scaled]{uarial}

%%%%%%%%%%%%%%%%%%%%%%%%%%%%%%%%%%%%%%%%
% Pacote de gráficos
%%%%%%%%%%%%%%%%%%%%%%%%%%%%%%%%%%%%%%%%
\usepackage{graphicx}

\usepackage{fancybox}

\usepackage[utf8]{inputenc}
\usepackage[T1]{fontenc}



%%%%%%%%%%%%%%%%%%%%%%%%%%%%%%%%%%%%%%%%
% Pacote de gráficos
%%%%%%%%%%%%%%%%%%%%%%%%%%%%%%%%%%%%%%%%
% \usepackage{graphicx}

\usepackage{boxedminipage}

 \usepackage{pdfpages}

% For (non-printing) notes  \PWnote{date}{text}
\newcommand{\PWnote}[2]{} 
\PWnote{2009/04/29}{Added fonttable to the used packages}
\PWnote{2009/08/19}{Made Part I a separate doc (memdesign.tex).}

% same
\newcommand{\LMnote}[2]{} 

\begin{document}

 \includepdf[pages=1]{capa_livro.pdf}
% \input{capa_livro}


\frontmatter
\pagestyle{empty}


% % half-title page
% \vspace*{\fill}
% \begin{adjustwidth}{1in}{1in}
% \begin{flushleft}
% \HUGE\sffamily The
% \end{flushleft}
% \begin{center}
% \HUGE\sffamily  Memoir
% \end{center}
% \begin{flushright}
% \HUGE\sffamily  Class
% \end{flushright}
%%\begin{center}
%%\sffamily (Draft Edition 7)
%%\end{center}
% \end{adjustwidth}
% \vspace*{\fill}
% \cleardoublepage

% title page
\vspace*{\fill}
\begin{center}
\HUGE\textsf{Koncepto}\par
\end{center}
\begin{center}
\LARGE\textsf{\~}\par
\end{center}
\begin{center}
\HUGE\textsf{FrontOffice}\par
\end{center}

\begin{center}
\Huge\textsf{Manual do Utilizador}\par
\end{center}
\begin{center}
\LARGE\textsf{Sindesi}\par
\bigskip
\normalsize\textsf{Mantido por Nuno Leitão}\par
\medskip
\normalsize\textsf{Correspondendo à versão 4.3.4}\par  
\end{center}
\vspace*{\fill}
\def\THP{T\kern-0.2em S\kern-0.4em P}%   OK for CMR
\def\THP{T\kern-0.15em S\kern-0.3em P}%   OK for Palatino
\newcommand*{\THPress}{The Sindesi Press}%
\begin{center}
\settowidth{\droptitle}{\textsf{\THPress}}%
\textrm{\normalsize \THP} \\
\textsf{\THPress} \\[0.2\baselineskip]
%% \includegraphics[width=\droptitle]{anvil2.mps}
\setlength{\droptitle}{0pt}%
\end{center}
\clearpage



\PWnote{2009/06/26}{Updated the copyright page for 9th impression}
% copyright page
\begingroup
\footnotesize
\setlength{\parindent}{0pt}
\setlength{\parskip}{\baselineskip}
%%\ttfamily
\textcopyright{} 2008 --- 2013 Nuno Leitão \\
% \textcopyright{} 2011 --- 2013 Lars Madsen \\
Todos os direitos reservados.
All rights reserved.

The Sindesi Press, Normandy Park, WA.

Printed in the World 

The paper used in this publication may meet the minimum requirements
of the American National Standard for Information 
Sciences --- Permanence of Paper for Printed Library Materials, 
ANSI Z39.48--1984.

\PWnote{2009/07/08}{Changed manual date to 8 July 2009}
\begin{center}
10 09 08 07 06 05 04 03 02 01\hspace{2em}19 18 17 16 15 14 13
\end{center}
\begin{center}
\begin{tabular}{ll}
First edition:                        & 3 June 2001 \\
Second impression, with corrections:    & 2 July 2001 \\
Second edition:                       & 14 July 2001 \\
Second impression, with corrections:    & 3 August 2001 \\
Third impression, with minor additions: & 31 August 2001 \\
Third edition:                        & 17 November 2001 \\
Fourth edition:                       & 16 March 2002 \\
Fifth edition:                        & 10 August 2002 \\
Sixth edition:                        & 31 January 2004 \\
%%Draft Seventh edition:                & 31 January 2008 \\
Seventh edition:                       & 10 May 2008 \\
Eighth impression, with very minor corrections: & 12 July 2008 \\
Ninth impression, with additions and corrections: & 8 July 2009 \\
Eighth edition:                        & August 2009 \\
\end{tabular}
\end{center}
% \ifMASTER
Manual last changed today% \svnyear/\svnmonth/\svnday
% \fi

\endgroup


%%%%%%%%%%%%%%%%%% CITAÇAO %%%%%%%%%%%%%%%%%%%%%%%%%%%%%%%%%
% \clearpage
% \vspace*{\fill}
% \begin{quote}
% \textbf{FrontOffice koncepto,} é um livro que levou algum tempo a escrever.
% 
% O objectivo foi torna-lo não num livro de leitura, mas mais um livro de consulta rápida,
% permitindo ao utilizador poder manusear o programa de forma rápida e intuitiva.
%   \hspace*{\fill} 
%       \textit{Nuno Leitão}.
% \end{quote}
%%%%%%%%%%%%%%%%%% FIM DE CITAÇAO %%%%%%%%%%%%%%%%%%%%%%%%%%%%%%%%%

% \vspace{2\baselineskip}
% 
% \begin{quote}
% \textbf{memoir,} \textit{n.} [Fr. \textit{m\'{e}moire,} masc., a memorandum,
%     memoir, fem., memory $<$ L. \textit{memoria,} \textsc{memory}]
%   \hspace{1ex} \textbf{1.} a biography or biographical notice, 
 %      usually written by a relative or personal friend of the subject 
 %  \hspace{1ex} \textbf{2.} [\textit{pl.}] an autobiography, 
%       usually a full or highly personal account
 %  \hspace{1ex} \textbf{3.} [\textit{pl.}] a report or record of 
%       important events based on the writer's personal observation, 
%       special knowledge, etc.
%   \hspace{1ex} \textbf{4.} a report or record of a scholarly 
%       investigation, scientific study, etc.
%   \hspace{1ex} \textbf{5.} [\textit{pl.}] the record of the proceedings
%       of a learned society \\[0.5\baselineskip]
%   \hspace*{\fill} \textit{Webster's New World Dictionary, Second College Edition}.
% \end{quote}

% \vspace{2\baselineskip}


% \begin{quote}
% \textbf{memoir,} \textit{n.} a fiction designed to flatter the subject 
%   and to impress the reader. \\[0.5\baselineskip]
% \hspace*{\fill} With apologies to Ambrose Bierce % and Reuben Thomas
% \end{quote}

\vspace*{\fill}

\cleardoublepage

% ToC, etc
%%%\pagenumbering{roman}
\pagestyle{headings}
%%%%\pagestyle{Ruled}

% \setupshorttoc
% \tableofcontents
% \clearpage
% \setupparasubsecs
% \setupmaintoc
% \tableofcontents
% \setlength{\unitlength}{1pt}
% \clearpage
% \listoffigures
% \clearpage
% \listoftables
% \clearpage
% \listofegresults



\tableofcontents

\pagenumbering{arabic}

% body
\mainmatter


% \section{FrontOffice Koncepto}     %   1.  FrontOffice Koncepto
\subsection{Licença}

Copyright 2004 - 2015 Sindesi - Sistemas e Soluções Informáticas, Lda.

Todos os direitos reservados.

A utilização do software Koncepto incluído neste pacote foi concebida segundo os termos
descritos no Acordo de Licença de utilização do Software.

O utilizador só poderá realizar as cópias do Software permitidas no Acordo de Licença de
utilização do Software.

O utilizador poderá carregar e executar o Software na memória RAM do computador e utilizá-lo
num computador monoposto ou numa estação de trabalho individual.

O utilizador só poderá utilizar o Software em uma rede de comunicações se a licença de uso o
permitir e descriminar por escrito o número de postos que a licença permite usar em simultâneo.
A duplicação, cópia, distribuição ou qualquer outro tipo de cedência deste manual constitui um
delito contra a propriedade intelectual, se não for autorizada por escrito pela Sindesi - 
Sistemas e Soluções Informáticas, Lda.

Importante: Este acordo contém as condições da Licença de utilização do Software contido no
pacote selado. A abertura do pacote selado significa, da parte do cliente, a aceitação na íntegra,
das condições do presente acordo.
                    %   1.1.      Licença 
\subsection{Acordo de Licença de utilização de Software}
\subsubsection{Licença de Utilização}
A propriedade do Software contido no pacote selado não é
transferida para o cliente. A Sindesi - Sistemas e Soluções Informáticas, Lda, concede ao cliente, uma licença de
utilização, não transferível a terceiros, para usar o Software em uma estação de trabalho. Se o
Software estiver a ser partilhado através de uma rede de comunicações o cliente deverá ter uma
licença que o autorize a partilhar o Software, descriminando o número de postos de trabalho que
estão autorizados a partilhar o Software em simultâneo.


\subsubsection{Restrições de Cópia}
O cliente poderá copiar o Software, objecto da licença, unicamente
para fins de segurança, podendo no máximo dispor de uma cópia de segurança.

\subsubsection{Cumprimento e uso não Autorizado}
O cliente deverá comunicar por escrito à Sindesi - Sistemas e Soluções Informáticas, Lda.
 se tiver conhecimento da utilização não autorizada do Software a si licenciado,
por parte de terceiros e fazer todos os possíveis ao seu alcance para que esta utilização cesse
de imediato.

                     %   1.2.      Acordo de Licença de utilização de Software 
%

\subsection{Para quem se destina este manual}

Este manual destina-se aos técnicos que façam manutenção do sistema Koncepto. 

A filosofia do sistema é bastante simples, o que permite alguma plasticidade na parte de configuração. 

A forma como este é construído permite muito facilmente fazer um sistema distribuido, quer isto dizer que haverá sempre 
a possibilidade --- dentro de certos limites --- de permitir um equipamento danificado poder faturar com apoio a outros equipamentos na mesma rede.

Tenta-se abrangir neste manual as temáticas:
\begin{itemize}
\item A arquitetura da aplicação Koncepto;
\item Os requisitos mínimos do Sistema Operativo, e como resolver os principais problemas de dependências;
\item Apresentação dos periféricos mais comuns do mercado e como interagir com eles;
\item Instalação de drivers específicos e afinação para conseguir um bom compromisso de estabilidade;
\item manutenção de sistema operativo de forma a garantir backups 
\end{itemize}

Servirá como referência base na iniciação e conhecimento do produto. Pretende-se
que ele sirva como um guia de referência rápida para o apoio na execução das
principais tarefas necessárias ao bom funcionamento do programa.



\subsection{Requisitos mínimos}

Para utilização desta aplicação são considerados os seguintes requisitos mínimos:
\begin{itemize}
\item Conhecimentos básicos de informática e do sistema operativo Linux.
\item Conhecimentos básicos da aplicação de frontoffice Koncepto e conceitos adjacentes.
\item Vontade de aprender.
\end{itemize}






\newpage
\part{Introdução}

\chapter{Antes de começar}

Os técnico que pretende adquirir conhecimentos na manutenção eficiente deste programa, 
deverão ter um conhecimento minimo de navegação no sistema operativo, manipulação de directórios e ficheiros.

Por diversas vezes será necessário proceder à alteração de um ficheiro usando como unico recurso a linha de comandos, 
pois a manutenção de um equipamento muitas vezes implica que algumas ferramentas gráficas não se encontrem disponíveis.

A formação no sentido de adquirir conhecimentos nestes campos, cai fora do âmbito deste manual, contudo recomenda-se fortemente 
a leitura sobre:
\begin{itemize}
\item conhecimento base sobre os sistemas de ficheiros em ambientes *NIX:
\begin{itemize}
\item a árvore de directórios;
\item o sistema de permissões;
\item conhecimento base sobre o \textbf{init}
\end{itemize}
\item o editor de texto \textbf{vi};
\end{itemize}

O programa funciona em camadas de abstração, mas conforme pode ser visto neste manual, todo o interface trabalha muito perto da camada de hardware.

Este compromisso leva a uma estabilidade e longevidade de material pois as falhas detetadas, são rapidamente identificadas e em muitos casos facilmente contornadas.

Este manual pretende que um técnico seja capaz de conseguir instalar e parametrizar um programa Koncepto para um cliente.

Este manual foi escrito considerando a distribuição \textbf{Debian GNU Linux}, contudo não é condição necessária para que possa instalar o programa.

Tendo conhecimentos mínimos de outra distribuição, espera-se que consiga ter os mesmos resultados noutras distribuições. 

A leitura deste manual não dispensa a leitura do manual de utilizador de koncepto.

% O programa Koncepto foi desenhado para ser intuitivo e de fácil utilização mediante o uso do toque do dedo 
% no ecrãn para aceder às suas funcionalidades. 

Neste capítulo será explicada a forma de como interagir com o programa nas diversas fases.

\newpage
\section{Interface Gráfico}

% Todo este programa tenta ser o mais simples possível, contudo
% concordamos que este é um compromisso que nem sempre é simples de conseguir.

% Por esta razão passamos a apresentar o inteface de utilizador de forma a tornar o 
% ambiente o mais familiar possível, permitindo aos utilizadores usarem o programa de 
% uma forma fluida e natural.



\subsection{Setas de Selecção}

\begin{figure}[h]
\begin{center}
\includegraphics[height=5cm]{../images/setasSelecao.png}
\caption[Submanifold]{Setas de Selecção.}
\end{center}
\end{figure}


As setas de selecção permitem seleccionar um elemento de um lista.

As setas são representadas por triângulos em cor cinza no fundo da janela onde se encontra a lista. 

Pressionando as zonas onde se encontram os triângulos, poder-se-á navegar para cima e para baixo nos elementos da lista.

\subsection{Botão Definir}

\begin{figure}[h]
\begin{center}
\includegraphics[height=3cm]{../images/botaodefinir.png}
\caption[Submanifold]{Botão Definir.}
\label{botaodefinir}
\end{center}
\end{figure}


É muito comum aparecer botões de \emph{toggle} ou de comutação. 

Para facilitar o acesso a estes botões, vai encontrar sempre associado
a estas listas de opções um botão \keystroke{Definir}.

Cada vez que pressionar o botão \keystroke{Definir} irá marcar a opção 
como activa ou inactiva alternadamente.

Caso não tenha a certeza da alteração que pretende fazer, terá sempre 
disponível um botão que permite sair da configuração que está a fazer.





\newpage
\subsection{Teclados}

O interface do utilizador é sempre o ecrã, por este motivo é necessário que 
haja uma forma que se possa interagir directamente com o programa. 
Por isso haverá sempre disponível um teclado para todas as interacções do operador.

Estas opções são simplificadas ao máximo de forma a evitar o recurso aos mesmos,
 pois pretende-se que a operabilidade 
seja o mais célere possível.

Existem dois tipos de teclado disponíveis, o alfanumérico e o numérico, que estarão disponíveis
de acordo com as necessidades.


\subsubsection{Teclado alfanumérico}

\begin{figure}[h]
\begin{center}
\includegraphics[height=5cm]{../images/teclado.png}
\caption[Submanifold]{Representação do teclado alfanumérico no ecrã.}
\label{fig:alfanumerico}
\end{center}
\end{figure}



O teclado alfanumérico tem uma abrangência maior, e é invocado em quase todos os casos em
que se prevê a necessidade dele. O teclado está representado na Figura \ref{fig:alfanumerico}

Para colocar maiúsculas poderá pressionar a tecla \keystroke{CAPS} à esquerda, ou simplesmente pressionar
uma das teclas \keystroke{SHIFT} disponíveis em cada lado do teclado.

Se por alguma razão se enganou e pretender voltar ao menu anterior sem modificar os dados, bastará pressionar
a tecla \keystroke{ESC} que se encontra no canto superior esquerdo.

Para validar os dados submetidos, pressione a tecla \keystroke{$\hookleftarrow$} (ou ''Enter'') que aqui é representada como
um caractere de mudança de linha no lado direito do teclado.


\subsubsection{Teclado numérico}

\begin{figure}[h]
\begin{center}
\includegraphics[height=5cm]{../images/numerico.png}
\caption[Submanifold]{Representação do teclado numérico no ecrã.}
\end{center}
\end{figure}


O teclado numérico é utilizado para submissão de palavras chave de utilizadores, códigos de barra,
indicação de quantidades, preços, taxas, etc, ...

Este teclado conta com os botões:


 \begin{table}[ht]
%  \caption{Hardware} 
 \centering
\small
\def\arraystretch{1.5}
 \begin{tabular}{c p{12cm}}  %   l c r r } % centered columns 
 \textbf{Botão} & \textbf{Significado}  \\ % Garantia & Descrição & Qtd & C. Unit & Subtotal \\ [0.5ex]
%  \multicolumn{6}{l}{\textbf{Servidor}} \\			
 \hline
\keystroke{Atrás} & permite apagar o último caractere pressionado. \\
\keystroke{Cancelar} &  voltará ao menu anterior sem modificar qualquer dados. \\
\keystroke{OK} & Validará os dados colocados. \\
%  \multicolumn{6}{l}{\textbf{Bastidor}} \\			
 \hline
 \end{tabular}
 \end{table}







Os números premidos deverão aparecer sempre no ecrã, salvo duas excepções: a introdução de uma 
palavra passe, e a senha de identificação de um cartão cliente.






\subsection{Botão Propriedades}

\begin{figure}[h]
\begin{center}
\includegraphics[height=5cm]{../images/propriedades2.png}
\caption[Submanifold]{Propriedades de Botões.}
\label{propriedadesbotoes}
\end{center}
\end{figure}


Em diversos pontos do programa, pode-se encontrar o botão \keystroke{Propriedades}.

Esta opção permite personalizar botões de forma a que ganhem realce no interface gráfico. As características que se podem modificar são:
 % a secção em causa permitido desde uma simples troca de cor, aplicação de textura ou aplicação de uma imagem.



 \begin{table}[ht]
%  \caption{Hardware} 
 \centering
\small
\def\arraystretch{1.5}
 \begin{tabular}{c p{12cm}}  %   l c r r } % centered columns 
 \textbf{Botão} & \textbf{Significado}  \\ % Garantia & Descrição & Qtd & C. Unit & Subtotal \\ [0.5ex]
%  \multicolumn{6}{l}{\textbf{Servidor}} \\			
 \hline
 	\keystroke{Cor de Fundo} &  Muda a cor de fundo do botão. \\
	\keystroke{Textura de fundo} &  Associa uma imagem como padrão de fundo do botão. \\ 
	\keystroke{Cor de Texto} &  Altera a cor do texto. \\ 
	\keystroke{Letra} & Altera o tipo de letra dos caracteres do botão. \\ 
	\keystroke{Imagem} & Associa uma imagem ao botão. \\ 
	\keystroke{Limpar Imagem} & Desassocia qualquer imagem que esteja associada a este botão. \\ 
	\keystroke{Limpar Propriedades} & este botão limpa todas as características fazendo com que o botão fique com os valores por omissão. \\
%  \multicolumn{6}{l}{\textbf{Bastidor}} \\			
 \hline
 \end{tabular}
 \end{table}





% \begin{bclogo}[couleur=blue!10,arrondi=0.1,logo=\bctakecare,ombre=true]
\bcatencao{Tipos de letra e imagens}{
Quando fizer alterações, tenha presente que:
\begin{itemize}
\item os tipos de letra serão somente as disponibilizadas pelo sistema operativo, podendo em muitos casos não se encontrarem muitos tipos de letra instalados.
\item as imagens disponibilizadas são as imagens padrão incluídas no software, podendo ser aumentado mediante solicitação. 
\end{itemize}
}




% \subsection{Caixas}

% \subsection{Barras de Selecção}




\newpage
\section{Conceitos}

A fim de melhorar a experiência do utilizador e permitir melhores análises, foram criados alguns elementos 
que serão utilizados neste programa.

\subsection{Sessão}

A sessão diz respeito ao horário em que se pretende fazer o controlo de caixa.

A partir desta, é feito o controlo de empregado e de vendas de artigos. 

Todos os relatórios são elaborados a partir das Sessões decorridas, pois são estas 
que determinam a abertura e fecho de um estabelecimento e contagem de caixa.

Vejamos o seguinte exemplo:


\begin{bclogo}[couleur=blue!10,arrondi=0.1,logo=\bccrayon,ombre=true]{Configuração de Sessões}
Para um melhor entendimento do que consiste uma sessão, tomemos o exemplo de um estabelecimento 
que abra às 8h e encerre às 2h do dia seguinte.

Consideremos ainda que este estabelecimento 
trabalha em dois turnos, um das 8h às 17h e outro das 17h às 2h do dia seguinte.

Para agilizar o processo de controlo de caixa faz-se duas sessões, uma referente ao período da manhã e outro 
referente ao período da tarde. 

A partir daqui é possível os funcionários do turno da manhã darem o trabalho por terminado encerrando o caixa. 

O turno da tarde poderá iniciar o trabalho sem que este seja afectado de alguma forma pelo serviço da manhã.
\end{bclogo}

\vspace{5mm}
\begin{bclogo}[couleur=blue!10,arrondi=0.1,logo=\bclampe,ombre=true]{Configuração de Sessões}
As sessões podem ser configuradas para funcionar de uma forma manual, permitindo ao gerente ter algum controlo
sobre o fecho de sessão, ou automática, fazendo com que as contas transitem entre sessões 
\end{bclogo}


\begin{bclogo}[couleur=blue!10,arrondi=0.1,logo=\bctakecare,ombre=true]{Mapa de IVA por Controlo de Caixa}
Há uma diferença entre o Mapa de IVA obtido mediante o controlo de caixa (que diz respeito aos intervalos das
sessões) e o Mapa de IVA diário, que diz respeito aos períodos de 24 horas. Obviamente o controlo de caixa privilegia
o controlo do negócio. Ambos os mapas podem ser extraídos do software, mas é importante perceber a diferença entre eles.
\end{bclogo}


\subsection{Família}

Por uma família, entende-se por agrupamento de produtos similares. 

Em vez de atribuir determinadas características por produto pode-se atribuir uma regra para associar a todos os produtos de um mesmo tipo, facilitando 
assim a interação com o programa.

Para além disso, criando uma família, é possível ter um entendimento sobre a venda de determinados tipos de produtos e como estes afectam o volume de vendas. 



\begin{bclogo}[couleur=blue!10,arrondi=0.1,logo=\bclampe,ombre=true]{Análise por Família}
Por exemplo, agrupando todos os artigos de um determinado fornecedor numa única família, 
permite fazer uma leitura imediata do volume de vendas de um determinado produto, 
assim como o impacto percentual na faturação aquando o fecho de caixa.
\end{bclogo}


\subsection{Perfil de Utilizador}

Consoante a função de cada funcionário, pode-se especificar a cada momento os privilégios de cada um,
retirando as responsabilidade da parte humana e delegando ao software a tarefa de gestão de contas.

\begin{bclogo}[couleur=blue!10,arrondi=0.1,logo=\bccrayon,ombre=true]{Funções por utilizador}
Por exemplo impedindo vários funcionário de faturar (fazendo apenas pedidos) e 
delegando a apenas a um gerente a tarefa de fazer o serviços e fechar contas terá
sempre a certeza de que tem controlo absoluto do seu negócio.

Note-se que é impossível cobrar ou fazer chegar um pedido ao cliente sem que este tenha passado pelo sistema.
\end{bclogo}


% \subsection{Perfil de Máquina}
% Em situações 





% \section{Nomenclatura utilizada neste manual}


\chapter{Guia de Iniciação Rápida}  %  

Neste capítulo será descrito resumidamente os passos necessários para a utilização imediata do sistema no dia de trabalho

Note que este guia poderá ser ajustado em função dos equipamentos ou metodologias implementadas em cada negócio.

\begin{enumerate}
	\item Ligar equipamentos;
	\item Certifique-se que tem uma sessão a decorrer;
	\item Proceda à faturação;
	\item Feche a sessão;
	\item Desligue computador.
\end{enumerate}


\section{Ligar equipamentos}

Deverá ligar todo o equipamento, começando pelo hardware periférico e de seguida
o ponto de venda.

Verifique que os cabos estão bem ligados e caso haja uma UPS no local, que esta esteja 
bem ligada e em funcionamento normal.

\subsection{Como ligar a impressora e outros periféricos?}
Esta operação depende da marca e do modelo disponível. De modo geral, deverá
pressionar um botão (on/off), e obter como resposta o acendimento de um led
indicador.

Para mais indicações deverá recorrer ao manual disponibilizado pelo fabricante do
equipamento.

\begin{bclogo}[couleur=blue!10,arrondi=0.1,logo=\bclampe,ombre=true]{Pontos de venda integrados}
Alguns equipamentos estão alimentados directamente ao equipamento ponto de venda, 
pelo que só deverão responder quando tudo estiver ligado.
\end{bclogo}

\begin{bclogo}[couleur=blue!10,arrondi=0.1,logo=\bctakecare,ombre=true]{Cuidado com as UPS}
Se possível evite ligar impressoras ou outros aparelhos de elevado consumo nas UPS,
pois não só aumentam bastante o desgaste das mesmas, como reduzem significativamente o tempo de autonomia
em caso de quebra de energia.
\end{bclogo}



\section{Como aceder à aplicação?}

Ao iniciar o ponto de venda, deverá aguardar o arranque do sistema operativo, após
o qual a aplicação será iniciada automaticamente. É apresentada a interface
principal com a listagem de utilizadores pertencentes à sessão actual. Apenas estes
têm acesso à aplicação.


\begin{figure}
\begin{center}
\includegraphics[height=7cm]{../images/user2.png}
\caption[Submanifold]{Imagem de início.}
\label{welcomeScreen}
\end{center}
\end{figure}


Neste momento deverá aparecer uma imagem semelhante à apresentada na Figura \ref{welcomeScreen}

Note que neste momento todo programa está com acesso restringido, obrigando a que 
o utilizador se identifique para que possa executar qualquer operação
	
Para aceder deverá pressionar o botão com o utilizador respectivo, e indicar a chave
de acesso, caso lhe seja requisitada.


Caso já tenha uma sessão iniciada, logo que se autentique poderá começar a faturar.



\section{Como determinar se existe alguma sessão activa?}



Todas as transacções se operam dentro de sessões válidas.

Quer isto dizer que fora do período de uma sessão, será impossível proceder a qualquer operação.


O nome da sessão deverá aparecer no ecrãn principal (ver figura \ref{userlogged}).

Na dúvida, acedendo ao nível \textsc{Gerente}, escolhe-se o botão \textsc{Gestor de Sessão} e verifica-se qual
a sessão principal a decorrer.

\begin{figure}
\begin{center}
\includegraphics[height=7cm]{../images/user.png}
\caption[Submanifold]{"Sessão Diaria" em funcionamento}
\label{userlogged}
\end{center}
\end{figure}



\section{Como iniciar uma sessão?}
Caso a sessão seja definida como manual, será necessário iniciar a sessão (ver figura \ref{gestorSessao}). 

Para iniciar uma sessão basta aceder ao nível de \textsc{Gerente}, no botão \textsc{Gestor de Sessão} 
e carregar no botão \textsc{Iniciar Sessão}.


Quando a sessão inicia, deverá sair na impressora um papel com a informações:
\begin{itemize}
	\item nome da sessão,
	\item hora de início da sessão, 
	\item o nome de todos os funcionários que foram designados para trabalhar na sessão escolhida.
\end{itemize}

\begin{bclogo}[couleur=red!30,arrondi=0.1,logo=\bctakecare,ombre=true]{\emph{''Falhou a ligação ao printserver''}} 
Caso a impressora esteja desligada quando se tentar imprimir algo, irá surgir a mensagem \emph{''Falhou a ligação ao printserver''}.

Por defeito o programa espera que esta falha esteja associada a hardware, e não volta a tentar enviar mais documentos para essa impressora 
para não atrasar o sistema deixando documentos no spool.

Se tem a certeza de que o problema não é da impressora, reinicie o sistema e certifique-se de que a impressora se encontra bem ligada antes 
de imprimir um documento.

\vspace{5mm}
NOTA: pode-se contornar esta característica alterando as definições da impressora para serem unidireccionais.
O erro deixará de aparecer, mas não é situação recomendável
\end{bclogo}



\begin{figure}
\begin{center}
\includegraphics[height=7cm]{../images/gestorSessao.png}
\caption[Submanifold]{Gestor de Sessão}
\label{gestorSessao}
\end{center}
\end{figure}




\section{Proceder a uma venda}

Neste momento já é possível abrir uma mesa. Selecione o Nível \textsc{Zonas} e escolha uma mesa em que pretende trabalhar (Figura \ref{nivelMesa}).

Na parte de cima terá os \textsc{Índices} e por baixo destes estarão as \textsc{Páginas}.


\begin{figure}
\begin{center}
\includegraphics[height=7cm]{../images/mesa.png}
\caption[Submanifold]{Nível Mesa}
\label{nivelMesa}
\end{center}
\end{figure}



Para colocar um artigo, bastará navegar nas diversas páginas e pressionar o botão correspondente ao artigo prentendido.

\subsection{Como cancelar uma artigo}

Consoante o nível de privilégios que tiver, poderá ser possível ou não proceder ao cancelamento de um artigo. Caso não tenha privilégios, por favor contacte o gerente de loja.

Caso o artigo já tenha sido pedido (as letras ficarão com a cor destacada) e ao pressionar a tecla \textsc{Cancelar}.

%% irá aparecer a caixa de diálogo de desperdício. Esta caixa diz respeito à movimentação de stock.


Poderá consultar mais informação sobre este assunto na página \pageref{VendaArtigos}.


\section{Facturação}


Uma vez colocado todos os artigos, deverá pressionar o botão \textsc{Fatura}. 

Imediatamente irá aparecer um painel a pedir o número contribuinte do cliente. 

Se preencher o contribuinte, pressione em \textsc{Pesquisar} e se o cliente existir na base de dados, os dados serão preenchidos automaticamente,
 e poderá proceder à faturação, caso contrário terá de preencher os restantes campos.

Se no painel pressionar \textsc{Cancelar}, então seguirá imediatamente para o painel de faturação.

No painel de faturação aparece do lado esquerdo o valor a pagar, e o método de pagamento usado.

Do lado direito aparece o valor entregue. Quando se pressiona \textsc{OK}, surge uma caixa de diálogo com o valor do troco a entregar ao cliente (figura \ref{fatura}).


\begin{figure}
\begin{center}
\includegraphics[height=7cm]{../images/VD_troco.png}
\caption[Submanifold]{Faturação}
\label{fatura}
\end{center}
\end{figure}





\section{Como proceder a uma nota de crédito}

Caso tenha registado uma venda indevidamente, a única forma de proceder à anulação, é emitindo um documento de Nota de Crédito.

Para tal terá de ir até ao nível \textsc{Gerente} e em \textsc{História}, procurar o documento que gerou e proceder à anulação pressionando o botão \textsc{Nota de Crédito} (figura \ref{Historia}.


\begin{figure}
\begin{center}
\includegraphics[height=7cm]{../images/historia.png}
\caption[Submanifold]{Painel História}
\label{Historia}
\end{center}
\end{figure}


Quando se pressiona em  \textsc{Nota de Crédito}, surge uma nova caixa de diálogo que pergunta se se pretende refazer a fatura ou não.

Independentemente da resposta, será impressa uma Nota de Crédito com os dados vigentes na fatura em causa. 

Caso se tenha optado por refazer a fatura, o sistema irá criar uma mesa temporária onde colocará todos os artigos que foram anulados para que se possa fazer a correcção e imprimir a fatura correta.


\begin{bclogo}[couleur=blue!10,arrondi=0.1,logo=\bclampe,ombre=true]{Configuração de Sessões}
Todas as notas de crédito emitidas, são efetuadas sobre vendas efetuadas.

O sistema não permite gerar crédito a favor de outrem sem que tenha havido anteriormente um documento de venda.
\end{bclogo}


\subsection{Como encerrar uma sessão?}


Para encerrar uma sessão basta aceder ao nível de \textsc{Gerente}, no botão \textsc{Gestor de Sessão} 
 seleccionar a sessão a fechar e clicar no botão \textsc{Parar Sessão} e fechar a
respectiva janela.

\begin{bclogo}[couleur=blue!10,arrondi=0.1,logo=\bclampe,ombre=true]{Controlo de Ponto}
Quando se faz o primeiro login, o sistema regista a hora em que este foi efetuado, 
pelo que quando que um funcionário dá o seu dia por terminado, deverá fazer login novamente, 
ir ao nível de \textsc{Utilizador} e pressionar \textsc{Fechar Turno}.

Uma vez esta operação efetuada, ficará registada a hora de saída do funcionário.
\end{bclogo}



\section{Relatórios}

Para extrair os relatórios referentes ao sessão, é necessário ir ao nível de 
\textsc{Gerente}, no botão \textsc{Relatórios Oficiais} e selecionar o período
que se pretende extrair o relatório.

Os relatórios são extraídos por sessão, e por isso não é possível extrair relatórios referentes a sessões 
que estejam ainda a decorrer.

 
\begin{bclogo}[couleur=blue!10,arrondi=0.1,logo=\bctakecare,ombre=true]{\emph{''A operação não pode ser terminada''}} 
A data de um relatório \textbf{diz sempre respeito ao início de uma sessão}. Se por por acaso surgir o erro \emph{''A operação não pode ser terminada''} 
deve-se ao caso de não se ter aberto a sessão nesse dia, e as vendas acumularão com o dia(s) anterior(es)
\end{bclogo}

Poderá consultar mais informação sobre este assunto na página \pageref{reports}.

\section{Como desligar a máquina?}
O POS deverá ser desligado através do software. No nível de \textsc{Utilizador} encontrará
um botão \textsc{Desligar Máquina}, que ao pressionar fará aparecer uma mensagem a
confirmar a operação. 

Confirmando, a máquina será desligada.

\begin{bclogo}[couleur=blue!10,arrondi=0.1,logo=\bclampe,ombre=true]{Acesso a desligar a máquina}
Em certas configurações esta operação só pode ser efectuada após a validação de
determinado tipo de utilizadores – geralmente um técnico, supervisor ou gerente.
Em alguns modelos de equipamento, é necessário pressionar um botão físico
(on/off) no hardware.
\end{bclogo}

Pressionando este botão, irá aparecer uma nova caixa de diálogo a perguntar se tem realmente a certeza
de que pretende desligar o equipamento.

Se confirmar, o computador irá dar iníco ao processo de encerramento, desligando o sistema progressivamente
até ao momento em este ficará totalmente inactivo.

\begin{bclogo}[couleur=red!30,arrondi=0.1,logo=\bcbombe,ombre=true]{Desligar o Equipamento}
        Caso o sistema não possua suporte a gestão de energia,
	aguarde até aparecer a mensagem \textbf{System Halted} e então será seguro desligar o equipamento.
	
	Não desligue o equipamento de forma abrupta, pois poderá danificar o dispositivo de massa, 
	podendo levar à perda irreversivel de dados.
\end{bclogo}






Para mais informação consulte o manual do fabricante.




% \newpage

% \section{Como proceder a uma venda}


\subsection{Conceito de Sessão}

Todas as transacções se operam dentro de sessões válidas.

Quer isto dizer que fora do período de uma sessão, será impossível proceder a qualquer operação.

\subsection{Como iniciar uma Sessão}

Para se iniciar a sessão para começar a fazer vendas, é necessário ir até \textbf{Gerente -> Gestor de Sessão}

\includegraphics[height=3in]{../images/gestorSessao.png}

A partir daqui, deve-se escolher a sessão que se pretende iniciar da lista da esquerda de pressionar o botão \textbf{Iniciar Sessão}.

Quando a sessão inicia, deverá sair na impressora um papel com a informações:
\begin{itemize}
	\item nome da sessão,
	\item hora de início da sessão, 
	\item o nome de todos os funcionários que foram designados para trabalhar na sessão escolhida.
\end{itemize}

\subsection{Proceder a uma venda}

Neste momento já é possível abrir uma mesa. Selecione o tabulador \textbf{Zonas} e escolha uma mesa em que pretende trabalhar.

Na parte de cima terá os \textbf{Índices} e por baixo destes estarão as \textbf{Páginas}.

\includegraphics[height=3in]{../images/mesa.png}

Para colocar um artigo, bastará navegar nas diversas páginas e pressionar o botão correspondente ao artigo prentendido.

\subsection{Como cancelar uma artigo}

Consoante o nível de privilégios que tiver, poderá ser possível ou não proceder ao cancelamento de um artigo. Caso não tenha privilégios, por favor contacte o gerente de loja.

Caso o artigo já tenha sido pedido (as letras ficarão com a cor negra) e ao pressionar a tecla cancelar, irá aparecer a caixa de diálogo de desperdício. Esta caixa diz respeito à movimentação de stock.

\subsection{Como proceder a uma nota de crédito}

Caso tenha registado a venda, a única forma de proceder à anulação, é emitindo um documento de Nota de Crédito.
Para tal terá de ir até ao nível \textbf{Gerente -> História}, procurar o documento que gerou e proceder à emissão de uma Nota de Crédito.


\subsection{Como Fechar a sessão}

\subsection{Como extrair os relatórios}

Ao final da sessão de trabalho, pode-se finalmente proceder à extracção dos relatórios.

Para tal deve-se ir até ao nível gerente, e pressionar o botão \textbf{Relatórios Oficiais}.

Escolha a data e a Sessão correcta e pressione imprimir. 

% \newpage

% \section{Perguntas Frequentes}


\subsection{Quando tento tirar um relatório aparece a mensagem A operação não pode ser terminada}


\subsection{Quando tento tirar um relatório, o sistema demora muito tempo a responder e aparece a mensagem Falhou a ligação ao pserver}

\part{Procedimentos}

% \begin{bcbombe}{Mon Titre}
% Du texte qui se répète encore et encore pour l ’ exemple , du texte qui
% se répète encore et encore pour l ’ exemple , du texte qui se répète
% encore et encore pour l ’ exemple \ dots
% \end{bcbombe}





\section{Como ver um documento já emitido?}
Através do botão História que se encontra no nível de Gerente.
Este botão permite o
acesso a documentos passados que surgem da utilização da aplicação. Podem ser
vistos os detalhes dos documentos, imprimir duplicados,
criar notas de crédito e anular consumos.

% Na aplicação Koncepto, ao realizar uma venda é gerado um documento do tipo
% modelo de Fatura.
% 
% O Koncepto não emite facturas. Caso a implementação realizada
% o permita, poderão ser realizados consumos a crédito, sendo a facturação realizada
% à posteriori numa aplicação de backoffice, tal como o \textbf{Regi X12}.

\section{Gestão de Mesas}

\label{VendaArtigos}
Na parte das \textbf{Zonas}, pode-se fazer o acompanhamento do estado das mesas e assim
permitir verificar em tempo real o plano geral dos consumos do estabelecimento.

Na parte de gestão das mesas, pode-se considerar úteis duas operações:
\begin{itemize}
\item \textbf{Juntar mesas} --- situação em por exemplo vários elementos do mesmo agregado familiar
se distribuem por mesas distintas e então se pretende uma factura única com todos os items
consumidos, ou haja a solicitação do cliente para fazer a junção física de duas mesas ou mais mesas;
\item \textbf{transferir mesa} --- situação em que por alguma razão uma ou mais pessoas de uma mesa
se mudam para outro local ou zona do espaço comercial. 
\end{itemize}

\subsection{Como juntar mesas?}

Pressione o botão \textbf{Juntar} que se encontra no nível de Zonas.

Clique numa das mesas que prentende juntar.

clique na segunda mesa que prentende juntar.

Agora as duas mesas têm o mesmo nome e o somatório dos valores das duas mesas.

\section{Como seleccionar um produto para consumo?}

No nível das Zonas pressionar o botão com a mesa pretendida. A aplicação muda
para o nível das Mesas, mostrando os produtos organizados por índices e páginas.
Neste nível, deverá escolher os produtos pretendidos, pressionando o botão com o
seu nome. Estes serão apresentados numa listagem com o conteúdo de todos os
produtos a fornecer.


\subsection{Como pedir?}
O botão Pedir gera a ordem de entrega de todos os produtos seleccionados para
consumo, ainda por pedir, remetendo o pedido para as impressoras pré-definidas
para tal (caso existam e estejam definidas). A funcionalidade “Pedir” pode ser
configurada de modo a que ocorra automaticamente ao fazer o registo.

\subsection{Como cancelar um produto já seleccionado?}
Antes de registar, deverá seleccionar o produto a cancelar, navegando na lista de
produtos adicionados para consumo. Depois de seleccionado, deverá pressionar o
botão Cancelar, sendo o mesmo removido da listagem.


\subsection{Como dividir mesas?}
Através do botão Dividir mesas que se encontra no nível Mesa é possível realizar a
divisão de uma mesa em várias sub-mesas. A divisão (artigos/preços) pode ser
realizada de forma manual ou automática.

É possível, depois, controlar o que se pretende em cada uma das divisões, podendo mesmo acrescentar ou cancelar
produtos.


\subsection{Como registar?}
No caso de uma venda a dinheiro o registo é realizado através do botão Registo ou
Factura. A mesa é fechada e é apresentada uma janela onde é permitido escolher o
método de pagamento, fazer o troco e anexar um cliente ao pagamento.

\subsection{Como abrir a gaveta?}
Através do botão abrir Gaveta no nível de Mesas, quando disponível.


\subsection{Como fazer uma nota de crédito?}
Através do botão História que se encontra no nível de Gerente. Deverá localizar o
documento a creditar, ver o documento de modo a garantir que seleccionou o
documento correcto e então emitir a nota de crédito, pressionando o botão com o
mesmo nome.

\subsection{Como reimprimir uma factura?}
Esta função pode ser efectuada na janela aberta pelo botão História, no nível de
Gerente. Aí, deverá localizar o documento que lhe interessa e depois Imprimir.




\chapter{Consumos a crédito}

O Koncepto poderá ser configurado para que realize sempre vendas a dinheiro,
sempre consumos a crédito, ou funciona de uma forma composta. Avançamos com
indicações base para esta última situação, a mais complexa, que integra as
anteriores.

\section{Como faço um consumo a crédito?}
Um consumo a crédito pode ser efectuado através do botão Cartão Cliente onde
surge uma janela com três opções: Venda Normal onde volta à venda normal,
Código de Cliente onde se pode colocar o número de cartão manualmente, e por
último Cancelar.

Através do botão Cartão de Cliente (se activo), no nível Mesa, abrirá uma janela que
permite fazer uma venda a determinado cliente. Isto possibilita atribuir pontos,
descontos ou consumos a clientes específicos.

\section{Como faço o cancelamento de um consumo a crédito?}
Através do botão História no nível Gerente. Para cancelar o registo de um consumo
a crédito deve identificar e seleccionar o consumo na listagem apresentada e
pressionar no botão Nota de crédito.

\chapter{Reservas}

O Koncepto poderá ser configurado de forma a emitir reservas. Estas podem ser
apenas indicativas ou debitadas, e neste último caso, através da adição automática
de um produto ao documento de consumo. Estas parametrizações deverão ser
definidas na configuração inicial, aquando da instalação do sistema,

\section{Como criar Índices para as reservas?}
Para criar Índices para as Reservas deverá primeiro ir ao nível de Base de Dados e
no botão Índices criar um Índice principal ex: \textbf{Índice Reservas}, De seguida deverá
criar um sub – índice ex: \textbf{Menus Reserva} para isso não deve checar a opção \textbf{Índice Principal}. 

Dentro deste sub–índice deverá colocar a página com os produtos
pretendidos para o nível das Reservas. Para isso clicar no botão \textbf{Estrutura} e na
janela onde diz páginas colocar a página pretendida seleccionado-a e colocando-a
através da seta para a direita. 
Depois no Índice principal clicar no botão \textbf{Estrutura}.
Na parte da janela onde diz Índices colocar o sub- índice \textbf{Menus Reserva} através
da seta indicada para a direita. Fazer \textbf{OK} para guardar as alterações.

Para que depois este apareça no Layer das Reservas deverá ir ao nível de Base de
Dados, botão \textbf{Perfis}, e no separador \textbf{Cantinas,} na opção Índice para as Reservas
deverá indicar então o Índice principal criado anteriormente \textbf{Índice Reservas}
Não esquecer de o indicar para todos os perfis. No fim fazer \textbf{OK}.


\section{Como criar uma reserva?}
Quando se pretende criar uma reserva debitada no acto de emissão é necessário,
durante a venda ou realização de um consumo a crédito, carregar no botão Reserva
para aceder ao nível das Reservas onde poderá identificar o produto pretendido e a
data do seu consumo. 

Após completar esta operação, deverá gerar a reserva
pressionando o botão Terminar – por vezes também identificado como Imprimir.

Automaticamente voltará ao nível Mesa para terminar o registo.

Quando pretende criar uma reserva que será debitada apenas no acto do consumo,
esta é realizada de forma independente do registo do consumo. A qualquer
momento poderá aceder ao nível Reservas, identificar o cliente através da
passagem do cartão, ou introdução do PIN, carregando no botão PIN ou Cliente,
identificar os dados da reserva e terminar ou imprimir.
Em certas parametrizações, após um registo a aplicação pode abrir
automaticamente o nível das reservas.

\section{Como cancelar uma reserva?}
Caso seja uma reserva pré paga será necessário criar uma nota de crédito, que
anulará tanto a venda ou consumo como a reserva propriamente dita Em termos
operacionais, a senha se existir deve ser recolhida pelo operador. Se refizer o
documento a todas as linhas do documento original são importadas para o novo,
incluindo a \textbf{Reserva}, mas a senha foi cancelada.

Deverá então cancelar a reserva
e adicionar de novo, fazendo o registo e forçando a criação da nova marcação e
emissão de senha. No caso de uma reserva indicativa basta aceder ao nível
Reserva, identificar o cliente através da passagem do cartão de cliente no MSR ou
carregando no botão Código cliente e digitar o respectivo PIN e, de seguida
identificar a data e sessão da reserva em causa, obtendo como resultado a
visualização produto anteriormente reservado. Para o cancelar deverá selecciona-lo,
e pressionar o botão Cancelar. 

Para finalizar, deverá carregar no botão Terminar.

\section{Como definir a data sugerida para uma reserva?}
Aceder ao nível Gerente seleccionar o botão Definições Globais e no separador
Cantinas pode definir se a data sugerida para a reserva é dinamicamente alterada
para o próximo dia de trabalho ou se é fixa. Neste caso é necessário identificar a
data pretendida.

\section{Como verificar o estado das reservas?}
Através do botão Estado das Reservas que se encontra no nível de Gerente. Aqui
pode observar os produtos reservados que já foram consumidos e/ou aqueles por
consumir. Através do botão Relatórios de Gestão, pode imprimir um Relatório de
Reservas.

\section{Como bloquear a criação de reservas?}
Através do botão Bloquear Reservas no nível Gerente. Este bloqueia a realização de
reservas para a sessão escolhida, a partir desse momento. Desta forma, no futuro
não será possível efectuar mais reservas para essa mesma sessão.

\section{Como fazer um controlo das reservas não consumidas?}
No layer de Base de Dados no botão Definições Globais, no separador Cantinas
deve activar a opção \textbf{Converte reservas não consumidas em consumos de cantina},
Esta opção vai agarrar nas reservas não consumidas, após o fecho de sessão, de
clientes com consumo a crédito e marca-as como consumidas e faz uma venda a
dinheiro para a reserva. Esta opção funciona apenas com clientes com consumo a
crédito.

\chapter{Relatórios}

\label{reports}
O Koncepto poderá ser configurado de forma a permitir a emissão de relatórios para
certos utilizadores. Por defeito, todos os relatórios estão disponíveis para
utilizadores do tipo gerente, ou superior. Existem dois tipos de relatórios, disponíveis
através de objectos diferentes: Relatórios Oficiais e Relatórios Gestão.

Todos os relatórios são apresentados num ambiente gráfico comum, o qual se passa a
descrever:

À esquerda, apresenta-se o selector do Tipo de Relatório. Deverá ser seleccionado
o relatório a apresentar.

À direita, o período de tempo no qual tiveram início as sessões a incluir no relatório,
e a listagem do tipo de sessões a incluir. Deverá ser indicado o intervalo de tempo a
considerar, e os tipos de sessões a incluir.

No fundo, um botão para imprimir o relatório com as características identificadas, e
outro para sair.

\section{Como emitir um relatório de gestão?}
Aceder ao nível de Gerente, e pressionar o botão Relatórios Gestão, obtendo uma
janela na qual deverá identificar o Período de Tempo e o Tipo de Sessões que teve
início nesse período, que pretende incluir no relatório. Após estas opções, deverá
seleccionar o relatório pretendido e carregar em Imprimir.

Estão disponíveis os seguintes relatórios:
\begin{itemize}
\item Relatório de Empregados
Discrimina todas as acções dos utilizadores por sessão, os
cancelamentos, as anulações com e sem desperdício, os descontos as notas
de crédito e vendas a dinheiro. Ainda indica o inicio e o fim de ponto com o
total de horas.. Pode escolher tirar um relatório de todos os empregados ou
um a um seleccionando o empregado pretendido.
\item  Relatório de Produtos
Apresenta todos os produtos vendidos e também por família de produtos.
\item  Relatório de Vendas por Hora
Disponibiliza o total de vendas a dinheiro por hora.
\item  Relatório de Vendas por Máquina
Apresenta as vendas feitas na máquina onde se encontra.
\item  Relatório de Reservas (Cantina)
Lista as reservas realizadas onde inclui a sessão, o produto e a quantidade.
\item  Relatório de Vendas (Cantina)
Exibe a listagem de consumos a crédito realizados onde inclui a sessão a
data o produto e a quantidade.
\item  A lista de Produtos
Imprime uma lista com todos os produtos definidos.
\end{itemize}

\section{Como emitir um relatório para a contabilidade?}
No nível Gerente, através do botão Relatórios Oficiais que imprime um relatório das
vendas totais de um determinado período de tempo, este irá apresentar um conjunto
de dados necessários à contabilidade. Nos relatórios oficiais, só pode aceder a
informação relativa a sessões de trabalho encerradas. Os dados da sessão actual
nunca são incluídos.

Se esta não estiver encerrada, aparecerá uma mensagem de
alerta a dizer que a operação não pode ser concluída .


\chapter{Manutenção da Aplicação}

\section{Como criar um produto?}

No nível de Base de Dados, tocar no botão Produtos. Dentro da janela pressionar o
botão Novo produto, aparecerá um teclado virtual onde deve atribuir um nome ao
produto, este não deve conter apóstrofes pois não é compatível com o sistema.

De seguida ainda nesse mesmo separador Comum deve indicar a que Família pertence
esse produto, o Preço também deve ser indicado assim como a respectiva Taxa.

Ainda nesse mesmo separador através do botão Propriedades pode mudar a cor do
botão, a cor da letra entre outros, para se tornar mais apelativo e fácil de se
localizar. 

No separador Outros deverá indicar a Unidade do produto. Se pretender
que este produto seja pedido deverá indicar a que Impressora Virtual ele pertence
através do botão Definir, depois da impressora desejada se encontrar seleccionada.
No separador Tipo deverá indicar o Tipo de produto , se é um produto normal, extra,
entre outros. Por fim no último separador Páginas deverá indicar para que página irá
o produto criado para isso deverá seleccionar a página pretendida e fazer Adicionar.

No fim fazer OK para salvar a informação. 

Esta ultima operação de incluir produtos
nas páginas também pode ser feita através do botão Páginas, para isso deverá
aceder ao botão Páginas, dentro da janela, seleccionar a página pretendida, e
através do botão Produtos colocar o(os) produto(s) pretendidos para dentro da
página. Para terminar fazer OK. 

Também podemos definir famílias de produtos
através do botão Famílias de Produtos para isso deverá clicar no botão Nova
Família , aparecerá um teclado virtual onde deve dar um nome à família, de seguida
deverá definir as propriedades da família tal como indicado na janela. No fim fazer
OK para salvar as alterações.

\section{Como adicionar um produto a uma página?}
Através do botão Páginas no Layer de Base de Dados. Este irá abrir uma janela com
todas as páginas existentes. Seleccionar a página pretendida e clicar no botão
Produtos. Este irá abrir uma janela com todos os produtos existentes na Base de
Dados do lado esquerdo da janela. Deverá então seleccionar o produto pretendido e
clicar na seta para a direita para este ser inserido na página. No fim fazer OK para
guardar a informação.

\section{Como alterar o preço de um produto?}

Através do botão Produtos no nível de Base de Dados. Este irá abrir uma janela com
todos os produtos, deverá seleccionar o produto pretendido e pressionar em cima do
preço para que este lhe abra uma janela onde poderá apagar e colocar o novo
preço. No fim fazer OK para salvar as alterações. Também através do botão Mudar
Preço, no nível de Mesas, poderá alterar o preço, mas só se este se encontrar
activo.

\section{Como criar uma nova página?}
Através do botão Páginas no nível de Base de Dados. 
Este irá abrir uma janela com
todas as páginas existentes. Para criar uma nova terá que clicar no botão Nova
Página. De seguida surgirá um teclado virtual onde deve dar um nome à nova
página. 

Para colocar os produtos dentro dessa página terá de clicar no botão
Produtos. Este abrirá uma janela com todos os produtos existentes. Deverá
seleccionar o produto pretendido e passá-lo um a um para a página através da seta
que indica para o lado direito, se quiser tirar um produto da página terá de o
seleccionar e clicar na seta indicada para a esquerda. 

Também é permitida, através
do botão Organização página a disposição dos botões segundo o seu critério. No fim
fazer OK para salvar a informação introduzida. 

Depois de criar a página deverá ir ao
botão Índices no nível Base de Dados e colocar a página no Índice pretendido para
que a operação anterior surta efeito senão a página não aparecerá. Para isso deverá
na janela dos Índices, no separador Páginas, seleccionar o índice pretendido e
colocar a página dentro desse índice. Para salvar as alterações fazer OK.

\section{Como contar o Fundo de Maneio?}
No nível Gerente através do botão Fundo de Maneio permite fazer o controlo do
Fundo de Maneio. Este pode ser definido por máquina ou por utilizador.

\section{Como aplicar um desconto a um produto?}
Através do botão Generosidade que se encontra no nível de Gerente, podemos
aplicar descontos sobre a venda, estes podem ser aplicados sobre o total da mesa
e/ou produto a produto através de uma percentagem e/ou de um valor.

\section{Como criar um Empregado?}
Tocando no botão Empregados no nível Base de Dados, pode definir os funcionários
da loja. Para além dos dados pessoais, deve definir os tipos de acções operativas
permitidas e a integração, ou não, nas equipas de trabalho existentes.

Estas definições não alteram o perfil de utilizador, mas restringem a utilização de
funcionalidades nele disponíveis (ex: possibilidade de utilizar códigos para autentificar
clientes).

Quando pertence a uma equipa, o empregado assume por defeito as
definições dela, mas cada empregado pode ser tratado de forma independente,
através da definição das suas permissões próprias, que se sobrepõe às definições
do grupo. Além destes dados, deverá ser indicado um perfil para cada um dos
empregados.

\section{Como criar uma Equipa?}
Através do Botão Equipas no nível Base de Dados. Este irá abrir uma janela, onde
para criar uma nova Equipa terá de tocar no botão Nova Equipa e dar um nome a
esta. De seguida irá escolher as definições para essa nova Equipa. No fim fazer OK
para guardar as alterações.

\section{Como criar uma nova mesa?}

Através do botão Mesa no nível de Base de Dados. Este irá abrir uma janela onde
deverá carregar no botão Nova Mesa. Deverá dar um nome à mesa e fazer OK para
guardar o registo. Para que depois a mesa apareça, deverá ir ao botão Zonas no
nível Base de Dados e no separador Mesas deverá colocar a mesa na zona
pretendida. No fim fazer OK para guardar a informação.

\section{Como definir Sessões?}
Através do botão Sessões no nível Base de Dados. Uma sessão permite a definição
do preço pretendido por zona e é definida por períodos de tempo específicos e pelos
empregados que nela operam. Para criar uma nova sessão deverá na janela de
Sessões no nível Base de Dados clicar na opção Nova Sessão, dar um nome à nova
sessão, de seguida no separador Intervalos de Tempo deverá indicar a hora de
inicio e a hora de fim da sessão, e clicar na opção Adicionar Semana, para que este
assuma esse horário para todos os dias da semana. No fim fazer OK para salvar as
alterações.



\part{Dominando o Koncepto}

% \chapter{Introdução}              
 \chapter{Os níveis de utilização}

Como em qualquer indústria em que segmenta cada área de trabalho a cada trabalhador,
 este programa foi desenhado de forma a simplificar o mais possível o interface com
o utilizador


Desta forma, o programa adapta-se de acordo com a pessoa que utiliza o programa, facultando
a possibilidade de que a pessoa que introduz as compras tenha um perfil completamente distinto
de quem serve à mesa, que por sua vez terá um aspecto completamente diferente do supervisor
que no fim do dia pretende fazer as leituras do dia ou do mês.


O programa tem várias áreas onde acedendo a cada uma delas poderá fazer operações específicas. 
essas áreas denominam-se por \textbf{camadas} ou \textbf{níveis}.

% \includegraphics[height=3in]{../images/user.png}

Seguidamente serão apresentadas as opçõe disponíveis em cada um desses mesmos níveis.
                 


\begin{figure}
\includegraphics[width=15cm]{../images/koncepto.png}
\caption[Submanifold]{Representação gráfica do conceito de Níveis.}
\end{figure}

\begin{figure}
\includegraphics[width=15cm]{../images/konceptoTree.jpg}
\caption[Submanifold]{Árvore de configurações do programa}
\end{figure}




\chapter{Utilizador} 

O nível \textbf{utilizador} é o nível onde se inicia tudo, e é o estado nativo do programa.
De um dos lados do ecrân estarão visiveis as outras áreas do programa, que serão desbloqueadas 
consoante o perfil do funcionário que se identificar (por código numérico, leitor de cartões ou interface biomético)

Neste nível poderão estar visíveis:
\begin{itemize}
\item o relógio
\item lista de utilizadores
\item o botão desligar máquina
\item o botão fechar turno
\item o botão bloquear níveis.
\end{itemize}
\begin{figure}
\includegraphics[height=3in]{../images/user.png}
\caption[Submanifold]{Exemplo de um interface de utilizador.}
\end{figure}

\section{Bloquear Níveis}

Este botão tem uma importância vital na gestão e permite ao utilizador sair da sua área de trabalho
impedindo assim que outros utilizadores possam consultar dados que normalmente não teriam acesso.

Este botão obriga a que o próximo funcionário que queira utilizar o computador tenha de se identificar.

Esta simples metodologia permite fazer leitura correctas nos relatórios de empregados (capítulo Gerente).

\section{Fechar Turno}

Este botão permite fazer o controlo de ponto de funcionário

Ver capítulo gerente, fundo maneio
Ver capítulo gerente, relatório de empregados


\section{Desligar Máquina}

Pressionando este botão, irá aparecer uma nova caixa de diálogo a perguntar se tem realmente a certeza
de que pretende desligar o equipamento.

Se confirmar, o computador irá dar iníco ao processo de encerramento, desligando o sistema progressivamente
até ao momento em este ficará totalmente inactivo.

\begin{boxedminipage}{\textwidth}
        Caso o sistema não possua suporte a gestão de energia,
	aguarde até aparecer a mensagem \textbf{System Halted} e então será seguro desligar o equipamento.
\end{boxedminipage}



\chapter{Zona}

Entende-se por zonas as áreas que compõem o estabelecimento comercial.

Cada zona é composta por mesas e tem regras específicas no que diz respeito ao comportamento de cada zona.




\chapter{Mesa}

\chapter{Gerente}

\section{Relatórios}

\subsection{Relatórios de Contabilidade}

\begin{figure}
\includegraphics[width=15cm]{../images/diagramaRelatOficiais.jpg}
\caption[Submanifold]{Funcionamento dos Relatórios Oficiais.}
\end{figure}

No relatório de contabilidade figuram os elementos correspondentes à sessão de vendas selecionada.

Será impossível extrair o relatório de contabilidade de uma sessão actualmente em curso ou de
 um dia em que não tenha havido uma sessão a decorrer

Para uma dada sessão válida, o relatório de contabilidade irá mostrar os determinados elementos:
\begin{itemize}
\item Dados da empresa
\item Intervalo de Vendas a Dinheiro emitidas
\item Intervalo de Notas de Crédito emitidas
\item Total ilíquido facturado
\item Total líquido facturado
\item Total de IVA por taxa de Incidência
\end{itemize}

Opcionalmente pode-se adicionar outras informaçoes, como por exemplo: 
\begin{itemize}
\item Número de clientes
\item Consumo médio realizado por cada cliente.
\item etc.
\end{itemize}


\subsection{Relatórios de Gestão}

\begin{figure}
\includegraphics[width=15cm]{../images/diagramaRelatGestao.jpg}
\caption[Submanifold]{Funcionamento dos Relatórios de Gestão.}
\end{figure}

Os relatórios de gestão, não vão fazer a pesquisa por sessão, mas por horário.

Este facto poderá ser revelador de informações relacionada a períodos específicos.

Por este relatório pode-se extrair informação como:
\begin{itemize}
\item Relatório de empregado
\item Relatório de Produtos
\item (...)
\end{itemize}


\chapter{Base de Dados}

Neste nível permite-se fazer toda a configuração do comportamento do sistema e por isso o acesso
ao mesmo aos funcionários deve ser feito de uma forma criteriosa.

Aqui é possível proceder à criação e ou modificação de todas as variáveis de gestão, desde as 
características dos artigos, a forma como aparecem no interface dos funcionários, o horário
de funcionamento do sistema, os privilégios dos funcionários, gestão de cartões cliente, etc etc.
\nopagebreak
\begin{figure}
\includegraphics[height=7cm]{../images/basedados.png}
\caption[Submanifold]{O nível Base de Dados.}
\end{figure}

\newpage
\section{Famílias}

No botão Famílias, é-nos facultada a possibilidade de configurar famílias de produtos,
facilitando assim o processo de criação de artigos posteriormente.


Pode-se definir regras por grupo de artigos, como as taxas de IVA a aplicar 
em determinados produtos, descontos máximos, direccionamento das ordens de impressão, etc.


\begin{figure}
\includegraphics[height=3in]{../images/familias.png}
\caption[Submanifold]{Famílas}
\end{figure}

\textbf{NOTA:} é contudo possível parametrizar um artigo individualmente bastando para isso ir ao painel
Produtos.

Relacionado com este assunto:
\begin{itemize}
\item Tipos de Cliente
\item Relatório de Produtos
\end{itemize}

\section{Produtos}

\begin{figure}
\includegraphics[height=3in]{../images/produtos.png}
\caption[Submanifold]{Painel por defeito de Produtos.}
\end{figure}

\subsection{Comum}

No separador “Comum" é onde se encontram as principais propriedades de cada artigo, nomeadamente:
\begin{itemize}
\item Tipo de Artigo - está relacionado com a categoria emque este artigo se insere. Um artigo pode-se categorizar como:
\begin{itemize}
\item Comum - Artigo de venda normal;
\item Matéria Prima - Artigo de compra que será alvo de transformação
\item Extra de Texto - artigos que se destinam a descrever uma propriedade do artigo, para informar a produção, sem afectar o custo final do mesmo. Por ex: “mal passado”
\item Extra de Texto com preço - Artigos que complementam artigos ou grupo de artigos pré definidos e com custo associado. Por ex. “com ovo”
\item Menu de Acerto - Permite a integração de todos os artigos de uma mesa num menu único com preço pré-definido.
\item Menu pré-definido - Obriga o utilizador a escolher os componentes de uma ou mais listas pré-definidas impedindo a venda de outros artigos.
\end {itemize}
\item Família: diz respeito à família a que este artigo se insere.
\item Descrição - o nome que é impresso nos documentos
\item Nome - a designação do artigo que irá aparecer no botão - permite várias traduções.
\item Preço - apresentado com três casas décimais, permite lidar mais facilmente com balanças. Cada artigo pode ter até 7 preços diferentes, podendo ser cada um atribuido a cada secção (ver Base de Dados -> Sessões)
\item Código de Barras - Atribuindo um código ao artigo, pode-se fazer a leitura do mesmo com leitor específico, seja este um leitor de mesa, de pistola ou outro leitor similar.
\item PLU - Price Look Up Code ou código de busca de artigo, pode ser usado para invocar artigos rapidamente por teclado numérico no ponto de venda ou remotamente.
\item definição de “Pede quantidade"- quando activa irá pedir a quantidade de artigos a servir
\item definição de “Pede preço Unitário" - quando activa pergunta qual o preço do artigo que se vai servir.
\end {itemize}

\subsection{Outros}
\begin{figure}
\includegraphics[height=3in]{../images/produtos2.png}
\caption[Submanifold]{Tabulador Outros de Produtos.}
\end{figure}


Aqui define-se as características dos artigos, tais como a unidade de compra e destino da ordem de produção

\subsection{Relações}
\begin{figure}
\includegraphics[height=3in]{../images/produtos3.png}
\caption[Submanifold]{Tabulador Relações de Produtos.}
\end{figure}


Aqui configuram-se os produtos unidos para e os artigos que devem figurar em menu (caso o produto seja do tipo menu pré-definido)

\subsection{Página}
\begin{figure}
\includegraphics[height=3in]{../images/produtos4.png}
\caption[Submanifold]{Tabulador Páginas de Produtos.}
\end{figure}


Aqui definem-se a página ou as páginas onde o artigo deve estar apresentado.
é possível o mesmo artigo estar apresentado em diversos locais. Por exemplo o artigo café, pode-se configurar para poder invocado nas páginas “rápidas”, “cafetaria” ou “sobremesas”.

itens relacionados: páginas, famílias, sessões, zonas, reservas, relatório de produtos.

\section{Clientes}

\begin{figure}
\includegraphics[width=15cm]{../images/diagramaClientes.jpg}
\caption[Submanifold]{Estrutura de cliente especial.}
\end{figure}


A secção clientes diz respeito aos clientes "especiais". 

Entende-se por cliente especial, todo aquele que é identificado e reconhecido pelo sistema. 

Um cliente especial poder ter associado regras de consumo, tais como:
\begin{itemize}
\item ter as suas despesas transformadas em consumo a crédito (para pagamento posterior)
\item ter descontos em determinados artigos
\item não ter qualquer regra de pagamento, mas no entanto aparecem os seus elementos no documento de venda.
\end{itemize}

Para além disto, é possível no caso de haver descontos impor restrições de quantidade de artigos consumidos
por dia, impedindo assim situações de possível abuso.

itens relacionados: Tipos de cliente, tipos de cartão.

\section{Feriados}

Aqui configuram-se as partes referentes aos intervalos em que as reservas de cantinas não devem funcionar.

Considerando os dias úteis definidos em sessões, esta função permite criar as excepções em que as cantinas não funcionam.

itens relacionados: Sessões, reservas, estado das reservas.

\section{Sessões}
Em sessões pode-se definir os turnos de trabalho que se fazem por dia. As sessões podem ser parametrizadas para serem manuais ou automáticas 
na secção de \textbf{Definições Globais}.

Nas sessões pode-se definir:
\begin{itemize}
\item Os horários de funcionamento de cada turno;
\item O preço a aplicar em cada zona em cada turnos;
\item Os funcionários por turno;
\end{itemize}

itens relacionados: Produtos.


\section{Perfis}

Aqui se associam perfis gráficos a utilizadores e máquinas.

Aqui pode-se determinar a língua, zonas e permissões que ficam vinculadas a cada utilizador ou máquina.

itens relacionados: zonas, sessões, máquinas, reservas.

\chapter{Aplicação}

\chapter{Técnico}

\begin{figure}
\includegraphics[width=15cm]{../images/diagramaMaquinas.jpg}
\caption[Submanifold]{Organização de hardware.}
\end{figure}

\section{Máquinas}

Aqui definem-se os pontos de venda  e respectivos IPs.

Neste ponto pode-se associar um perfil gráfico a uma máquina.

Pode-se ainda criar as parametrizações de X64 que funcionam usando PLUs dos artigos.

Pode-se por exemplo associar a regra de que a máquina "Restaurante" tem somente acesso às mesas da zona restaurante. 

itens relacionados: perfis, produtos.

\section{Dispositivos}

Aqui definem-se todos os periféricos e definem-se onde ficam ligados a cada equipamento.

Pode-se criar impressoras, \emph{displays} de cliente, balanças, gavetas de valores, leitores vários, etc

Este quadro é separado por quatro quadros relacionados: 
\begin{itemize}
\item máquinas
\item dispositivo
\item tipo de ligação
\item configuração da ligação.
\end{itemize}

As parametrizações de baixo nível, tais como configurações de porta série são feitas a nível de sistema operativo.

selecionando um periférico, pode-se alterar o nome do mesmo, e caso seja um dispositivo de entrada --- como é o caso 
de um leitor biométrico --- é possível criar a expressão regular para validar o valor lido.

Caso haja um sistema de entrada de interface PS2 (entrada convencionalmente utilizada pelo teclado), 
a expressão regular deverá ser utilizada em \textbf{Definições Globais}, afectando todos os equipamentos da rede.

itens relacionados: Definições Globais.

\section{Impressoras Virtuais}

Com impressoras virtuais entende-se como o papel que uma impressora toma no desenrolar das suas tarefas num espaço comercial.

Pode ser cómodo designar uma impressora com um nome específico, criando um nível de abstracção do sistema informático
permitindo ao gerente fazer o redireccionamento de pedidos de uma forma expedita sem quaisquer conhecimentos técnicos.

itens relacionados: Ligação de Impressoras, Dispositivos, Impressão de Documentos.

\section{Ligação de Impressoras}

A partir daqui pode-se definir que impressoras imprimem consoante o local em que nos encontramos.

Pode-se definir que se estou num local de transformação, e se fizer um pedido de um produto transformado neste local,
não faz sentido que haja emissão de papel.

Por outro lado se estiver longe do local de transformação, deverá sair sempre um pedido na zona de transformação, 
identificando o local para onde se destina e o funcionário que fez a requisição.

itens relacionados: Impressoras Virtuais, Dispositivos, Impressão de Documentos.

\section{Impressão de Documentos}

Pode-se aqui definir em que impressora deverá ser impresso cada documento.

itens relacionados: Impressoras Virtuais, Ligação de Impressoras, Dispositivos.

\section{Upload de logo para Impressora}

Para exportar uma imagem para a impressora, é necessário satisfazer as seguintes condições:
\begin{itemize}
\item a imagem esteja em formato \emph{bmp};
\item a imagem  esteja a duas cores;
\item o ficheiro esteja no directório apropriado do software.
\end{itemize}


\chapter{Reservas}


\end{document}
